\documentclass[onlymath]{beamer}
% \documentclass[onlymath,handout]{beamer}

% Macros used by all lectures, but not necessarily by excercises

%%% General setup and dependencies:

% \usetheme[ddcfooter,nosectionnum]{tud}
\usetheme[nosectionnum,pagenum,noheader]{tud}
% \usetheme[nosectionnum,pagenum]{tud}
\setbeamertemplate{enumerate items}[default]

% Increase body font size to a sane level:
\let\origframetitle\frametitle
% \renewcommand{\frametitle}[1]{\origframetitle{#1}\normalsize}
\renewcommand{\frametitle}[1]{\origframetitle{#1}\fontsize{10pt}{13.2}\selectfont}
\setbeamerfont{itemize/enumerate subbody}{size=\small} % tud defaults to scriptsize!
\setbeamerfont{itemize/enumerate subsubbody}{size=\small}
% \setbeamerfont{normal text}{size=\small}
% \setbeamerfont{itemize body}{size=\small}

\renewcommand{\emph}[1]{\textbf{#1}}

\def\arraystretch{1.3}% Make tables even less cramped vertically

\usepackage[ngerman]{babel}
\usepackage[utf8]{inputenc}
\usepackage[T1]{fontenc}

%\usepackage{graphicx}
\usepackage[export]{adjustbox} % loads graphicx
\usepackage{import}
\usepackage{stmaryrd}
\usepackage[normalem]{ulem} % sout command
% \usepackage{times}
\usepackage{txfonts}
\usepackage{array}

% \usepackage[perpage]{footmisc} % reset footnote counter on each page -- fails with beamer (footnotes gone)
\usepackage{perpage}  % reset footnote counter on each page
\MakePerPage{footnote}

\usepackage{tikz}
\usetikzlibrary{arrows,positioning,decorations.pathreplacing}
% Inspired by http://www.texample.net/tikz/examples/hand-drawn-lines/
\usetikzlibrary{decorations.pathmorphing}
\pgfdeclaredecoration{penciline}{initial}{
    \state{initial}[width=+\pgfdecoratedinputsegmentremainingdistance,
    auto corner on length=1mm,]{
        \pgfpathcurveto%
        {% From
            \pgfqpoint{\pgfdecoratedinputsegmentremainingdistance}
                      {\pgfdecorationsegmentamplitude}
        }
        {%  Control 1
        \pgfmathrand
        \pgfpointadd{\pgfqpoint{\pgfdecoratedinputsegmentremainingdistance}{0pt}}
                    {\pgfqpoint{-\pgfdecorationsegmentaspect
                     \pgfdecoratedinputsegmentremainingdistance}%
                               {\pgfmathresult\pgfdecorationsegmentamplitude}
                    }
        }
        {%TO 
        \pgfpointadd{\pgfpointdecoratedinputsegmentlast}{\pgfpoint{1pt}{1pt}}
        }
    }
    \state{final}{}
}
\tikzset{handdrawn/.style={decorate,decoration=penciline}}
\tikzset{every shadow/.style={fill=none,shadow xshift=0pt,shadow yshift=0pt}}
% \tikzset{module/.append style={top color=\col,bottom color=\col}}

% Use to make Tikz attributes with Beamer overlays
% http://tex.stackexchange.com/a/6155
\tikzset{onslide/.code args={<#1>#2}{%
  \only<#1| handout:0>{\pgfkeysalso{#2}} 
}}
\tikzset{onslideprint/.code args={<#1>#2}{%
  \only<#1>{\pgfkeysalso{#2}} 
}}

%%% Title -- always set this first

\newcommand{\defineTitle}[3]{
	\newcommand{\lectureindex}{#1}
	\title{Formale Systeme}
	\subtitle{\href{\lectureurl}{#1. Vorlesung: #2}}
	\author{\href{https://iccl.inf.tu-dresden.de/web/Markus_Kr\%C3\%B6tzsch}{Markus Kr\"{o}tzsch}\\[1ex]Professur für Wissensbasierte Systeme}
	\date{#3}
	\datecity{TU Dresden}
% 	\institute{CC-By 3.0, sofern keine anderslautenden Bildrechte angegeben sind}
}

%%% Table of contents:

\RequirePackage{ifthen}

\newcommand{\highlight}[2]{%
	\ifthenelse{\equal{#1}{\lectureindex}}{\alert{#2}}{#2}%
}

\def\myspace{-0.7ex}
\newcommand{\printtoc}{
\begin{tabular}{r@{$\quad$}l}
\highlight{1}{1.} & \highlight{1}{Willkommen/Einleitung formale Sprachen}\\[\myspace]
\highlight{2}{2.} & \highlight{2}{Grammatiken und die Chomsky-Hierarchie}\\[\myspace]
\highlight{3}{3.} & \highlight{3}{Endliche Automaten}\\[\myspace]
\highlight{4}{4.} & \highlight{4}{Complexity of FO query answering}\\[\myspace]
\highlight{5}{5.} & \highlight{5}{Conjunctive queries}\\[\myspace]
\highlight{6}{6.} & \highlight{6}{Tree-like conjunctive queries}\\[\myspace]
\highlight{7}{7.} & \highlight{7}{Query optimisation}\\[\myspace]
\highlight{8}{8.} & \highlight{8}{Conjunctive Query Optimisation / First-Order~Expressiveness}\\[\myspace]
\highlight{9}{9.} & \highlight{9}{First-Order~Expressiveness / Introduction to Datalog}\\[\myspace]
\highlight{10}{10.} & \highlight{10}{Expressive Power and Complexity of Datalog}\\[\myspace]
\highlight{11}{11.} & \highlight{11}{Optimisation and Evaluation of Datalog}\\[\myspace]
\highlight{12}{12.} & \highlight{12}{Evaluation of Datalog (2)}\\[\myspace]
\highlight{13}{13.} & \highlight{13}{Graph Databases and Path Queries}\\[\myspace]
\highlight{14}{14.} & \highlight{14}{Outlook: database theory in practice}
\end{tabular}
}

\newcommand{\overviewslide}{%
\begin{frame}\frametitle{Overview}
\printtoc
\medskip

Siehe \href{\lectureurl}{course homepage [$\Rightarrow$ link]} for more information and materials
\end{frame}
}

%%% Colours:
\usepackage{xcolor,colortbl}
\definecolor{redhighlights}{HTML}{FFAA66}
\definecolor{lightblue}{HTML}{55AAFF}
\definecolor{lightred}{HTML}{FF5522}
\definecolor{lightpurple}{HTML}{DD77BB}
\definecolor{lightgreen}{HTML}{55FF55}
\definecolor{darkred}{HTML}{CC4411}
\definecolor{darkblue}{HTML}{176FC0}%{1133AA}
\definecolor{nightblue}{HTML}{2010A0}%{1133AA}
\definecolor{alert}{HTML}{176FC0}
\definecolor{darkgreen}{HTML}{36AB14}
\definecolor{strongyellow}{HTML}{FFE219}
\definecolor{devilscss}{HTML}{666666}

\newcommand{\redalert}[1]{\textcolor{darkred}{#1}}

%%% Style commands

\newcommand{\quoted}[1]{\texttt{"}{#1}\texttt{"}}
\newcommand{\squote}{\texttt{"}} % straight quote
\newcommand{\Sterm}[1]{\ensuremath{\mathtt{\textcolor{purple}{#1}}}}    % letters in alphabets
\newcommand{\Snterm}[1]{\textsf{\textcolor{darkblue}{#1}}} % nonterminal symbols
\newcommand{\Sntermsub}[2]{\ensuremath{\Snterm{#1}_{\Snterm{#2}}}} % nonterminal symbols
\newcommand{\Slang}[1]{\textbf{\textcolor{black}{#1}}}    % languages
\newcommand{\Slangsub}[2]{\ensuremath{\Slang{#1}_{\Slang{#2}}}}    % languages
% Code
\newcommand{\Scode}[1]{\textbf{#1}}    % reserved words in program listings, e.g., "if"
\newcommand{\Scodelit}[1]{\textcolor{purple}{#1}}    % literals in program listings, e.g., strings
\newcommand{\Scomment}[1]{\textcolor{gray}{#1}}    % comment in program listings

\newcommand{\epstrastar}{\mathrel{\mathord{\stackrel{\epsilon}{\to}}{}^*}} % transitive reflexive closure of epsilon transitions in an epslion-NFA

\newcommand{\narrowcentering}[1]{\mbox{}\hfill#1\hfill\mbox{}}

\newcommand{\Smach}[1]{\ensuremath{\mathcal{#1}}}    % machines

\newcommand{\mytrue}{\Scodelit{1}}
\newcommand{\myfalse}{\Scodelit{0}}
% \newcommand{\emptyClause}{\bot}

\newcommand{\Scomplclass}[1]{{\textsc{#1}}} % font for complexity classes, used on slides where the "too many alphabets" LaTeX error appears when using the correct sc font :-(
% \newcommand{\complclass}[1]{\ensuremath{\mathsc{#1}}} % font for complexity classes

%%% Slide layout commands:

\newcommand{\sectionSlide}[1]{
\frame{\begin{center}
\LARGE
#1
\end{center}}
}
\newcommand{\sectionSlideNoHandout}[1]{
\frame<handout:0>{\begin{center}
\LARGE
#1
\end{center}}
}

\newcommand{\mydualbox}[3]{%
 \begin{minipage}[t]{#1}
 \begin{beamerboxesrounded}[upper=block title,lower=block body,shadow=true]%
    {\centering\usebeamerfont*{block title}#2}%
    \raggedright%
    \usebeamerfont{block body}
%     \small
    #3%
  \end{beamerboxesrounded}
  \end{minipage}
}
% 
\newcommand{\myheaderbox}[2]{%
 \begin{minipage}[t]{#1}
 \begin{beamerboxesrounded}[upper=block title,lower=block title,shadow=true]%
    {\centering\usebeamerfont*{block title}\rule{0pt}{2.6ex} #2}%
  \end{beamerboxesrounded}
  \end{minipage}
}

\newcommand{\mycontentbox}[2]{%
 \begin{minipage}[t]{#1}%
 \begin{beamerboxesrounded}[upper=block body,lower=block body,shadow=true]%
    {\centering\usebeamerfont*{block body}\rule{0pt}{2.6ex}#2}%
  \end{beamerboxesrounded}
  \end{minipage}
}

\newcommand{\mylcontentbox}[2]{%
 \begin{minipage}[t]{#1}%
 \begin{beamerboxesrounded}[upper=block body,lower=block body,shadow=true]%
    {\flushleft\usebeamerfont*{block body}\rule{0pt}{2.6ex}#2}%
  \end{beamerboxesrounded}
  \end{minipage}
}

% label=180:{\rotatebox{90}{{\footnotesize\textcolor{darkgreen}{Beispiel}}}}
% \hspace{-8mm}\ghost{\raisebox{-7mm}{\rotatebox{90}{{\footnotesize\textcolor{darkgreen}{Beispiel}}}}}\hspace{8mm}
\newcommand{\examplebox}[1]{%
	\begin{tikzpicture}[decoration=penciline, decorate]
		\pgfmathsetseed{1235}
		\node (n1) [decorate,draw=darkgreen, fill=darkgreen!10,thick,align=left,text width=\linewidth, inner ysep=2mm, inner xsep=2mm] at (0,0) {#1};
% 		\node (n2) [align=left,text width=\linewidth,inner sep=0mm] at (n1.92) {{\footnotesize\raisebox{3mm}{\textcolor{darkgreen}{Beispiel}}}};
% 		\node (n2) [decorate,draw=darkgreen, fill=darkgreen!10,thick, align=left,text width=\linewidth,inner sep=2mm] at (n1.90) {{\footnotesize\raisebox{0mm}{\textcolor{darkgreen}{Beispiel}}}};
	\end{tikzpicture}%
}%

\newcommand{\codebox}[1]{%
	\begin{tikzpicture}[decoration=penciline, decorate]
		\pgfmathsetseed{1236}
		\node (n1) [decorate,draw=strongyellow, fill=strongyellow!10,thick,align=left,text width=\linewidth, inner ysep=2mm, inner xsep=2mm] at (0,0) {#1};
	\end{tikzpicture}%
}%

\newcommand{\defbox}[1]{%
	\begin{tikzpicture}[decoration=penciline, decorate]
		\pgfmathsetseed{1237}
		\node (n1) [decorate,draw=darkred, fill=darkred!10,thick,align=left,text width=\linewidth, inner ysep=2mm, inner xsep=2mm] at (0,0) {#1};
	\end{tikzpicture}%
}%

\newcommand{\theobox}[1]{%
	\begin{tikzpicture}[decoration=penciline, decorate]
		\pgfmathsetseed{1240}
		\node (n1) [decorate,draw=darkblue, fill=darkblue!10,thick,align=left,text width=\linewidth, inner ysep=2mm, inner xsep=2mm] at (0,0) {#1};
	\end{tikzpicture}%
}%

\newcommand{\anybox}[2]{%
	\begin{tikzpicture}[decoration=penciline, decorate]
		\pgfmathsetseed{1240}
		\node (n1) [decorate,draw=#1, fill=#1!10,thick,align=left,text width=\linewidth, inner ysep=2mm, inner xsep=2mm] at (0,0) {#2};
	\end{tikzpicture}%
}%


\newsavebox{\mybox}%
\newcommand{\doodlebox}[2]{%
\sbox{\mybox}{#2}%
	\begin{tikzpicture}[decoration=penciline, decorate]
		\pgfmathsetseed{1238}
		\node (n1) [decorate,draw=#1, fill=#1!10,thick,align=left,inner sep=1mm] at (0,0) {\usebox{\mybox}};
	\end{tikzpicture}%
}%
\newcommand{\widedoodlebox}[2]{%
\sbox{\mybox}{#2}%
	\begin{tikzpicture}[decoration=penciline, decorate]
		\pgfmathsetseed{1238}
		\node (n1) [decorate,draw=#1, fill=#1!10,thick,align=left,inner sep=1mm,text width=\linewidth] at (0,0) {\usebox{\mybox}};
	\end{tikzpicture}%
}%

% Common notation

\usepackage{amsmath,amssymb,amsfonts}
\usepackage{xspace}

\newcommand{\lectureurl}{https://iccl.inf.tu-dresden.de/web/FS2023}

\DeclareMathAlphabet{\mathsc}{OT1}{cmr}{m}{sc} % Let's have \mathsc since the slide style has no working \textsc

% Dual of "phantom": make a text that is visible but intangible
\newcommand{\ghost}[1]{\raisebox{0pt}[0pt][0pt]{\makebox[0pt][l]{#1}}}

\newcommand{\tuple}[1]{\langle{#1}\rangle}
\newcommand{\defeq}{\mathrel{:=}}

%%% Annotation %%%

\usepackage{color}
\newcommand{\todo}[1]{{\tiny\color{red}\textbf{TODO: #1}}}



%%% Old macros below; move when needed

\newcommand{\blank}{\text{\textvisiblespace}} % empty tape cell for TM

% table syntax
\newcommand{\dom}{\textbf{dom}}
\newcommand{\adom}{\textbf{adom}}
\newcommand{\dbconst}[1]{\texttt{"#1"}}
\newcommand{\pred}[1]{\textsf{#1}}
\newcommand{\foquery}[2]{#2[#1]}
\newcommand{\ground}[1]{\textsf{ground}(#1)}
% \newcommand{\foquery}[2]{\{#1\mid #2\}} %% Notation as used in Alice Book
% \newcommand{\foquery}[2]{\tuple{#1\mid #2}}

\newcommand{\quantor}{\mathord{\reflectbox{$\text{\sf{Q}}$}}} % the generic quantor

% logic syntax
\newcommand{\Inter}{\mathcal{I}} %used to denote an interpretation
\newcommand{\Jnter}{\mathcal{J}} %used to denote another interpretation
\newcommand{\Knter}{\mathcal{K}} %used to denote yet another interpretation
\newcommand{\Zuweisung}{\mathcal{Z}} %used to denote a variable assignment

% query languages
\newcommand{\qlang}[1]{{\sf #1}} % Font for query languages
\newcommand{\qmaps}[1]{\textbf{QM}({\sf #1})} % Set of query mappings for a query language

%%% Complexities %%%

\hyphenation{Exp-Time} % prevent "Ex-PTime" (see, e.g. Tobies'01, Glimm'07 ;-)
\hyphenation{NExp-Time} % better that than something else

% \newcommand{\complclass}[1]{{\sc #1}\xspace} % font for complexity classes
\newcommand{\complclass}[1]{\ensuremath{\mathsc{#1}}\xspace} % font for complexity classes

\newcommand{\ACzero}{\complclass{AC$_0$}}
\newcommand{\LogSpace}{\complclass{L}}
\newcommand{\NLogSpace}{\complclass{NL}}
\newcommand{\PTime}{\complclass{P}}
\newcommand{\NP}{\complclass{NP}}
\newcommand{\coNP}{\complclass{coNP}}
\newcommand{\PH}{\complclass{PH}}
\newcommand{\PSpace}{\complclass{PSpace}}
\newcommand{\NPSpace}{\complclass{NPSpace}}
\newcommand{\ExpTime}{\complclass{ExpTime}}
\newcommand{\NExpTime}{\complclass{NExpTime}}
\newcommand{\ExpSpace}{\complclass{ExpSpace}}
\newcommand{\TwoExpTime}{\complclass{2ExpTime}}
\newcommand{\NTwoExpTime}{\complclass{N2ExpTime}}
\newcommand{\ThreeExpTime}{\complclass{3ExpTime}}
\newcommand{\kExpTime}[1]{\complclass{#1ExpTime}}
\newcommand{\kExpSpace}[1]{\complclass{#1ExpSpace}}


\defineTitle{13}{Das Pumping Lemma kontextfreier Sprachen}{24. November 2016}

\begin{document}

\maketitle

\sectionSlideNoHandout{Rückblick: Der CYK-Algorithmus}

\newcommand{\Rightarrowstarquest}{\mathrel{{\stackrel{?}{\Rightarrow}}{}^*}}

\begin{frame}\frametitle{CYK: Grundidee}

Der CYK-Algorithmus arbeitet mit einer kontextfreien Grammatik $G$ in CNF.
\bigskip

\alert{Wie kann man prüfen, ob ein Wort $w=\Sterm{a_1}\cdots\Sterm{a_n}$ durch so eine Grammatik abgeleitet werden kann?}


\begin{itemize}
\item Falls $|w|=1$, dann ist $w\in\Sigma$ und es gilt:\\
$w\in\Slang{L}(G)$ genau dann wenn es eine Regel $\Snterm{S}\to w$ in $G$ gibt
\item Falls $|w|>1$, dann ist:\\
$w\in\Slang{L}(G)$ genau dann wenn es eine Regel
$\Snterm{S}\to\Snterm{A}\Snterm{B}$ und eine Zahl $i$ gibt, so dass gilt
\[\Snterm{A}\Rightarrow^* \Sterm{a_1}\cdots\Sterm{a_i}\qquad\text{und}\qquad \Snterm{B}\Rightarrow^* \Sterm{a_{i+1}}\cdots\Sterm{a_n}\]
\end{itemize}

\emph{Idee:} Fall 2 reduziert das Problem $\Snterm{S}\Rightarrowstarquest w$ in zwei einfachere Probleme
$\Snterm{A}\Rightarrowstarquest \Sterm{a_1}\cdots\Sterm{a_i}$ und $\Snterm{B}\Rightarrowstarquest \Sterm{a_{i+1}}\cdots\Sterm{a_n}$, die man allerdings für alle Regeln $\Snterm{S}\to\Snterm{A}\Snterm{B}$ und Indizes $i$ lösen muss

\end{frame}

\begin{frame}\frametitle{CYK: Praktische Umsetzung}

\anybox{purple}{Notation: Für $w=\Sterm{a_1}\cdots\Sterm{a_n}$ schreiben wir $w_{i,j}$ für das
Teilwort $\Sterm{a_i}\cdots\Sterm{a_{j}}$, also das Infix der Länge $j-i+1$, welches an Position $i$ beginnt.
}

\emph{Vorgehen:} Wir berechnen für jedes Teilwort $w_{i,j}$ die Menge aller Variablen $\Snterm{A}$, für
die gilt $\Snterm{A}\Rightarrow^* w_{i,j}$. Diese Menge nennen wir $V[i,j]$.
\begin{itemize}
\item Wir beginnen mit den kürzesten Teilwörtern ($i=j$)
\item Für längere Wörter betrachten wir jede mögliche Zweiteilung $w_{i,\ell}=w_{i,k}w_{k+1,j}$ und suchen Regeln der Form $\Snterm{A}\to\Snterm{B}\Snterm{C}$ so dass $\Snterm{B}\in V[i,k]$ und $\Snterm{C}\in V[k+1,j]$
\end{itemize}
Ist am Ende das Startsymbol $S\in V[1,|w|]$, dann liegt $w$ in der Sprache

\end{frame}

\newcommand{\MEC}[1]{\multicolumn{1}{c}{#1}}
\begin{frame}\frametitle{Notation}

% Je größer $i$, desto kürzer die mögliche Wortlänge $\ell$, denn $i+\ell \leq |w|$\\
Für die Teilwörter $w_{i,j}$ gilt $i\leq j$\\
$\leadsto$ Die Mengen $V[i,j]$ können als Dreiecksmatrix notiert werden
\bigskip

\emph{Beispiel:} Wir betrachten das Wort $w=\Sterm{a+b*c}$ der Länge $|w|=5$.\bigskip

Darstellung der Mengen $V[i,j]$:\bigskip

\narrowcentering{%
\begin{tabular}{c|c|c|c|c|c|}
%  \multicolumn{1}{c}{}  & \multicolumn{1}{c}{A} & \multicolumn{1}{c}{B} & \multicolumn{1}{c}{C} & \multicolumn{1}{c}{D}  \\\cline{2-5}
\cline{2-6}
\Sterm{a}
	& $V[1,1]$ & $V[1,2]$ & $V[1,3]$ & $V[1,4]$ & $V[1,5]$ \\\cline{2-6}
\MEC{\Sterm{+}} &  
	           & $V[2,2]$ & $V[2,3]$ & $V[2,4]$ & $V[2,5]$ \\\cline{3-6}
 \MEC{\Sterm{b}} &\MEC{}   &
	                      & $V[3,3]$ & $V[3,4]$ & $V[3,5]$ \\\cline{4-6}
 \MEC{\Sterm{*}} &\MEC{}   &\MEC{}   &   
	                                 & $V[4,4]$ & $V[4,5]$ \\\cline{5-6}
 \MEC{\Sterm{c}} &\MEC{}   &\MEC{}   &\MEC{}   &
	                                            & $V[5,5]$ \\\cline{6-6}
 \MEC{}          &\MEC{\Sterm{a}} & \MEC{\Sterm{+}} & \MEC{\Sterm{b}} & \MEC{\Sterm{*}} & \MEC{\Sterm{c}}
\end{tabular}}

\end{frame}

\newcommand{\hicell}[1]{\only<#1|handout:0>{\cellcolor{strongyellow}}}
\newcommand{\upcell}[2]{\only<#1|handout:0>{\cellcolor{strongyellow}}\visible<#1->{#2}}
\newcommand{\locell}[1]{\only<#1|handout:0>{\cellcolor{darkgreen!20}}}
\newcommand{\hicellstatic}[1]{}
\newcommand{\upcellstatic}[2]{#2}
\newcommand{\locellstatic}[1]{}

\begin{frame}\frametitle{Beispiel}

Grammatik:\\[1ex]
$
\Snterm{S}\to \Snterm{S}\Snterm{A}\mid \Snterm{S}\Snterm{M} \mid \Sterm{a}\mid \Sterm{b}\mid \Sterm{c}$\\
$\Snterm{A}\to \Snterm{P}\Snterm{S} \qquad \Snterm{M} \to \Snterm{T}\Snterm{S}\qquad
\Snterm{P}\to \Sterm{+} \qquad \Snterm{T} \to \Sterm{*}
% \\
% \Snterm{N} \to \Snterm{P}\Snterm{Z}\qquad \Snterm{Z}\to \Sterm{a}\mid \Sterm{b}\mid \Sterm{c}
$\\[2ex]
Wort: $w=\Sterm{a+b*c}$
\bigskip

Berechnung der Mengen $V[i,j]$:\bigskip

\narrowcentering{%
\begin{tabular}{c|c|c|c|c|c|}
%  \multicolumn{1}{c}{}  & \multicolumn{1}{c}{A} & \multicolumn{1}{c}{B} & \multicolumn{1}{c}{C} & \multicolumn{1}{c}{D}  \\\cline{2-5}
\cline{2-6}
\Sterm{a}
	& \hicellstatic{2}\upcellstatic{3}{$\Snterm{S}$}\locellstatic{10,20,21,31,40-41}
		& \hicellstatic{9,10}\locellstatic{22,32,42}
			& \hicellstatic{19,20,22}\upcellstatic{21}{$\Snterm{S}$}\locellstatic{33,43}
				& \hicellstatic{30-33}\locellstatic{44}
					& \hicellstatic{39-43}\upcellstatic{41}{$\Snterm{S}$} \\\cline{2-6}
\MEC{\Sterm{+}} &  
		& \hicellstatic{4}\upcellstatic{5}{$\Snterm{P}$}\locellstatic{10,12-13,24,35-36}
			& \hicellstatic{11,12}\upcellstatic{13}{$\Snterm{A}$}\locellstatic{20-21,25,37}
				& \hicellstatic{23-25}\locellstatic{31,38}
					& \hicellstatic{34-38}\upcellstatic{36}{$\Snterm{A}$}\locellstatic{40-41} \\\cline{3-6}
 \MEC{\Sterm{b}} &\MEC{}   &
			& \upcellstatic{6}{$\Snterm{S}$}\locellstatic{12,13,15,22,27-28}
				&  \hicellstatic{14,15}\locellstatic{24,29,32}
					& \hicellstatic{26-29}\upcellstatic{28}{$\Snterm{S}$}\locellstatic{35-36,42} \\\cline{4-6}
 \MEC{\Sterm{*}} &\MEC{}   &\MEC{}   &   
				& \upcellstatic{7}{$\Snterm{T}$}\locellstatic{15,17,18,25,33}
					& \hicellstatic{16,17}\upcellstatic{18}{$\Snterm{M}$}\locellstatic{27-28,37,43} \\\cline{5-6}
 \MEC{\Sterm{c}} &\MEC{}   &\MEC{}   &\MEC{}   &
					& \upcellstatic{8}{$\Snterm{S}$}\locellstatic{17,18,29,38,44} \\\cline{6-6}
 \MEC{}          &\MEC{\Sterm{a}} & \MEC{\Sterm{+}} & \MEC{\Sterm{b}} & \MEC{\Sterm{*}} & \MEC{\Sterm{c}}
\end{tabular}}\bigskip

\narrowcentering{\visible<1->{$S\in V[1,5]$ $\leadsto$ \redalert{$w$ kann erzeugt werden}}}

\end{frame}

\begin{frame}\frametitle{Der CYK-Algorithmus}

\codebox{
\emph{Eingabe:} Wort $w=\Sterm{a_1}\cdots\Sterm{a_n}$, CNF-Grammatik $G=\tuple{V,\Sigma,P, \Snterm{S}}$\\
\emph{Ausgabe:} "`ja"' wenn $w\in\Slang{L}(G)$; sonst "`nein"'\\[1ex]
%
% Grundfall:\\
\emph{for} $i=1,\ldots,n$:~~~\textcolor{devilscss}{// Teilwörter der Länge 1}\\
~~~$V[i,i]\defeq \{\Snterm{A}\in V\mid \Snterm{A}\to\Sterm{a_i}\in P\}$\\[2ex]
\emph{for} $d=1,\ldots,n-1$:~~~\textcolor{devilscss}{// Differenz Endindex $-$ Startindex}\\
~~~\emph{for} $i=1,\ldots,n-d$:~~~\textcolor{devilscss}{// Startindex}\\
~~~~~~$j\defeq i+d$\\
~~~~~~$V[i,j]\defeq \emptyset$\\
~~~~~~\emph{for} $k=i,\ldots,j-1$:~~~\textcolor{devilscss}{// Trennindex}\\
~~~~~~~~~$V[i,j]\defeq V[i,j]\cup \{\Snterm{A}\in V\mid \text{es gibt }\Snterm{A}\to\Snterm{B}\Snterm{C}\in P\text{ mit}$\\
~~~~~~~~~\hspace{3.8cm}$\Snterm{B}\in V[i,k]\text{ und }\Snterm{C}\in V[k+1,j]\}$\\[2ex]
\emph{return} $\Snterm{S}\in V[1,n]$ \emph{?} "`ja"' \emph{:} "`nein"'
}

\end{frame}

\begin{frame}\frametitle{Bezug zur Tabelle}

\scalebox{0.7}{%
\begin{minipage}{8cm}\codebox{
\emph{Eingabe:} Wort $w=\Sterm{a_1}\cdots\Sterm{a_n}$,\\
\hspace{2cm}CNF-Grammatik $G=\tuple{V,\Sigma,P, \Snterm{S}}$\\
\emph{Ausgabe:} "`ja"' wenn $w\in\Slang{L}(G)$; sonst "`nein"'\\[1ex]
%
% Grundfall:\\
\ghost{\hspace{8.3cm}{\huge$\Leftarrow$}
\begin{tabular}{c|c|c|c|c|c|}
%  \multicolumn{1}{c}{}  & \multicolumn{1}{c}{A} & \multicolumn{1}{c}{B} & \multicolumn{1}{c}{C} & \multicolumn{1}{c}{D}  \\\cline{2-5}
\cline{2-6}
\Sterm{a}
	& \cellcolor{purple!20}  &   &   &   &   \\\cline{2-6}
\MEC{\Sterm{+}} &  
	& \cellcolor{purple!40}  &   &   &   \\\cline{3-6}
 \MEC{\Sterm{b}} &\MEC{}   &
	& \cellcolor{purple!60}  &   &   \\\cline{4-6}
 \MEC{\Sterm{*}} &\MEC{}   &\MEC{}   &   
	& \cellcolor{purple!80}  &  \\\cline{5-6}
 \MEC{\Sterm{c}} &\MEC{}   &\MEC{}   &\MEC{}   &
	& \cellcolor{purple!100}  \\\cline{6-6}
 \MEC{}          &\MEC{\Sterm{a}} & \MEC{\Sterm{+}} & \MEC{\Sterm{b}} & \MEC{\Sterm{*}} & \MEC{\Sterm{c}}
\end{tabular}
}%
\emph{for} $i=1,\ldots,n$:~~~\textcolor{devilscss}{// Teilwörter der Länge 1}\\
~~~$V[i,i]\defeq \{\Snterm{A}\in V\mid \Snterm{A}\to\Sterm{a_i}\in P\}$\\[2ex]
\emph{for} $d=1,\ldots,n-1$:~~~\textcolor{devilscss}{// Differenz Endind. $-$ Startind.}\\
~~~\emph{for} $i=1,\ldots,n-d$:~~~\textcolor{devilscss}{// Startindex}\\
~~~~~~$j\defeq i+d$\\
~~~~~~$V[i,j]\defeq \emptyset$\\
~~~~~~\emph{for} $k=i,\ldots,j-1$:~~~\textcolor{devilscss}{// Trennindex}\\
~~~~~~~~~$V[i,j]\defeq V[i,j]\cup{}$\\
~~~~~~~~~\hspace{1.4cm}$\{\Snterm{A}\in V\mid \text{es gibt }\Snterm{A}\to\Snterm{B}\Snterm{C}\in P\text{ mit}$\\
~~~~~~~~~\hspace{2cm}$\Snterm{B}\in V[i,k]\text{ und }\Snterm{C}\in V[k+1,j]\}$\\[2ex]
\emph{return} $\Snterm{S}\in V[1,n]$ \emph{?} "`ja"' \emph{:} "`nein"'
}\end{minipage}}

\end{frame}

\begin{frame}\frametitle{Bezug zur Tabelle}

\scalebox{0.7}{%
\begin{minipage}{8cm}\codebox{
\emph{Eingabe:} Wort $w=\Sterm{a_1}\cdots\Sterm{a_n}$,\\
\hspace{2cm}CNF-Grammatik $G=\tuple{V,\Sigma,P, \Snterm{S}}$\\
\emph{Ausgabe:} "`ja"' wenn $w\in\Slang{L}(G)$; sonst "`nein"'\\[1ex]
%
% Grundfall:\\
\emph{for} $i=1,\ldots,n$:~~~\textcolor{devilscss}{// Teilwörter der Länge 1}\\
~~~$V[i,i]\defeq \{\Snterm{A}\in V\mid \Snterm{A}\to\Sterm{a_i}\in P\}$\\[2ex]
\ghost{\hspace{8.3cm}{\huge$\Leftarrow$}
\begin{tabular}{c|c|c|c|c|c|}
%  \multicolumn{1}{c}{}  & \multicolumn{1}{c}{A} & \multicolumn{1}{c}{B} & \multicolumn{1}{c}{C} & \multicolumn{1}{c}{D}  \\\cline{2-5}
\cline{2-6}
\Sterm{a}
	&  & \cellcolor{purple!20}   & \cellcolor{purple!40}   & \cellcolor{purple!60}  & \cellcolor{purple!80}  \\\cline{2-6}
\MEC{\Sterm{+}} &  
	&  & \cellcolor{purple!20}   & \cellcolor{purple!40}  &  \cellcolor{purple!60} \\\cline{3-6}
 \MEC{\Sterm{b}} &\MEC{}   &
	&  & \cellcolor{purple!20}   & \cellcolor{purple!40}  \\\cline{4-6}
 \MEC{\Sterm{*}} &\MEC{}   &\MEC{}   &   
	&  & \cellcolor{purple!20}  \\\cline{5-6}
 \MEC{\Sterm{c}} &\MEC{}   &\MEC{}   &\MEC{}   &
	&  \\\cline{6-6}
 \MEC{}          &\MEC{\Sterm{a}} & \MEC{\Sterm{+}} & \MEC{\Sterm{b}} & \MEC{\Sterm{*}} & \MEC{\Sterm{c}}
\end{tabular}
}%
\emph{for} $d=1,\ldots,n-1$:~~~\textcolor{devilscss}{// Differenz Endind. $-$ Startind.}\\
~~~\emph{for} $i=1,\ldots,n-d$:~~~\textcolor{devilscss}{// Startindex}\\
~~~~~~$j\defeq i+d$\\
~~~~~~$V[i,j]\defeq \emptyset$\\
~~~~~~\emph{for} $k=i,\ldots,j-1$:~~~\textcolor{devilscss}{// Trennindex}\\
~~~~~~~~~$V[i,j]\defeq V[i,j]\cup{}$\\
~~~~~~~~~\hspace{1.4cm}$\{\Snterm{A}\in V\mid \text{es gibt }\Snterm{A}\to\Snterm{B}\Snterm{C}\in P\text{ mit}$\\
~~~~~~~~~\hspace{2cm}$\Snterm{B}\in V[i,k]\text{ und }\Snterm{C}\in V[k+1,j]\}$\\[2ex]
\emph{return} $\Snterm{S}\in V[1,n]$ \emph{?} "`ja"' \emph{:} "`nein"'
}\end{minipage}}

\end{frame}

\begin{frame}\frametitle{Bezug zur Tabelle}

\scalebox{0.7}{%
\begin{minipage}{8cm}\codebox{
\emph{Eingabe:} Wort $w=\Sterm{a_1}\cdots\Sterm{a_n}$,\\
\hspace{2cm}CNF-Grammatik $G=\tuple{V,\Sigma,P, \Snterm{S}}$\\
\emph{Ausgabe:} "`ja"' wenn $w\in\Slang{L}(G)$; sonst "`nein"'\\[1ex]
%
% Grundfall:\\
\emph{for} $i=1,\ldots,n$:~~~\textcolor{devilscss}{// Teilwörter der Länge 1}\\
~~~$V[i,i]\defeq \{\Snterm{A}\in V\mid \Snterm{A}\to\Sterm{a_i}\in P\}$\\[2ex]
\emph{for} $d=1,\ldots,n-1$:~~~\textcolor{devilscss}{// Differenz Endind. $-$ Startind.}\\
\ghost{\hspace{8.3cm}{\huge$\Leftarrow$}
\begin{tabular}{c|c|c|c|c|c|}
%  \multicolumn{1}{c}{}  & \multicolumn{1}{c}{A} & \multicolumn{1}{c}{B} & \multicolumn{1}{c}{C} & \multicolumn{1}{c}{D}  \\\cline{2-5}
\cline{2-6}
\Sterm{a}
	&  & \cellcolor{purple!20}  &   &   &   \\\cline{2-6}
\MEC{\Sterm{+}} &  
	&  & \cellcolor{purple!40}  &   &   \\\cline{3-6}
 \MEC{\Sterm{b}} &\MEC{}   &
	&  & \cellcolor{purple!60}  &   \\\cline{4-6}
 \MEC{\Sterm{*}} &\MEC{}   &\MEC{}   &   
	&  & \cellcolor{purple!80} \\\cline{5-6}
 \MEC{\Sterm{c}} &\MEC{}   &\MEC{}   &\MEC{}   &
	&  \\\cline{6-6}
 \MEC{}          &\MEC{\Sterm{a}} & \MEC{\Sterm{+}} & \MEC{\Sterm{b}} & \MEC{\Sterm{*}} & \MEC{\Sterm{c}}
\end{tabular}
}%
~~~\emph{for} $i=1,\ldots,n-d$:~~~\textcolor{devilscss}{// Startindex}\\
~~~~~~$j\defeq i+d$\\
~~~~~~$V[i,j]\defeq \emptyset$\\
~~~~~~\emph{for} $k=i,\ldots,j-1$:~~~\textcolor{devilscss}{// Trennindex}\\
~~~~~~~~~$V[i,j]\defeq V[i,j]\cup{}$\\
~~~~~~~~~\hspace{1.4cm}$\{\Snterm{A}\in V\mid \text{es gibt }\Snterm{A}\to\Snterm{B}\Snterm{C}\in P\text{ mit}$\\
~~~~~~~~~\hspace{2cm}$\Snterm{B}\in V[i,k]\text{ und }\Snterm{C}\in V[k+1,j]\}$\\[2ex]
\emph{return} $\Snterm{S}\in V[1,n]$ \emph{?} "`ja"' \emph{:} "`nein"'
}\end{minipage}}

\end{frame}


\begin{frame}\frametitle{Bezug zur Tabelle}

\scalebox{0.7}{%
\begin{minipage}{8cm}\codebox{
\emph{Eingabe:} Wort $w=\Sterm{a_1}\cdots\Sterm{a_n}$,\\
\hspace{2cm}CNF-Grammatik $G=\tuple{V,\Sigma,P, \Snterm{S}}$\\
\emph{Ausgabe:} "`ja"' wenn $w\in\Slang{L}(G)$; sonst "`nein"'\\[1ex]
%
% Grundfall:\\
\emph{for} $i=1,\ldots,n$:~~~\textcolor{devilscss}{// Teilwörter der Länge 1}\\
~~~$V[i,i]\defeq \{\Snterm{A}\in V\mid \Snterm{A}\to\Sterm{a_i}\in P\}$\\[2ex]
\emph{for} $d=1,\ldots,n-1$:~~~\textcolor{devilscss}{// Differenz Endind. $-$ Startind.}\\
~~~\emph{for} $i=1,\ldots,n-d$:~~~\textcolor{devilscss}{// Startindex}\\
~~~~~~$j\defeq i+d$\\
~~~~~~$V[i,j]\defeq \emptyset$\\
\ghost{\hspace{8.3cm}{\huge$\Leftarrow$}
\begin{tabular}{c|c|c|c|c|c|}
%  \multicolumn{1}{c}{}  & \multicolumn{1}{c}{A} & \multicolumn{1}{c}{B} & \multicolumn{1}{c}{C} & \multicolumn{1}{c}{D}  \\\cline{2-5}
\cline{2-6}
\Sterm{a}
	&  &   &   &   &   \\\cline{2-6}
\MEC{\Sterm{+}} &  
	& \cellcolor{purple!20} & \cellcolor{purple!50}  & \cellcolor{purple!80}  &   \\\cline{3-6}
 \MEC{\Sterm{b}} &\MEC{}   &
	&  &   &  \cellcolor{purple!20} \\\cline{4-6}
 \MEC{\Sterm{*}} &\MEC{}   &\MEC{}   &   
	&  & \cellcolor{purple!50} \\\cline{5-6}
 \MEC{\Sterm{c}} &\MEC{}   &\MEC{}   &\MEC{}   &
	& \cellcolor{purple!80} \\\cline{6-6}
 \MEC{}          &\MEC{\Sterm{a}} & \MEC{\Sterm{+}} & \MEC{\Sterm{b}} & \MEC{\Sterm{*}} & \MEC{\Sterm{c}}
\end{tabular}
}%
~~~~~~\emph{for} $k=i,\ldots,j-1$:~~~\textcolor{devilscss}{// Trennindex}\\
~~~~~~~~~$V[i,j]\defeq V[i,j]\cup{}$\\
~~~~~~~~~\hspace{1.4cm}$\{\Snterm{A}\in V\mid \text{es gibt }\Snterm{A}\to\Snterm{B}\Snterm{C}\in P\text{ mit}$\\
~~~~~~~~~\hspace{2cm}$\Snterm{B}\in V[i,k]\text{ und }\Snterm{C}\in V[k+1,j]\}$\\[2ex]
\emph{return} $\Snterm{S}\in V[1,n]$ \emph{?} "`ja"' \emph{:} "`nein"'
}\end{minipage}}

\end{frame}

\begin{frame}\frametitle{Komplexität}

\alert{Wie komplex ist der CYK-Algorithmus?}
\begin{itemize}
\item Drei geschachtelte Schleifen mit Lauflänge $O(|w|)$
\item Operationen für Zugriff auf Grammatik und Teilmengen von $V$:%
\begin{itemize}
\item polynomiell bezüglich der Größe der Grammatik
\item konstant wenn die Grammatik vorher bekannt und fest ist
\end{itemize}
\end{itemize}\bigskip

\theobox{Satz: Das Wortproblem für kontextfreie Sprachen ist in $O(n^3)$ lösbar.}\bigskip

\emph{Anmerkung:} Die Umwandlung in CNF verlangt $\epsilon\notin\Slang{L}(G)$, aber 
im Fall $\epsilon\in\Slang{L}(G)$ kann man eine Grammatik $G'$ mit $\epsilon\notin\Slang{L}(G')$ erzeugen und
$w=\epsilon$ gesondert testen.

\end{frame}

\newcommand{\colcell}[2]{\only<#1|handout:0>{\cellcolor{#2}}}

\begin{frame}\frametitle{Beispiel 2}

\begin{minipage}{5.5cm}
Grammatik:\\[1ex]
$
\Snterm{S}\to \Snterm{C}\Snterm{D}\mid \Sterm{a}\qquad\Snterm{D}\to \Snterm{S}\Snterm{E} \qquad \Snterm{C}\to \Snterm{A}\Snterm{B}$\\
$\Snterm{E}\to \Snterm{B}\Snterm{A} \hspace{1.15cm} \Snterm{A} \to \Sterm{a}\hspace{0.97cm} \Snterm{B} \to \Sterm{b}
% \\
% \Snterm{N} \to \Snterm{P}\Snterm{Z}\qquad \Snterm{Z}\to \Sterm{a}\mid \Sterm{b}\mid \Sterm{c}
$
\end{minipage}\hspace{10mm}
\begin{minipage}{2.5cm}
Wort:\\[1ex] $w=\Sterm{ababababa}$\\
\end{minipage}
\bigskip

% Berechnung der Mengen $V[i,j]$:\bigskip

\narrowcentering{%
\begin{tabular}{c|c|c|c|c|c|c|c|c|c|}
%  \multicolumn{1}{c}{}  & \multicolumn{1}{c}{A} & \multicolumn{1}{c}{B} & \multicolumn{1}{c}{C} & \multicolumn{1}{c}{D}  \\\cline{2-5}
\MEC{}          &\MEC{\Sterm{a}} & \MEC{\Sterm{b}} & \MEC{\Sterm{a}} & \MEC{\Sterm{b}} & \MEC{\Sterm{a}}& \MEC{\Sterm{b}} & \MEC{\Sterm{a}} & \MEC{\Sterm{b}} & \MEC{\Sterm{a}}\\
\cline{2-10}
\Sterm{a}
	& \upcell{2}{\Snterm{S}}\upcell{3}{,\Snterm{A}}\colcell{12,13}{purple!40}\colcell{21,22,30,31,38,39,44,45}{purple!20}
		\colcell{49,50}{purple!10}\colcell{53,54}{purple!10}\colcell{56,57}{purple!10}
		& \upcell{13}{\Snterm{C}}\colcell{21,22}{purple!40}\colcell{30,31}{purple!40}\colcell{38,39}{purple!30}
			\colcell{44,45}{purple!30}\colcell{49,50}{purple!20}\colcell{53,54}{purple!20}\colcell{56,57}{purple!20}
			& \upcell{22}{\Snterm{D}}\colcell{30,31}{purple!60}\colcell{38,39}{purple!40}\colcell{44,45}{purple!40}
				\colcell{49,50}{purple!30}\colcell{53,54}{purple!30}\colcell{56,57}{purple!30}
				& \upcell{31}{--}\colcell{38,39}{purple!50}\colcell{44,45}{purple!50}\colcell{49,50}{purple!40}
					\colcell{53,54}{purple!40}\colcell{56,57}{purple!40}
					& \upcell{39}{\Snterm{S}}\colcell{44,45}{purple!60}\colcell{49,50}{purple!50}\colcell{53,54}{purple!50}
						\colcell{56,57}{purple!50}
						& \upcell{45}{--}\colcell{49,50}{purple!60}\colcell{53,54}{purple!60}\colcell{56,57}{purple!60}
							& \upcell{50}{\Snterm{D}}\colcell{53,54}{purple!70}\colcell{56,57}{purple!70}
								& \upcell{54}{--}\colcell{56,57}{purple!80}
									&  \upcell{57}{\Snterm{S}}\\\cline{2-10}
\MEC{\Sterm{b}} &  
		& \upcell{4}{\Snterm{B}}\phantom{,\Snterm{A}}\colcell{12,13}{purple!40}\colcell{23,24}{purple!20}
				\colcell{32,33}{purple!20}
			& \upcell{14}{\Snterm{E}}\colcell{21,22}{purple!20}\colcell{23,24}{purple!40}\colcell{32,33}{purple!40}
				& \upcell{24}{--}\colcell{30,31}{purple!20}\colcell{32,33}{purple!60}
					& \upcell{33}{--}\colcell{38,39}{purple!20}
						& \upcell{40}{--}\colcell{44,45}{purple!20}
							& \upcell{46}{--}\colcell{49,50}{purple!10}
								& \upcell{51}{--}\colcell{53,54}{purple!10}
									&  \upcell{55}{--}\colcell{56,57}{purple!10}\\\cline{3-10}
 \MEC{\Sterm{a}} &\MEC{}   &
			& \upcell{5}{\Snterm{S},\Snterm{A}}\colcell{21,22}{purple!40}
				& \upcell{15}{\Snterm{C}}\colcell{23,24}{purple!20}\colcell{30,31}{purple!40}
					& \upcell{25}{\Snterm{D}}\colcell{32,33}{purple!20}\colcell{38,39}{purple!30}
						& \upcell{34}{--}\colcell{44,45}{purple!30}
							& \upcell{41}{\Snterm{S}}\colcell{49,50}{purple!20}
								& \upcell{47}{--}\colcell{53,54}{purple!20}
									& \upcell{52}{\Snterm{D}}\colcell{56,57}{purple!20}\\\cline{4-10}
 \MEC{\Sterm{b}} &\MEC{}   &\MEC{}   &   
				& \upcell{6}{\Snterm{B}}\phantom{,\Snterm{A}}\colcell{23,24}{purple!40}\colcell{30,31}{purple!60}
					& \upcell{16}{\Snterm{E}}\colcell{32,33}{purple!40}\colcell{38,39}{purple!40}
						& \upcell{26}{--}\colcell{44,45}{purple!40}
							& \upcell{35}{--}\colcell{49,50}{purple!30}
								& \upcell{42}{--}\colcell{53,54}{purple!30}
									& \upcell{48}{--}\colcell{56,57}{purple!30}\\\cline{5-10}
 \MEC{\Sterm{a}} &\MEC{}   &\MEC{}   &\MEC{}   &
 					& \upcell{7}{\Snterm{S}, \Snterm{A}}\colcell{32,33}{purple!60}\colcell{38,39}{purple!50}
						& \upcell{17}{\Snterm{C}}\colcell{44,45}{purple!50}
							& \upcell{27}{\Snterm{D}}\colcell{49,50}{purple!40}
								& \upcell{36}{--}\colcell{53,54}{purple!40}
									& \upcell{43}{\Snterm{S}}\colcell{56,57}{purple!40}\\\cline{6-10}
 \MEC{\Sterm{b}} &\MEC{}   &\MEC{}   &\MEC{}   &\MEC{} &
						& \upcell{8}{\Snterm{B}}\phantom{,\Snterm{A}}\colcell{44,45}{purple!60}
							& \upcell{18}{\Snterm{E}}\colcell{49,50}{purple!50}
								& \upcell{28}{--}\colcell{53,54}{purple!50}
									& \upcell{37}{--}\colcell{56,57}{purple!50} \\\cline{7-10}
 \MEC{\Sterm{a}} &\MEC{}   &\MEC{}   &\MEC{}   &\MEC{}  &\MEC{}&
							& \upcell{9}{\Snterm{S}, \Snterm{A}}\colcell{49,50}{purple!60}
								& \upcell{19}{\Snterm{C}}\colcell{53,54}{purple!60}
									& \upcell{29}{\Snterm{D}}\colcell{56,57}{purple!60}\\\cline{8-10}
 \MEC{\Sterm{b}} &\MEC{}   &\MEC{}   &\MEC{}   &\MEC{}  &\MEC{} &\MEC{}&
								& \upcell{10}{\Snterm{B}}\phantom{,\Snterm{A}}\colcell{53,54}{purple!70}
									& \upcell{20}{\Snterm{E}}\colcell{56,57}{purple!70} \\\cline{9-10}
 \MEC{\Sterm{a}} &\MEC{}   &\MEC{}   &\MEC{}   &\MEC{}  &\MEC{} &\MEC{} &\MEC{}&
									& \upcell{11}{\Snterm{S},\Snterm{A}}\colcell{56,57}{purple!80} \\\cline{10-10}
\end{tabular}}\\
\ghost{\visible<58->{\raisebox{0.5cm}{\hspace{1.5cm}$S\in V[1,9]$ $\leadsto$ {$w$ kann erzeugt werden}}}}

\end{frame}

\sectionSlide{Das Pumping-Lemma für kontextfreie~Sprachen}

\begin{frame}\frametitle{Rückblick: Pumpen regulärer Sprachen}

\theobox{Satz (Pumping Lemma):
Für jede reguläre Sprache $\Slang{L}$\\
gibt es eine Zahl $n\geq 0$, so dass gilt:\\
~~für jedes Wort $z\in\Slang{L}$ mit $|z|\geq n$\\
~~gibt es eine Zerlegung $z=uvw$ mit $|v|\geq 1$ und $|uv|\leq n$, so dass:\\
~~~~für jede Zahl $k\geq 0$ gilt: $u v^k w\in\Slang{L}$
}\medskip

\emph{In Worten:} Jedes ausreichend lange Wort einer regulären Sprache enthält ein Teilwort, welches beliebig oft wiederholt werden kann um weitere Wörter der Sprache zu erhalten.
\bigskip

\alert{Aber:} Für viele kontextfreie Sprachen wie $\Sterm{a}^n\Sterm{b}^n$ trifft das nicht zu.

\end{frame}

\begin{frame}\frametitle{Pumpen für kontextfreie Sprachen}

\theobox{Satz (Pumping Lemma):
Für jede kontextfreie Sprache $\Slang{L}$\\
gibt es eine Zahl $n\geq 0$, so dass gilt:\\
~~für jedes Wort $z\in\Slang{L}$ mit $|z|\geq n$\\
~~gibt es eine Zerlegung $z=uvwxy$ mit $|vx|\geq 1$ und $|vwx|\leq n$, s.d.:\\
~~~~für jede Zahl $k\geq 0$ gilt: $u v^k w x^k y\in\Slang{L}$
}\medskip\pause

\examplebox{Beispiel: Für die Sprache $\{\Sterm{a}^i\Sterm{b}^i\mid i\geq 0\}$ gilt der Satz. Wir wählen $n=2$.
Sei $z=\Sterm{a}^i\Sterm{b}^i$ mit $i\geq 1$ ein beliebiges Wort mit $|z|\geq 2$. Wir wählen die Zerlegung $u=\Sterm{a}^{i-1}$,
$v=\Sterm{a}$, $w=\epsilon$, $x=\Sterm{b}$ und $y=\Sterm{b}^{i-1}$.\\
Dann ist $u v^k w x^k y  = \Sterm{a}^{i-1}\Sterm{a}^{k}\Sterm{b}^{k}\Sterm{b}^{i-1}=\Sterm{a}^{i+k-1}\Sterm{b}^{i+k-1}\in\Slang{L}$ für alle $k\geq 0$.
}

\end{frame}

\begin{frame}\frametitle{Kontextfreies Pumpen: Idee}

\emph{Grundidee beim regulären Pumping Lemma:}
\begin{itemize}
\item Endliche Automaten haben nur endlich viele Zustände
\item Also muss beim Akzeptieren langer Wörter ein Zustand mehrfach besucht werden:\\
	\narrowcentering{$q_0 \stackrel{u_1}{\to} \ldots \stackrel{u_{|u|}}{\to} \redalert{p} \stackrel{v_1}{\to}\ldots \stackrel{v_{|v|}}{\to} \redalert{p}\stackrel{w_1}{\to}\ldots\stackrel{w_{|w|}}{\to} q_n$}
\item Also gibt es eine Schleife, die man beliebig oft durchlaufen kann
\end{itemize}
\bigskip\pause

\emph{Grundidee beim kontextfreien Pumping Lemma:}
\begin{itemize}
\item Grammatiken haben nur endlich viele Variablen
\item Also muss beim Generieren langer Wörter eine Variable zu etwas expandiert werden, was diese Variable nochmal enthält\ghost{:}\\
	\narrowcentering{$\Snterm{S}\Rightarrow\ldots\Rightarrow u\,\underline{\Snterm{A}}\,y \Rightarrow u\,\underline{z}\,y \Rightarrow\ldots\Rightarrow u\,\underline{v\Snterm{A}x}\,y \Rightarrow \ldots \Rightarrow u\,\underline{vwx}\,y$}
\item Also gibt es eine Schleife, die man beliebig oft durchlaufen kann
\end{itemize}

\end{frame}

\begin{frame}[fragile,t]\frametitle{Ableitungsbäume aufpumpen}

\vspace{-1ex}
Die Idee des Lemmas lässt sich gut am Ableitungsbaum darstellen:\\[1ex]

\narrowcentering{%
\begin{tikzpicture}[
	scale=1.50,
	decoration=penciline, decorate,
	node distance = 7mm and 9mm,
	mybox/.style args = {#1/#2}{
		draw=#1,% line color
		fill=#2,% fill color
% 		rounded corners,
		thick,
		text width=47mm, minimum height=8mm, inner sep=1mm, 
		align=flush center
	},
	myarrow/.style args = {#1}{
		line width=0.8mm,
		draw=#1,%line color
		%-{Triangle[length=2.8mm,width=4mm,fill=#1]},
		->,
		shorten >=0.5mm, shorten <=0.1mm
	},
	myarrowb/.style args = {#1}{
		line width=0.3mm,
		draw=#1,%line color
		%-{Triangle[length=2.8mm,width=4mm,fill=#1]},
		->,
		shorten >=1mm, shorten <=0.1mm
	}
]
\path[use as bounding box] (-3.2,0) rectangle (3.5,-4); % add "draw" to see it
\pgfmathsetseed{6571}
% \draw[help lines] (0,0) grid (5,5);
\node (s) [circle,draw=none,inner sep=1pt] at (0,0) {\Snterm{S}};
% 
\draw[fill=none,decorate,line width=0.3mm]
(s.270) -- (-1.5,-3.5) -- (1.5,-3.5) -- cycle;
%
\visible<2->{
\node (a1) [circle,draw=none,inner sep=0pt] at (0,-1.5) {\Snterm{A}};
%
\draw[decorate,dashed,line width=0.2mm]
(s.270) -- (-0.1,-0.8) -- (a1.90);
}
%
\visible<4->{
\draw[fill=darkgreen!30,decorate,line width=0.3mm]
(a1.270) -- (-1.0,-3.50) -- (1.0,-3.52) -- cycle;
}
%
\visible<3->{
\node (a2) [circle,draw=none,inner sep=0pt] at (0,-2.5) {\Snterm{A}};
%
\draw[decorate,dashed,line width=0.2mm]
(a1.270) -- (a2.90);
}
%
\visible<5->{
\draw[fill=darkred!30,decorate,line width=0.3mm]
(a2.270) -- (-0.5,-3.51) -- (0.5,-3.54) -- cycle;
}
%
\visible<6->{
\draw [thick, devilscss,decorate,decoration={brace,amplitude=3pt,mirror},xshift=0.4pt,yshift=-1mm](-1.5,-3.5) -- (-1.0,-3.5) node[black,midway,yshift=-0.25cm] {$u$};
\draw [thick, devilscss,decorate,decoration={brace,amplitude=3pt,mirror},xshift=0.4pt,yshift=-1.3mm](-1.0,-3.5) -- (-0.5,-3.5) node[black,midway,yshift=-0.28cm] {$v$};
\draw [thick, devilscss,decorate,decoration={brace,amplitude=3pt,mirror},xshift=0.4pt,yshift=-1.5mm](-0.5,-3.5) -- (0.5,-3.5) node[black,midway,yshift=-0.3cm] {$w$};
\draw [thick, devilscss,decorate,decoration={brace,amplitude=3pt,mirror},xshift=0.4pt,yshift=-1mm](0.5,-3.5) -- (1,-3.5) node[black,midway,yshift=-0.25cm] {$x$};
\draw [thick, devilscss,decorate,decoration={brace,amplitude=3pt,mirror},xshift=0.4pt,yshift=-0.7mm](1.0,-3.5) -- (1.5,-3.5) node[black,midway,yshift=-0.26cm] {$y$};
}
%
\draw [thick, alert,decorate,decoration={brace,amplitude=10pt,mirror},xshift=0.4pt,yshift=-0.4cm](-1.5,-3.5) -- (1.5,-3.5) node[black,midway,yshift=-0.55cm] {\alert{abgeleitetes Wort}};
% 
\node (l1) [align=left] at (-2,-1) {\alert{Ableitungsbaum}};
\draw[myarrowb=alert,bend right=10] (l1) edge[decorate] (-0.6,-1.5);
%
\visible<2->{
\node (l2) [align=left] at (2,-0.5) {\alert{Pfad im Baum}};
\draw[myarrowb=alert,bend left=5] (l2) edge[decorate] (0.05,-0.9);
}
%
\visible<4->{
\node (l3) [align=left] at (2.5,-2.0) {\alert{Teilbaum unterhalb}\\\alert{des ersten}\\\alert{Vorkommens von $\Snterm{A}$}};
\draw[myarrowb=alert,bend right=12] (l3) edge[decorate] (0.5,-2.4);
}
%
\visible<5->{
\node (l4) [align=right] at (-2.2,-2.3) {\alert{Teilbaum unterhalb}\\\alert{des zweiten}\\\alert{Vorkommens von $\Snterm{A}$}};
\draw[myarrowb=alert,bend right=12] (l4) edge[decorate] (-0.3,-3.1);
}
\end{tikzpicture}}

\end{frame}

\begin{frame}[fragile,t]\frametitle{Ableitungsbäume aufpumpen (2)}

\vspace{-1ex}
Die Idee des Lemmas lässt sich gut am Ableitungsbaum darstellen:\\[1ex]

\narrowcentering{%
\begin{tikzpicture}[
	scale=1.50,
	decoration=penciline, decorate,
	node distance = 7mm and 9mm,
	mybox/.style args = {#1/#2}{
		draw=#1,% line color
		fill=#2,% fill color
% 		rounded corners,
		thick,
		text width=47mm, minimum height=8mm, inner sep=1mm, 
		align=flush center
	},
	myarrow/.style args = {#1}{
		line width=0.8mm,
		draw=#1,%line color
		%-{Triangle[length=2.8mm,width=4mm,fill=#1]},
		->,
		shorten >=0.5mm, shorten <=0.1mm
	},
	myarrowb/.style args = {#1}{
		line width=0.3mm,
		draw=#1,%line color
		%-{Triangle[length=2.8mm,width=4mm,fill=#1]},
		->,
		shorten >=1mm, shorten <=0.1mm
	}
]
\path[use as bounding box] (-3.2,0) rectangle (3.5,-4); % add "draw" to see it
\pgfmathsetseed{6571}
% \draw[help lines] (0,0) grid (5,5);
\node (s) [circle,draw=none,inner sep=1pt] at (0,0) {\Snterm{S}};
% 
\draw[fill=none,decorate,line width=0.3mm]
(s.270) -- (-1.5,-3.5) -- (1.5,-3.5) -- cycle;
%
\node (a1) [circle,draw=none,inner sep=0pt] at (0,-1.5) {\Snterm{A}};
%
\draw[decorate,dashed,line width=0.2mm]
(s.270) -- (-0.1,-0.8) -- (a1.90);
%
\draw[fill=darkgreen!30,decorate,line width=0.3mm]
(a1.270) -- (-1.0,-3.50) -- (1.0,-3.52) -- cycle;
%
\node (a2) [circle,draw=none,inner sep=0pt] at (0,-2.5) {\Snterm{A}};
%
\draw[decorate,dashed,line width=0.2mm]
(a1.270) -- (a2.90);
%
\draw[fill=darkred!30,decorate,line width=0.3mm]
(a2.270) -- (-0.5,-3.51) -- (0.5,-3.54) -- cycle;
%
\draw [thick, devilscss,decorate,decoration={brace,amplitude=3pt,mirror},xshift=0.4pt,yshift=-1mm](-1.5,-3.5) -- (-1.0,-3.5) node[black,midway,yshift=-0.25cm] {$u$};
\draw [thick, devilscss,decorate,decoration={brace,amplitude=3pt,mirror},xshift=0.4pt,yshift=-1.3mm](-1.0,-3.5) -- (-0.5,-3.5) node[black,midway,yshift=-0.28cm] {$v$};
\draw [thick, devilscss,decorate,decoration={brace,amplitude=3pt,mirror},xshift=0.4pt,yshift=-1.5mm](-0.5,-3.5) -- (0.5,-3.5) node[black,midway,yshift=-0.3cm] {$w$};
\draw [thick, devilscss,decorate,decoration={brace,amplitude=3pt,mirror},xshift=0.4pt,yshift=-1mm](0.5,-3.5) -- (1,-3.5) node[black,midway,yshift=-0.25cm] {$x$};
\draw [thick, devilscss,decorate,decoration={brace,amplitude=3pt,mirror},xshift=0.4pt,yshift=-0.7mm](1.0,-3.5) -- (1.5,-3.5) node[black,midway,yshift=-0.26cm] {$y$};
%
% \draw [thick, alert,decorate,decoration={brace,amplitude=10pt,mirror},xshift=0.4pt,yshift=-0.4cm](-1.5,-3.5) -- (1.5,-3.5) node[black,midway,yshift=-0.55cm] {\alert{abgeleitetes Wort}};
\end{tikzpicture}}

\end{frame}

\begin{frame}[fragile,t]\frametitle{Ableitungsbäume aufpumpen (3)}

\vspace{-1ex}
Die Idee des Lemmas lässt sich gut am Ableitungsbaum darstellen:\\[1ex]

\narrowcentering{%
\begin{tikzpicture}[
	scale=1.50,
	decoration=penciline, decorate,
	node distance = 7mm and 9mm,
	mybox/.style args = {#1/#2}{
		draw=#1,% line color
		fill=#2,% fill color
% 		rounded corners,
		thick,
		text width=47mm, minimum height=8mm, inner sep=1mm, 
		align=flush center
	},
	myarrow/.style args = {#1}{
		line width=0.8mm,
		draw=#1,%line color
		%-{Triangle[length=2.8mm,width=4mm,fill=#1]},
		->,
		shorten >=0.5mm, shorten <=0.1mm
	},
	myarrowb/.style args = {#1}{
		line width=0.3mm,
		draw=#1,%line color
		%-{Triangle[length=2.8mm,width=4mm,fill=#1]},
		->,
		shorten >=1mm, shorten <=0.1mm
	}
]
\path[use as bounding box] (-3.2,0) rectangle (3.5,-4); % add "draw" to see it
\pgfmathsetseed{6571}
% \draw[help lines] (0,0) grid (5,5);
\node (s) [circle,draw=none,inner sep=1pt] at (-2,0) {\Snterm{S}};
%
\node (a1s) [circle,draw=none,inner sep=0pt] at (-2,-1.5) {\Snterm{A}};
% 
\draw[fill=none,decorate,line width=0.3mm]
(s.270) -- (-3.5,-3.5) -- (-3.0,-3.5) -- (a1s.270) -- (-1,-3.5) -- (-0.5,-3.5) -- cycle;
%
\draw[decorate,dashed,line width=0.2mm]
(s.270) -- (-2.1,-0.8) -- (a1s.90);
%
\node (a1) [circle,draw=none,inner sep=0pt] at (1,-1.5) {\Snterm{A}};
%
\node (a2a) [circle,draw=none,inner sep=0pt] at (1,-2.5) {\phantom{A}};
%
\draw[fill=darkgreen!30,decorate,line width=0.3mm]
(a1.270) -- (0,-3.50) -- (0.5,-3.50) -- (a2a.270) -- (1.5,-3.52) -- (2.0,-3.52) -- cycle;
\node (a2atop) [circle,draw=none,inner sep=0pt] at (1,-2.5) {\Snterm{A}};
%
\draw[decorate,dashed,line width=0.2mm]
(a1.270) -- (a2a.90);
%
\node (a2) [circle,draw=none,inner sep=0pt] at (3,-2.5) {\Snterm{A}};
%
\draw[fill=darkred!30,decorate,line width=0.3mm]
(a2.270) -- (2.5,-3.51) -- (3.5,-3.54) -- cycle;
%
\draw [thick, devilscss,decorate,decoration={brace,amplitude=3pt,mirror},xshift=0.4pt,yshift=-1mm](-3.5,-3.5) -- (-3.0,-3.5) node[black,midway,yshift=-0.25cm] {$u$};
\draw [thick, devilscss,decorate,decoration={brace,amplitude=3pt,mirror},xshift=0.4pt,yshift=-0.7mm](-1.0,-3.5) -- (-0.5,-3.5) node[black,midway,yshift=-0.26cm] {$y$};
%
\draw [thick, devilscss,decorate,decoration={brace,amplitude=3pt,mirror},xshift=0.4pt,yshift=-1.3mm](0,-3.5) -- (0.5,-3.5) node[black,midway,yshift=-0.28cm] {$v$};
\draw [thick, devilscss,decorate,decoration={brace,amplitude=3pt,mirror},xshift=0.4pt,yshift=-1mm](1.5,-3.5) -- (2,-3.5) node[black,midway,yshift=-0.25cm] {$x$};
%
\draw [thick, devilscss,decorate,decoration={brace,amplitude=3pt,mirror},xshift=0.4pt,yshift=-1.5mm](2.5,-3.5) -- (3.5,-3.5) node[black,midway,yshift=-0.3cm] {$w$};


%
% \draw [thick, alert,decorate,decoration={brace,amplitude=10pt,mirror},xshift=0.4pt,yshift=-0.4cm](-1.5,-3.5) -- (1.5,-3.5) node[black,midway,yshift=-0.55cm] {\alert{abgeleitetes Wort}};
\end{tikzpicture}}

\end{frame}

\begin{frame}[fragile,t]\frametitle{Ableitungsbäume aufpumpen (4)}

\vspace{-1ex}
\ghost{\hspace{7.5cm}\raisebox{-6.5cm}{Abgeleitetes Wort:}}%
\ghost{\hspace{7.5cm}\raisebox{-6.9cm}{$uwy$}}%
Die Idee des Lemmas lässt sich gut am Ableitungsbaum darstellen:\\[1ex]

\narrowcentering{%
\scalebox{1.0}{
\begin{tikzpicture}[
	scale=1.50,
	decoration=penciline, decorate,
	node distance = 7mm and 9mm,
	mybox/.style args = {#1/#2}{
		draw=#1,% line color
		fill=#2,% fill color
% 		rounded corners,
		thick,
		text width=47mm, minimum height=8mm, inner sep=1mm, 
		align=flush center
	},
	myarrow/.style args = {#1}{
		line width=0.8mm,
		draw=#1,%line color
		%-{Triangle[length=2.8mm,width=4mm,fill=#1]},
		->,
		shorten >=0.5mm, shorten <=0.1mm
	},
	myarrowb/.style args = {#1}{
		line width=0.3mm,
		draw=#1,%line color
		%-{Triangle[length=2.8mm,width=4mm,fill=#1]},
		->,
		shorten >=1mm, shorten <=0.1mm
	}
]
% \path[use as bounding box] (-3.2,0) rectangle (3.5,-4); % add "draw" to see it
\pgfmathsetseed{6571}
% \draw[help lines] (0,0) grid (5,5);
\node (s) [circle,draw=none,inner sep=1pt] at (0,0) {\Snterm{S}};
%
\node (a1s) [circle,draw=none,inner sep=0pt] at (0,-1.5) {\Snterm{A}};
% 
\draw[fill=none,decorate,line width=0.3mm]
(s.270) -- (-1.5,-3.5) -- (-1.0,-3.5) -- (0,-1.7) -- (1,-3.5) -- (1.5,-3.5) -- cycle;
%
\draw[decorate,dashed,line width=0.2mm]
(s.270) -- (-0.1,-0.8) -- (a1s.90);
%
\draw[fill=darkred!30,decorate,line width=0.3mm]
(a1s.270) -- (-0.5,-2.51) -- (0.5,-2.54) -- cycle;
%
\draw [thick, devilscss,decorate,decoration={brace,amplitude=3pt,mirror},xshift=0.4pt,yshift=-1.5mm](-0.5,-2.5) -- (0.5,-2.5) node[black,midway,yshift=-0.3cm] {$w$};
%
\draw [thick, devilscss,decorate,decoration={brace,amplitude=3pt,mirror},xshift=0.4pt,yshift=-1mm](-1.5,-3.5) -- (-1.0,-3.5) node[black,midway,yshift=-0.25cm] {$u$};
\draw [thick, devilscss,decorate,decoration={brace,amplitude=3pt,mirror},xshift=0.4pt,yshift=-0.7mm](1.0,-3.5) -- (1.5,-3.5) node[black,midway,yshift=-0.26cm] {$y$};
\end{tikzpicture}}}

\end{frame}

\begin{frame}[fragile,t]\frametitle{Ableitungsbäume aufpumpen (5)}

\vspace{-1ex}
\ghost{\hspace{7.5cm}\raisebox{-6.5cm}{Abgeleitetes Wort:}}%
\ghost{\hspace{7.5cm}\raisebox{-6.9cm}{$uvwxy$}}%
Die Idee des Lemmas lässt sich gut am Ableitungsbaum darstellen:\\[1ex]

\narrowcentering{%
\scalebox{1.0}{
\begin{tikzpicture}[
	scale=1.50,
	decoration=penciline, decorate,
	node distance = 7mm and 9mm,
	mybox/.style args = {#1/#2}{
		draw=#1,% line color
		fill=#2,% fill color
% 		rounded corners,
		thick,
		text width=47mm, minimum height=8mm, inner sep=1mm, 
		align=flush center
	},
	myarrow/.style args = {#1}{
		line width=0.8mm,
		draw=#1,%line color
		%-{Triangle[length=2.8mm,width=4mm,fill=#1]},
		->,
		shorten >=0.5mm, shorten <=0.1mm
	},
	myarrowb/.style args = {#1}{
		line width=0.3mm,
		draw=#1,%line color
		%-{Triangle[length=2.8mm,width=4mm,fill=#1]},
		->,
		shorten >=1mm, shorten <=0.1mm
	}
]
% \path[use as bounding box] (-3.2,0) rectangle (3.5,-4); % add "draw" to see it
\pgfmathsetseed{6571}
% \draw[help lines] (0,0) grid (5,5);
\node (s) [circle,draw=none,inner sep=1pt] at (0,0) {\Snterm{S}};
% 
\draw[fill=none,decorate,line width=0.3mm]
(s.270) -- (-1.5,-3.5) -- (1.5,-3.5) -- cycle;
%
\node (a1) [circle,draw=none,inner sep=0pt] at (0,-1.5) {\Snterm{A}};
%
\draw[decorate,dashed,line width=0.2mm]
(s.270) -- (-0.1,-0.8) -- (a1.90);
%
\draw[fill=darkgreen!30,decorate,line width=0.3mm]
(a1.270) -- (-1.0,-3.50) -- (1.0,-3.52) -- cycle;
%
\node (a2) [circle,draw=none,inner sep=0pt] at (0,-2.5) {\Snterm{A}};
%
\draw[decorate,dashed,line width=0.2mm]
(a1.270) -- (a2.90);
%
\draw[fill=darkred!30,decorate,line width=0.3mm]
(a2.270) -- (-0.5,-3.51) -- (0.5,-3.54) -- cycle;
%
\draw [thick, devilscss,decorate,decoration={brace,amplitude=3pt,mirror},xshift=0.4pt,yshift=-1mm](-1.5,-3.5) -- (-1.0,-3.5) node[black,midway,yshift=-0.25cm] {$u$};
\draw [thick, devilscss,decorate,decoration={brace,amplitude=3pt,mirror},xshift=0.4pt,yshift=-1.3mm](-1.0,-3.5) -- (-0.5,-3.5) node[black,midway,yshift=-0.28cm] {$v$};
\draw [thick, devilscss,decorate,decoration={brace,amplitude=3pt,mirror},xshift=0.4pt,yshift=-1.5mm](-0.5,-3.5) -- (0.5,-3.5) node[black,midway,yshift=-0.3cm] {$w$};
\draw [thick, devilscss,decorate,decoration={brace,amplitude=3pt,mirror},xshift=0.4pt,yshift=-1mm](0.5,-3.5) -- (1,-3.5) node[black,midway,yshift=-0.25cm] {$x$};
\draw [thick, devilscss,decorate,decoration={brace,amplitude=3pt,mirror},xshift=0.4pt,yshift=-0.7mm](1.0,-3.5) -- (1.5,-3.5) node[black,midway,yshift=-0.26cm] {$y$};
%
% \draw [thick, alert,decorate,decoration={brace,amplitude=10pt,mirror},xshift=0.4pt,yshift=-0.4cm](-1.5,-3.5) -- (1.5,-3.5) node[black,midway,yshift=-0.55cm] {\alert{abgeleitetes Wort}};
\end{tikzpicture}}}

\end{frame}

\begin{frame}[fragile,t]\frametitle{Ableitungsbäume aufpumpen (6)}

\vspace{-1ex}
\ghost{\hspace{7.5cm}\raisebox{-6.5cm}{Abgeleitetes Wort:}}%
\ghost{\hspace{7.5cm}\raisebox{-6.9cm}{$uvvwxxy$}}%
Die Idee des Lemmas lässt sich gut am Ableitungsbaum darstellen:\\[1ex]

\narrowcentering{%
\scalebox{0.90}{
\begin{tikzpicture}[
	scale=1.50,
	decoration=penciline, decorate,
	node distance = 7mm and 9mm,
	mybox/.style args = {#1/#2}{
		draw=#1,% line color
		fill=#2,% fill color
% 		rounded corners,
		thick,
		text width=47mm, minimum height=8mm, inner sep=1mm, 
		align=flush center
	},
	myarrow/.style args = {#1}{
		line width=0.8mm,
		draw=#1,%line color
		%-{Triangle[length=2.8mm,width=4mm,fill=#1]},
		->,
		shorten >=0.5mm, shorten <=0.1mm
	},
	myarrowb/.style args = {#1}{
		line width=0.3mm,
		draw=#1,%line color
		%-{Triangle[length=2.8mm,width=4mm,fill=#1]},
		->,
		shorten >=1mm, shorten <=0.1mm
	}
]
% \path[use as bounding box] (-3.2,0) rectangle (3.5,-5); % add "draw" to see it
\pgfmathsetseed{6571}
% \draw[help lines] (0,0) grid (5,5);
\node (s) [circle,draw=none,inner sep=1pt] at (0,0) {\Snterm{S}};
% 
\draw[fill=none,decorate,line width=0.3mm]
(s.270) -- (-1.5,-3.5) -- (1.5,-3.5) -- cycle;
%
\node (a11) [circle,draw=none,inner sep=0pt] at (0,-1.5) {\Snterm{A}};
%
\draw[decorate,dashed,line width=0.2mm]
(s.270) -- (-0.1,-0.8) -- (a11.90);
%
\draw[fill=darkgreen!30,decorate,line width=0.3mm]
(a11.270) -- (-1.0,-3.50) -- (1.0,-3.52) -- cycle;
%
\node (a1) [circle,draw=none,inner sep=0pt] at (0,-2.5) {\Snterm{A}};
%
\draw[decorate,dashed,line width=0.2mm]
(a11.270) -- (a1.90);
%
\draw[fill=darkgreen!30,decorate,line width=0.3mm]
(a1.270) -- (-1.0,-4.50) -- (1.0,-4.52) -- cycle;
%
\node (a2) [circle,draw=none,inner sep=0pt] at (0,-3.5) {\Snterm{A}};
%
\draw[decorate,dashed,line width=0.2mm]
(a1.270) -- (a2.90);
%
\draw[fill=darkred!30,decorate,line width=0.3mm]
(a2.270) -- (-0.5,-4.51) -- (0.5,-4.54) -- cycle;
%
\draw [thick, devilscss,decorate,decoration={brace,amplitude=3pt,mirror},xshift=0.4pt,yshift=-1mm](-1.5,-3.5) -- (-1.0,-3.5) node[black,midway,yshift=-0.25cm] {$u$};
\draw [thick, devilscss,decorate,decoration={brace,amplitude=3pt,mirror},xshift=0.4pt,yshift=-1.3mm](-1.0,-3.5) -- (-0.5,-3.5) node[black,midway,yshift=-0.28cm] {$v$};
\draw [thick, devilscss,decorate,decoration={brace,amplitude=3pt,mirror},xshift=0.4pt,yshift=-1mm](0.5,-3.5) -- (1,-3.5) node[black,midway,yshift=-0.25cm] {$x$};
\draw [thick, devilscss,decorate,decoration={brace,amplitude=3pt,mirror},xshift=0.4pt,yshift=-0.7mm](1.0,-3.5) -- (1.5,-3.5) node[black,midway,yshift=-0.26cm] {$y$};

\draw [thick, devilscss,decorate,decoration={brace,amplitude=3pt,mirror},xshift=0.4pt,yshift=-1.3mm](-1.0,-4.5) -- (-0.5,-4.5) node[black,midway,yshift=-0.28cm] {$v$};
\draw [thick, devilscss,decorate,decoration={brace,amplitude=3pt,mirror},xshift=0.4pt,yshift=-1mm](0.5,-4.5) -- (1,-4.5) node[black,midway,yshift=-0.25cm] {$x$};
\draw [thick, devilscss,decorate,decoration={brace,amplitude=3pt,mirror},xshift=0.4pt,yshift=-1.5mm](-0.5,-4.5) -- (0.5,-4.5) node[black,midway,yshift=-0.3cm] {$w$};
%
% \draw [thick, alert,decorate,decoration={brace,amplitude=10pt,mirror},xshift=0.4pt,yshift=-0.4cm](-1.5,-3.5) -- (1.5,-3.5) node[black,midway,yshift=-0.55cm] {\alert{abgeleitetes Wort}};
\end{tikzpicture}}}

\end{frame}

\begin{frame}[fragile,t]\frametitle{Ableitungsbäume aufpumpen (7)}

\vspace{-1ex}
\ghost{\hspace{7.5cm}\raisebox{-6.5cm}{Abgeleitetes Wort:}}%
\ghost{\hspace{7.5cm}\raisebox{-6.9cm}{$uvvvwxxxy$}}%
Die Idee des Lemmas lässt sich gut am Ableitungsbaum darstellen:\\[1ex]

\narrowcentering{%
\scalebox{0.75}{
\begin{tikzpicture}[
	scale=1.50,
	decoration=penciline, decorate,
	node distance = 7mm and 9mm,
	mybox/.style args = {#1/#2}{
		draw=#1,% line color
		fill=#2,% fill color
% 		rounded corners,
		thick,
		text width=47mm, minimum height=8mm, inner sep=1mm, 
		align=flush center
	},
	myarrow/.style args = {#1}{
		line width=0.8mm,
		draw=#1,%line color
		%-{Triangle[length=2.8mm,width=4mm,fill=#1]},
		->,
		shorten >=0.5mm, shorten <=0.1mm
	},
	myarrowb/.style args = {#1}{
		line width=0.3mm,
		draw=#1,%line color
		%-{Triangle[length=2.8mm,width=4mm,fill=#1]},
		->,
		shorten >=1mm, shorten <=0.1mm
	}
]
% \path[use as bounding box] (-3.2,0) rectangle (3.5,-5); % add "draw" to see it
\pgfmathsetseed{6571}
% \draw[help lines] (0,0) grid (5,5);
\node (s) [circle,draw=none,inner sep=1pt] at (0,0) {\Snterm{S}};
% 
\draw[fill=none,decorate,line width=0.3mm]
(s.270) -- (-1.5,-3.5) -- (1.5,-3.5) -- cycle;
%
\node (a12) [circle,draw=none,inner sep=0pt] at (0,-1.5) {\Snterm{A}};
%
\draw[decorate,dashed,line width=0.2mm]
(s.270) -- (-0.1,-0.8) -- (a12.90);
%
\draw[fill=darkgreen!30,decorate,line width=0.3mm]
(a12.270) -- (-1.0,-3.50) -- (1.0,-3.52) -- cycle;
%
\node (a11) [circle,draw=none,inner sep=0pt] at (0,-2.5) {\Snterm{A}};
%
\draw[decorate,dashed,line width=0.2mm]
(a12.270) -- (a11.90);
%
\draw[fill=darkgreen!30,decorate,line width=0.3mm]
(a11.270) -- (-1.0,-4.50) -- (1.0,-4.52) -- cycle;
%
\node (a1) [circle,draw=none,inner sep=0pt] at (0,-3.5) {\Snterm{A}};
%
\draw[decorate,dashed,line width=0.2mm]
(a11.270) -- (a1.90);
%
\draw[fill=darkgreen!30,decorate,line width=0.3mm]
(a1.270) -- (-1.0,-5.50) -- (1.0,-5.52) -- cycle;
%
\node (a2) [circle,draw=none,inner sep=0pt] at (0,-4.5) {\Snterm{A}};
%
\draw[decorate,dashed,line width=0.2mm]
(a1.270) -- (a2.90);
%
\draw[fill=darkred!30,decorate,line width=0.3mm]
(a2.270) -- (-0.5,-5.51) -- (0.5,-5.54) -- cycle;
%
\draw [thick, devilscss,decorate,decoration={brace,amplitude=3pt,mirror},xshift=0.4pt,yshift=-1mm](-1.5,-3.5) -- (-1.0,-3.5) node[black,midway,yshift=-0.25cm] {$u$};
\draw [thick, devilscss,decorate,decoration={brace,amplitude=3pt,mirror},xshift=0.4pt,yshift=-1.3mm](-1.0,-3.5) -- (-0.5,-3.5) node[black,midway,yshift=-0.28cm] {$v$};
\draw [thick, devilscss,decorate,decoration={brace,amplitude=3pt,mirror},xshift=0.4pt,yshift=-1mm](0.5,-3.5) -- (1,-3.5) node[black,midway,yshift=-0.25cm] {$x$};
\draw [thick, devilscss,decorate,decoration={brace,amplitude=3pt,mirror},xshift=0.4pt,yshift=-0.7mm](1.0,-3.5) -- (1.5,-3.5) node[black,midway,yshift=-0.26cm] {$y$};

\draw [thick, devilscss,decorate,decoration={brace,amplitude=3pt,mirror},xshift=0.4pt,yshift=-1.3mm](-1.0,-4.5) -- (-0.5,-4.5) node[black,midway,yshift=-0.28cm] {$v$};
\draw [thick, devilscss,decorate,decoration={brace,amplitude=3pt,mirror},xshift=0.4pt,yshift=-1mm](0.5,-4.5) -- (1,-4.5) node[black,midway,yshift=-0.25cm] {$x$};

\draw [thick, devilscss,decorate,decoration={brace,amplitude=3pt,mirror},xshift=0.4pt,yshift=-1.3mm](-1.0,-5.5) -- (-0.5,-5.5) node[black,midway,yshift=-0.28cm] {$v$};
\draw [thick, devilscss,decorate,decoration={brace,amplitude=3pt,mirror},xshift=0.4pt,yshift=-1mm](0.5,-5.5) -- (1,-5.5) node[black,midway,yshift=-0.25cm] {$x$};
\draw [thick, devilscss,decorate,decoration={brace,amplitude=3pt,mirror},xshift=0.4pt,yshift=-1.5mm](-0.5,-5.5) -- (0.5,-5.5) node[black,midway,yshift=-0.3cm] {$w$};
%
% \draw [thick, alert,decorate,decoration={brace,amplitude=10pt,mirror},xshift=0.4pt,yshift=-0.4cm](-1.5,-3.5) -- (1.5,-3.5) node[black,midway,yshift=-0.55cm] {\alert{abgeleitetes Wort}};
\end{tikzpicture}}}

\end{frame}

\begin{frame}[fragile,t]\frametitle{Ableitungsbäume aufpumpen (4+$k$)}

\vspace{-1ex}
\ghost{\hspace{7.5cm}\raisebox{-6.5cm}{Abgeleitetes Wort:}}%
\ghost{\hspace{7.5cm}\raisebox{-6.9cm}{$uv^kwx^ky$}}%
Die Idee des Lemmas lässt sich gut am Ableitungsbaum darstellen:\\[1ex]

\narrowcentering{%
\scalebox{0.65}{
\begin{tikzpicture}[
	scale=1.50,
	decoration=penciline, decorate,
	node distance = 7mm and 9mm,
	mybox/.style args = {#1/#2}{
		draw=#1,% line color
		fill=#2,% fill color
% 		rounded corners,
		thick,
		text width=47mm, minimum height=8mm, inner sep=1mm, 
		align=flush center
	},
	myarrow/.style args = {#1}{
		line width=0.8mm,
		draw=#1,%line color
		%-{Triangle[length=2.8mm,width=4mm,fill=#1]},
		->,
		shorten >=0.5mm, shorten <=0.1mm
	},
	myarrowb/.style args = {#1}{
		line width=0.3mm,
		draw=#1,%line color
		%-{Triangle[length=2.8mm,width=4mm,fill=#1]},
		->,
		shorten >=1mm, shorten <=0.1mm
	}
]
% \path[use as bounding box] (-3.2,0) rectangle (3.5,-5); % add "draw" to see it
\pgfmathsetseed{6571}
% \draw[help lines] (0,0) grid (5,5);
\node (s) [circle,draw=none,inner sep=1pt] at (0,0) {\Snterm{S}};
% 
\draw[fill=none,decorate,line width=0.3mm]
(s.270) -- (-1.5,-3.5) -- (1.5,-3.5) -- cycle;
%
\node (a12) [circle,draw=none,inner sep=0pt] at (0,-1.5) {\Snterm{A}};
%
\draw[decorate,dashed,line width=0.2mm]
(s.270) -- (-0.1,-0.8) -- (a12.90);
%
\draw[fill=darkgreen!30,decorate,line width=0.3mm]
(a12.270) -- (-1.0,-3.50) -- (1.0,-3.52) -- cycle;
%
\node (a11) [circle,draw=none,inner sep=0pt] at (0,-2.5) {\Snterm{A}};
%
\draw[decorate,dashed,line width=0.2mm]
(a12.270) -- (a11.90);
%
\node (a1u) [circle,draw=none,inner sep=0pt] at (0,-3.5) {\phantom{\Snterm{A}}};
%
\draw[fill=darkgreen!30,decorate,line width=0.3mm]
(a11.270) -- (-1.0,-4.50)-- (-0.5,-4.50)-- (a1u.270)-- (0.5,-4.50) -- (1.0,-4.52) -- cycle;
\node (a1utop) [circle,draw=none,inner sep=0pt] at (0,-3.5) {\Snterm{A}};
%
\draw[decorate,dashed,line width=0.2mm]
(a11.270) -- (a1u.90);
%
\node (dots) [circle,draw=none,inner sep=0pt] at (0,-4.0) {$\vdots$};
%
\node (a1) [circle,draw=none,inner sep=0pt] at (0,-4.5) {\Snterm{A}};
%
% \draw[decorate,dashed,line width=0.2mm]
% (0,-2.7) -- (a1.90);
%
\draw[fill=darkgreen!30,decorate,line width=0.3mm]
(a1.270) -- (-1.0,-6.50) -- (1.0,-6.52) -- cycle;
%
\node (a2) [circle,draw=none,inner sep=0pt] at (0,-5.5) {\Snterm{A}};
%
\draw[decorate,dashed,line width=0.2mm]
(a1.270) -- (a2.90);
%
\draw[fill=darkred!30,decorate,line width=0.3mm]
(a2.270) -- (-0.5,-6.51) -- (0.5,-6.54) -- cycle;
%
\draw [thick, devilscss,decorate,decoration={brace,amplitude=3pt,mirror},xshift=0.4pt,yshift=-1mm](-1.5,-3.5) -- (-1.0,-3.5) node[black,midway,yshift=-0.25cm] {$u$};
\draw [thick, devilscss,decorate,decoration={brace,amplitude=3pt,mirror},xshift=0.4pt,yshift=-1.3mm](-1.0,-3.5) -- (-0.5,-3.5) node[black,midway,yshift=-0.28cm] {$v$};
\draw [thick, devilscss,decorate,decoration={brace,amplitude=3pt,mirror},xshift=0.4pt,yshift=-1mm](0.5,-3.5) -- (1,-3.5) node[black,midway,yshift=-0.25cm] {$x$};
\draw [thick, devilscss,decorate,decoration={brace,amplitude=3pt,mirror},xshift=0.4pt,yshift=-0.7mm](1.0,-3.5) -- (1.5,-3.5) node[black,midway,yshift=-0.26cm] {$y$};

\draw [thick, devilscss,decorate,decoration={brace,amplitude=3pt,mirror},xshift=0.4pt,yshift=-1.3mm](-1.0,-4.5) -- (-0.5,-4.5) node[black,midway,yshift=-0.28cm] {$v$};
\draw [thick, devilscss,decorate,decoration={brace,amplitude=3pt,mirror},xshift=0.4pt,yshift=-1mm](0.5,-4.5) -- (1,-4.5) node[black,midway,yshift=-0.25cm] {$x$};

\draw [thick, devilscss,decorate,decoration={brace,amplitude=3pt,mirror},xshift=0.4pt,yshift=-1.3mm](-1.0,-6.5) -- (-0.5,-6.5) node[black,midway,yshift=-0.28cm] {$v$};
\draw [thick, devilscss,decorate,decoration={brace,amplitude=3pt,mirror},xshift=0.4pt,yshift=-1mm](0.5,-6.5) -- (1,-6.5) node[black,midway,yshift=-0.25cm] {$x$};
\draw [thick, devilscss,decorate,decoration={brace,amplitude=3pt,mirror},xshift=0.4pt,yshift=-1.5mm](-0.5,-6.5) -- (0.5,-6.5) node[black,midway,yshift=-0.3cm] {$w$};
%
% \draw [thick, alert,decorate,decoration={brace,amplitude=10pt,mirror},xshift=0.4pt,yshift=-0.4cm](-1.5,-3.5) -- (1.5,-3.5) node[black,midway,yshift=-0.55cm] {\alert{abgeleitetes Wort}};
\end{tikzpicture}}}

\end{frame}

\begin{frame}[t]\frametitle{Beweis des kontextfreien Pumping Lemma}

\vspace{-1ex}
\theobox{Satz (Pumping Lemma):
Für jede kontextfreie Sprache $\Slang{L}$\\
gibt es eine Zahl $n\geq 0$, so dass gilt:\\
~~für jedes Wort $z\in\Slang{L}$ mit $|z|\geq n$\\
~~gibt es eine Zerlegung $z=uvwxy$ mit $|vx|\geq 1$ und $|vwx|\leq n$, s.d.:\\
~~~~für jede Zahl $k\geq 0$ gilt: $u v^k w x^k y\in\Slang{L}$
}\medskip

\emph{Beweis:} Wir haben gesehen: wenn es einen Ableitungsbaum für $z$ gibt, der einen Pfad
enthält, in dem eine Variable \Snterm{A} mehrfach vorkommt, dann gibt es eine Zerlegung
$z=uvwxy$, die wesentliche Eigenschaften des Lemmas erfüllt.
\medskip\pause

Offene Fragen:
\begin{itemize}
\item Wieso sollte $|vx|\geq 1$ gelten?
\item Wie groß muss $n$ sein, damit der Ableitungsbaum garantiert solch einen Pfad mit doppelter Variable enthält?
\item Weshalb gilt $|vwx|\leq n$?
\end{itemize}\pause

\redalert{$\leadsto$ Hier hilft es, eine Grammatik in CNF anzunehmen}
% Diese Fragen sind einfacher zu beantworten, wenn man eine Grammatik in CNF annimmt!

\end{frame}

\begin{frame}[t,fragile]\frametitle{Beweis des kontextfreien Pumping Lemma (2)}

\vspace{-1ex}
\theobox{Satz (Pumping Lemma):
Für jede kontextfreie Sprache $\Slang{L}$\\
gibt es eine Zahl $n\geq 0$, so dass gilt:\\
~~für jedes Wort $z\in\Slang{L}$ mit $|z|\geq n$\\
~~gibt es eine Zerlegung $z=uvwxy$ mit $|vx|\geq 1$ und $|vwx|\leq n$, s.d.:\\
~~~~für jede Zahl $k\geq 0$ gilt: $u v^k w x^k y\in\Slang{L}$
}\medskip

\emph{Beweis:} Sei $G$ eine Grammatik für die Sprache $\Slang{L}\setminus\{\epsilon\}$ in CNF.
(Diese existiert immer. Das Wort $\epsilon$ ist im Beweis nicht relevant, da sicherlich $n>0$.)
\medskip

\alert{Wieso sollte $|vx|\geq 1$ gelten?}\pause\medskip

Wenn man einen Ableitungsbaum in $G$ findet, der die Form 
\ghost{\scalebox{0.4}{
\begin{tikzpicture}[
	baseline={-3.5cm},
	scale=1.00,
	decoration=penciline, decorate,
	node distance = 7mm and 9mm,
	mybox/.style args = {#1/#2}{
		draw=#1,% line color
		fill=#2,% fill color
% 		rounded corners,
		thick,
		text width=47mm, minimum height=8mm, inner sep=1mm, 
		align=flush center
	},
	myarrow/.style args = {#1}{
		line width=0.8mm,
		draw=#1,%line color
		%-{Triangle[length=2.8mm,width=4mm,fill=#1]},
		->,
		shorten >=0.5mm, shorten <=0.1mm
	},
	myarrowb/.style args = {#1}{
		line width=0.3mm,
		draw=#1,%line color
		%-{Triangle[length=2.8mm,width=4mm,fill=#1]},
		->,
		shorten >=1mm, shorten <=0.1mm
	}
]
% \path[use as bounding box] (-3.2,0) rectangle (3.5,-4); % add "draw" to see it
\pgfmathsetseed{6571}
% \draw[help lines] (0,0) grid (5,5);
\node (s) [circle,draw=none,inner sep=1pt] at (0,0) {\Snterm{S}};
% 
\draw[fill=none,decorate,line width=0.3mm]
(s.270) -- (-1.5,-3.5) -- (1.5,-3.5) -- cycle;
%
\node (a1) [circle,draw=none,inner sep=0pt] at (0,-1.5) {\Snterm{A}};
%
\draw[decorate,dashed,line width=0.2mm]
(s.270) -- (-0.1,-0.8) -- (a1.90);
%
\draw[fill=darkgreen!30,decorate,line width=0.3mm]
(a1.270) -- (-1.0,-3.50) -- (1.0,-3.52) -- cycle;
%
\node (a2) [circle,draw=none,inner sep=0pt] at (0,-2.5) {\Snterm{A}};
%
\draw[decorate,dashed,line width=0.2mm]
(a1.270) -- (a2.90);
%
\draw[fill=darkred!30,decorate,line width=0.3mm]
(a2.270) -- (-0.5,-3.51) -- (0.5,-3.54) -- cycle;
%
\draw [thick, devilscss,decorate,decoration={brace,amplitude=3pt,mirror},xshift=0.4pt,yshift=-1mm](-1.5,-3.5) -- (-1.0,-3.5) node[black,midway,yshift=-0.25cm] {$u$};
\draw [thick, devilscss,decorate,decoration={brace,amplitude=3pt,mirror},xshift=0.4pt,yshift=-1.3mm](-1.0,-3.5) -- (-0.5,-3.5) node[black,midway,yshift=-0.28cm] {$v$};
\draw [thick, devilscss,decorate,decoration={brace,amplitude=3pt,mirror},xshift=0.4pt,yshift=-1.5mm](-0.5,-3.5) -- (0.5,-3.5) node[black,midway,yshift=-0.3cm] {$w$};
\draw [thick, devilscss,decorate,decoration={brace,amplitude=3pt,mirror},xshift=0.4pt,yshift=-1mm](0.5,-3.5) -- (1,-3.5) node[black,midway,yshift=-0.25cm] {$x$};
\draw [thick, devilscss,decorate,decoration={brace,amplitude=3pt,mirror},xshift=0.4pt,yshift=-0.7mm](1.0,-3.5) -- (1.5,-3.5) node[black,midway,yshift=-0.26cm] {$y$};
%
% \draw [thick, alert,decorate,decoration={brace,amplitude=10pt,mirror},xshift=0.4pt,yshift=-0.4cm](-1.5,-3.5) -- (1.5,-3.5) node[black,midway,yshift=-0.55cm] {\alert{abgeleitetes Wort}};
\end{tikzpicture}}} \\hat, dann muss der obere \Snterm{A}-Knoten genau zwei Kinder haben (wie jeder innere Knoten in CNF).
Da CNF-Grammatiken $\epsilon$-frei sind, führt jeder Kindknoten zu einem nichtleeren Teilwort.
Daher muss entweder $v$ oder $x$ mindestens ein Symbol enthalten, d.h. $|vx|\geq 1$.
% Diese Fragen sind einfacher zu beantworten, wenn man eine Grammatik in CNF annimmt!

\end{frame}

\begin{frame}[t]\frametitle{Beweis des kontextfreien Pumping Lemma (3)}

% \vspace{-1ex}
% \theobox{Satz (Pumping Lemma):
% Für jede kontextfreie Sprache $\Slang{L}$\\
% gibt es eine Zahl $n\geq 0$, so dass gilt:\\
% ~~für jedes Wort $z\in\Slang{L}$ mit $|z|\geq n$\\
% ~~gibt es eine Zerlegung $z=uvwxy$ mit $|vx|\geq 1$ und $|vwx|\leq n$, s.d.:\\
% ~~~~für jede Zahl $k\geq 0$ gilt: $u v^k w x^k y\in\Slang{L}$
% }\medskip

\emph{Beweis:} Sei $G=\tuple{V,\Sigma,P,\Snterm{S}}$ eine Grammatik für $\Slang{L}\setminus\{\epsilon\}$ in CNF.
% (Diese existiert immer. Das Wort $\epsilon$ ist im Beweis nicht relevant, da sicherlich $n>0$.)
\medskip

\alert{Wie groß muss $n$ sein, damit der Ableitungsbaum garantiert solch einen Pfad mit doppelter Variable enthält?}\pause\medskip

Jeder Ableitungsbaum in $G$ ist ein Binärbaum (bis auf die letzte Ebene, wo Variablen durch Terminale ersetzt werden).\pause
\begin{itemize}
\item Ein Ableitungsbaum für $z$ hat $|z|$ Blätter\pause
\item Ein Binärbaum mit $|z|$ Blättern muss Pfade der Länge $\geq\log_2|z|$ enthalten\\
{\footnotesize\textcolor{devilscss}{(Der größte Binärbaum mit Pfaden der Länge $\ell$ ist der, in dem alle Pfade diese Länge haben und der hat $2^\ell$ Blätter)}}\pause
\item Jeder Pfad der Länge $\geq |V|$ muss mindestens eine Variable doppelt enthalten
{\footnotesize\textcolor{devilscss}{(Schubfachprinzip)}}\pause
\end{itemize}

\begin{enumerate}[$\leadsto$]
\item  Wenn $|z|\geq 2^{|V|}$ ist, dann gibt es einen Pfad, in dem eine Variable doppelt vorkommt
\end{enumerate}


\end{frame}

\begin{frame}[t]\frametitle{Beweis des kontextfreien Pumping Lemma (4)}

\vspace{-1ex}
\theobox{Satz (Pumping Lemma):
Für jede kontextfreie Sprache $\Slang{L}$\\
gibt es eine Zahl $n\geq 0$, so dass gilt:\\
~~für jedes Wort $z\in\Slang{L}$ mit $|z|\geq n$\\
~~gibt es eine Zerlegung $z=uvwxy$ mit $|vx|\geq 1$ und $|vwx|\leq n$, s.d.:\\
~~~~für jede Zahl $k\geq 0$ gilt: $u v^k w x^k y\in\Slang{L}$
}\medskip

\emph{Beweis:} % Sei $G=\tuple{V,\Sigma,P,\Snterm{S}}$ eine Grammatik für $\Slang{L}\setminus\{\epsilon\}$ in CNF.
% (Diese existiert immer. Das Wort $\epsilon$ ist im Beweis nicht relevant, da sicherlich $n>0$.)
%\medskip
% 
Gerade gezeigt: Die Konstante aus dem Lemma ist \ghost{$n=2^{|V|}$.}
\medskip

\alert{Weshalb gilt $|vwx|\leq n$?}\pause\medskip

Auch hier hilft es, dass Ableitungsbäume binär sind:
\begin{itemize}
\item Ein Binärbaum der Tiefe $\ell$ hat maximal $2^\ell$ Blätter\pause
\item Wir können annehmen, dass das obere doppelte Vorkommen der gewählten Variable maximal
$|V|$ Schritte von der vorletzten Ebene (d.h. der letzten im inneren Binärbaum) entfernt ist\pause
\item Also kann der Baum unterhalb dieses Vorkommens maximal $2^{|V|}=n$ Blätter haben (dies sind die Symbole in $vwx$)
\end{itemize}

\end{frame}

\begin{frame}[t]\frametitle{Beweis des kontextfreien Pumping Lemma (5)}

\vspace{-1ex}
\theobox{Satz (Pumping Lemma):
Für jede kontextfreie Sprache $\Slang{L}$\\
gibt es eine Zahl $n\geq 0$, so dass gilt:\\
~~für jedes Wort $z\in\Slang{L}$ mit $|z|\geq n$\\
~~gibt es eine Zerlegung $z=uvwxy$ mit $|vx|\geq 1$ und $|vwx|\leq n$, s.d.:\\
~~~~für jede Zahl $k\geq 0$ gilt: $u v^k w x^k y\in\Slang{L}$
}\medskip

\emph{Beweis (Zusammenfassung):} 
\begin{itemize}
\item Zum Aufpumpen genügt es, dass in einem Pfad eines Ableitungsbaumes eine Variable doppelt auftritt
\item Bei einer CNF-Grammatik $G=\tuple{V,\Sigma,P,\Snterm{S}}$ muss das Wort dafür $\geq n=2^{|V|}$ Zeichen haben
\item $|vx|\geq 1$ und $|vwx|\leq n$ ergeben sich, weil CNF-Ableitungsbäume binär sind
\end{itemize}
\qed
% Sei $G=\tuple{V,\Sigma,P,\Snterm{S}}$ eine Grammatik für $\Slang{L}\setminus\{\epsilon\}$ in CNF.
% (Diese existiert immer. Das Wort $\epsilon$ ist im Beweis nicht relevant, da sicherlich $n>0$.)
%\medskip
% 
\end{frame}

\begin{frame}\frametitle{Anwendung des Pumping Lemma}

Wie schon bei regulären Sprachen ist die Pump-Eigenschaft \redalert{notwendig, aber nicht hinreichend}
für Kontextfreiheit\\
$\leadsto$ Hauptanwendung: Erkennen nicht-kontextfreier Sprachen
\medskip\pause

\emph{Beispiel:} Die Sprache $\Slang{L}=\{\Sterm{a}^i\Sterm{b}^i\Sterm{c}^i\mid i\geq 0\}$ ist nicht kontextfrei.
\medskip\pause

Beweis durch Widerspruch: angenommen, $\Slang{L}$ wäre kontextfrei\pause
\begin{itemize}
\item Dann gibt es eine Konstante $n$ wie im Pumping Lemma.\pause
\item Dann muss es auch für das Wort $z=\Sterm{a}^n\Sterm{b}^n\Sterm{c}^n$ eine geeignete Zerlegung $z=uvwxy$ geben.\pause
\item Wegen $|vwx|\leq n$ muss $vwx$ entweder ein Teilwort von $\Sterm{a}^n\Sterm{b}^n$ oder von $\Sterm{b}^n\Sterm{c}^n$ sein.\pause
\item Fall 1: $vwx$ ist Teilwort von $\Sterm{a}^n\Sterm{b}^n$. Weil $|vx|\geq 1$ und $vx\subseteq\{\Sterm{a},\Sterm{b}\}^*$
hat $uv^2wx^2y$ mehr $\Sterm{a}$ oder $\Sterm{b}$ als $\Sterm{c}$. Also ist $uv^2wx^2y\notin\Slang{L}$.
\pause
\item Fall 2: $vwx$ ist Teilwort von $\Sterm{b}^n\Sterm{c}^n$. Analog zu Fall 1.\qed
\end{itemize}

\end{frame}

\begin{frame}\frametitle{Zusammenfassung und Ausblick}

Der \redalert{CYK-Algorithmus} ist von kubischer Komplexität.
\bigskip

Das \redalert{Pumping-Lemma für kontextfreie Sprachen} beruht auf der Ausnutzung von Schleifen in Ableitungen einer CNF-Grammatik
\bigskip

Die Sprache \redalert{$\{\Sterm{a}^i\Sterm{b}^i\Sterm{c}^i\mid i\geq 0\}$} ist nicht kontextfrei
\bigskip

\anybox{yellow}{
Offene Fragen:
\begin{itemize}
\item Wie steht es mit Abschlusseigenschaften bei kontextfreien Sprachen?
\item Haben kontextfreie Sprachen ein Berechnungsmodell?
\item Welche Probleme auf kontextfreien Grammatiken kann man lösen?
\end{itemize}
}

\end{frame}


\end{document}