\documentclass[aspectratio=1610,onlymath]{beamer}
% \documentclass[aspectratio=1610,onlymath,handout]{beamer}

\input{macros-lecture}
% Common notation

\usepackage{amsmath,amssymb,amsfonts}
\usepackage{xspace}

\newcommand{\lectureurl}{https://iccl.inf.tu-dresden.de/web/FS2025}

\DeclareMathAlphabet{\mathsc}{OT1}{cmr}{m}{sc} % Let's have \mathsc since the slide style has no working \textsc

% Dual of "phantom": make a text that is visible but intangible
\newcommand{\ghost}[1]{\raisebox{0pt}[0pt][0pt]{\makebox[0pt][l]{#1}}}

\newcommand{\tuple}[1]{\langle{#1}\rangle}
\newcommand{\defeq}{\mathrel{:=}}

%%% Annotation %%%

\usepackage{color}
\newcommand{\todo}[1]{{\tiny\color{red}\textbf{TODO: #1}}}



%%% Old macros below; move when needed

\newcommand{\blank}{\text{\textvisiblespace}} % empty tape cell for TM

% table syntax
\newcommand{\dom}{\textbf{dom}}
\newcommand{\adom}{\textbf{adom}}
\newcommand{\dbconst}[1]{\texttt{"#1"}}
\newcommand{\pred}[1]{\textsf{#1}}
\newcommand{\foquery}[2]{#2[#1]}
\newcommand{\ground}[1]{\textsf{ground}(#1)}
% \newcommand{\foquery}[2]{\{#1\mid #2\}} %% Notation as used in Alice Book
% \newcommand{\foquery}[2]{\tuple{#1\mid #2}}

\newcommand{\quantor}{\mathord{\reflectbox{$\text{\sf{Q}}$}}} % the generic quantor

% logic syntax
\newcommand{\Inter}{\mathcal{I}} %used to denote an interpretation
\newcommand{\Jnter}{\mathcal{J}} %used to denote another interpretation
\newcommand{\Knter}{\mathcal{K}} %used to denote yet another interpretation
\newcommand{\Zuweisung}{\mathcal{Z}} %used to denote a variable assignment

% query languages
\newcommand{\qlang}[1]{{\sf #1}} % Font for query languages
\newcommand{\qmaps}[1]{\textbf{QM}({\sf #1})} % Set of query mappings for a query language

%%% Complexities %%%

\hyphenation{Exp-Time} % prevent "Ex-PTime" (see, e.g. Tobies'01, Glimm'07 ;-)
\hyphenation{NExp-Time} % better that than something else

% \newcommand{\complclass}[1]{{\sc #1}\xspace} % font for complexity classes
\newcommand{\complclass}[1]{\ensuremath{\mathsc{#1}}\xspace} % font for complexity classes

\newcommand{\ACzero}{\complclass{AC$_0$}}
\newcommand{\LogSpace}{\complclass{L}}
\newcommand{\NLogSpace}{\complclass{NL}}
\newcommand{\PTime}{\complclass{P}}
\newcommand{\NP}{\complclass{NP}}
\newcommand{\coNP}{\complclass{coNP}}
\newcommand{\PH}{\complclass{PH}}
\newcommand{\PSpace}{\complclass{PSpace}}
\newcommand{\NPSpace}{\complclass{NPSpace}}
\newcommand{\ExpTime}{\complclass{ExpTime}}
\newcommand{\NExpTime}{\complclass{NExpTime}}
\newcommand{\ExpSpace}{\complclass{ExpSpace}}
\newcommand{\TwoExpTime}{\complclass{2ExpTime}}
\newcommand{\NTwoExpTime}{\complclass{N2ExpTime}}
\newcommand{\ThreeExpTime}{\complclass{3ExpTime}}
\newcommand{\kExpTime}[1]{\complclass{#1ExpTime}}
\newcommand{\kExpSpace}[1]{\complclass{#1ExpSpace}}


\defineTitle{20}{Typ 0 und Typ 1}{4. Januar 2024}

\begin{document}

\maketitle

% \begin{frame}\frametitle{}
% 
% ~\hfill
% \includegraphics[height=6.5cm]{a4}
% \hfill~
% 
% \end{frame}

\sectionSlideNoHandout{Rückblick}

\begin{frame}\frametitle{Die Turingmaschine}

\narrowcentering{%
\begin{tikzpicture}[
	scale=0.50,
	decoration=penciline, decorate
]
% \path[use as bounding box] (-3.2,0) rectangle (3.5,-5); % add "draw" to see it
% \draw[help lines] (0,0) grid (5,5);
\pgfmathsetseed{5712}
%
\node (inlabel) [rectangle,draw=none,inner sep=1pt] at (3,0.5) {\alert{Eingabe-/Speicherband}};
\draw[decorate,line width=0.3mm] (-1,0) -- (10.5,0);
\draw[decorate,line width=0.3mm] (-1,-1) -- (10.5,-1);
\foreach \x in {0,...,10} {
	\draw[decorate,line width=0.3mm] (\x-1,0) -- (\x-1,-0.9);
	\node (s\x) [circle,draw=none,inner sep=1pt] at (\x-0.5,-0.5) {\ifthenelse{\x<5}{\ifthenelse{\x<3}{\Sterm{a}}{\Sterm{b}}}{\ifthenelse{\x=5 \OR \x=7 \OR \x=8}{\Snterm{C}}{\ifthenelse{\x=9}{\Sterm{b}}{\Snterm{D}}}}};
}
\draw[decorate,line width=0.3mm] (10,0) -- (10,-0.9);
\node (dots) [circle,draw=none,inner sep=1pt] at (11.1,-0.5) {$\cdots$};

% \draw[decorate,line width=0.3mm] (5.5,0) -- (6,0) -- (6,-0.9) -- (5.5,-0.9) ;
% \draw[decorate,line width=0.3mm] (6,0) -- (6,-0.9);

\draw[fill=none,decorate,line width=0.3mm]
	(2,-3) -- (6,-3) -- (6,-7) -- (2,-7) -- cycle;
\node (falabel) [circle,draw=none,inner sep=1pt,align=left] at (4,-5) {Endliche\\Steuerung};
\draw[fill=none,decorate,line width=0.4mm,darkblue,->]
	(4,-3) -- (4,-2) -- (6.5,-2) -> (6.5,-1);
\node (headlabel) [rectangle,draw=none,inner sep=1pt,align=left] at (9.5,-2.0) {\footnotesize\alert{Lese-/Schreibkopf}\\[-0.6ex]\footnotesize\alert{(beweglich)}};

\node[rectangle,align=center,draw,line width=0.3mm,decorate, minimum width=8mm, minimum height=8mm] (state) at (8, -6) {$q$};
\draw[fill=none,decorate,line width=0.4mm,darkblue,->]
	(6,-6) -> (state.180);
\node (qlabel) [rectangle,draw=none,inner sep=1pt] at (11.5,-6) {\footnotesize\alert{Zustandsvariable}};

% \node[cloud, cloud puffs=15.7, cloud ignores aspect, minimum width=4cm, minimum height=1cm, align=center, draw,line width=0.3mm] (memory) at (11, -1) {\alert{zusätzlicher}\\\alert{Speicher}};
% 
% \draw[fill=none,decorate,line width=0.3mm] (9,0) -- (14,0);
% \draw[fill=none,decorate,line width=0.3mm] (9,-1) -- (14,-1);
% \draw[fill=none,decorate,line width=0.3mm] (9,0) -- (9,-1);
% \foreach \y in {10,...,14} {
% 	\draw[fill=none,decorate,line width=0.3mm] (\y,0) -- (\y,-0.9);
% 	\node (k\y) [circle,draw=none,inner sep=1pt] at (\y-0.5,-0.5) {\ifthenelse{\y<12}{\Snterm{A}}{\Snterm{B}}};
% }
% \draw[fill=none,decorate,line width=0.4mm,darkblue,->]
% 	(6,-3.5) -- (7.5,-3.5) -- (7.5,-0.5) -> (9,-0.5);
% \node (stacklabel) [circle,draw=none,inner sep=1pt] at (11.5,-1.5) {\alert{Warteschlange}};
\end{tikzpicture}}\bigskip

\defbox{Eine \redalert{(deterministische) Turingmaschine} (DTM) ist ein Tupel
$\Smach{M}=\tuple{Q,\Sigma,\Gamma,\delta,q_0,F}$ bestehend aus
\redalert{Zustandsmenge} $Q$, \redalert{Eingabealphabet} $\Sigma$,
\redalert{Arbeitsalphabet} $\Gamma\supseteq\Sigma\cup\{\blank\}$,
\redalert{Startzustand} $q_0\subseteq Q$, \redalert{Endzuständen} $F\subseteq Q$,
und einer partiellen \redalert{Übergangsfunktion} \\[1ex]
\narrowcentering{$\delta: Q\times\Gamma \to Q\times\Gamma\times\{L,R,N\}$}
}

\end{frame}

\begin{frame}\frametitle{Turing-Mächtigkeit}

\begin{itemize}
\item Die Turingmaschine ist das\\ \redalert{mächtigste bekannte Berechnungsmodell}\\
$\leadsto$ was nicht Turing-berechenbar ist, gilt als unberechenbar
\item Zahlreiche andere Modelle sind ebenso Turing-mächtig\\
\anybox{strongyellow}{\vspace{-2ex}\small%
\begin{itemize}
\item \alert{Turingmaschinen} in vielen Varianten (deterministisch/nichtdeterministisch, Einband/Mehrband,
einseitig/zweiseitig unendlich, \ghost{\ldots)}\vspace{-1.1ex}
\item alle "`echten"' \alert{Programmiersprachen} (C, PHP, Java, C++, Python, JavaScript, Perl, BASIC, Pascal, Fortran,
{\footnotesize C$^\sharp$, COBOL, Lisp, Ruby, Visual Basic, Assembler, Ada, Prolog, Lua, Haskell, ASP.NET, Scheme, ALGOL, R, Scratch, ASP, Logo, TeX, awk, Embarcadero Delphi, Smalltalk, 
 Objective-C, APL, Tcl, Go, Visual Basic .NET, 
 Scala, VHDL, ActionScript, XSLT, Brainfuck, \ldots$^*$)}\vspace{-1.1ex}
\item theoretische \alert{Kalküle} (Prädikatenlogik erster Stufe, $\lambda$-Kalkül, allgemeine rekursive Funktionen, \ldots)\vspace{-1.1ex}
\item manch \alert{Unerwartetes} (C++-Templates, SQL, Java Generics, Magic: The Gathering, Microsoft Powerpoint, \ldots)
\end{itemize}}
\end{itemize}
\ghost{\hspace{13.8cm}\rotatebox{90}{\tiny \hspace{5mm}$^*$ \href{https://query.wikidata.org/\#SELECT\%20\%3Flanguage\%20\%3FlanguageLabel\%20\%28COUNT\%28DISTINCT\%20\%3Farticle\%29\%20AS\%20\%3FwikipediaArticleCount\%29\%20WHERE\%20\%7B\%0A\%20\%20\%3Flanguage\%20wdt\%3AP31\%2Fwdt\%3AP279\%2a\%20wd\%3AQ9143\%20.\%0A\%20\%20\%3Farticle\%20schema\%3Aabout\%20\%3Flanguage\%20.\%0A\%20\%20FILTER\%20\%28SUBSTR\%28str\%28\%3Farticle\%29\%2C\%2011\%2C\%2015\%29\%20\%3D\%20\%22.wikipedia.org\%2F\%22\%29\%20.\%0A\%20\%20SERVICE\%20wikibase\%3Alabel\%20\%7B\%20bd\%3AserviceParam\%20wikibase\%3Alanguage\%20\%22de\%2Cen\%22.\%20\%7D\%0A\%7D\%20GROUP\%20BY\%20\%3Flanguage\%20\%3FlanguageLabel\%0AORDER\%20BY\%20DESC\%20\%28\%3FwikipediaArticleCount\%29}{populäre Programmiersprachen (geordnet nach Zahl internationaler Wikipedia-Artikel)}}}

\end{frame}

\begin{frame}\frametitle{Entscheidbarkeit}

% \defbox{Das \redalert{Halteproblem} besteht in der folgenden Frage:\\
% Gegeben eine TM $\Smach{M}$ und ein Wort $w$, wird $\Smach{M}$ für die Eingabe $w$ jemals anhalten?}
\defbox{Das \redalert{Halteproblem} ist das Wortproblem für die Sprache\\[1ex]
\narrowcentering{
$\{\textsf{enc}(\Smach{M})\Sterm{\#}\Sterm{\#}\textsf{enc}(w)\mid \text{$\Smach{M}$ hält bei Eingabe $w$}\}$.
}}

\begin{itemize}
\item \alert{Entscheidbar:} Sprache wird von Turing-Entscheider erkannt
% \examplebox{Beispiel: Die Sprache $\{ww\mid w\in\{\Sterm{a},\Sterm{b}\}^*\}$ ist entscheidbar}
\item \alert{Unentscheidbar:} Sprache wird von keinem Turing-Entscheider erkannt
\item \alert{Semi-entscheidbar:} Sprache wird von einer TM erkannt, die aber eventuell kein Entscheider ist
\end{itemize}

\examplebox{Beispiel: \begin{itemize}
\item Die Sprache $\{ww\mid w\in\{\Sterm{a},\Sterm{b}\}^*\}$ ist entscheidbar (und damit auch semi-entscheidbar)
\item Das Halteproblem ist nicht entscheidbar aber semi-entscheidbar
\item Das Komplement des Halteproblems ist nicht semi-entscheidbar (und damit auch nicht entscheidbar)
\end{itemize}}

\end{frame}

\begin{frame}\frametitle{Der Satz von Rice}

Ein interessantes Resultat von Henry Gordon Rice zeigt die Probleme Turing-mächtiger Formalismen:\medskip

\theobox{\emph{Satz von Rice (informelle Version):} Jede nichttriviale Frage über die von einer TM ausgeführte Berechnung ist unentscheidbar.}\bigskip\pause

\theobox{\emph{Satz von Rice (formell):} Sei $E$ eine Eigenschaft von Sprachen, die für manche Turing-erkennbare Sprachen gilt und für manche Turing-erkennbare Sprachen nicht gilt (="`nichttriviale Eigenschaft"'). Dann ist das folgende Problem unentscheidbar:
\begin{itemize}
\item Eingabe: Turingmaschine $\Smach{M}$
\item Ausgabe: Hat $\Slang{L}(\Smach{M})$ die Eigenschaft $E$?
\end{itemize}}

\emph{Beweis:} Durch eine nicht sonderlich komplizierte Reduktion auf das Halteproblem. Kein Vorlesungsstoff.

\end{frame}

\begin{frame}\frametitle{Alles unentscheidbar}

\examplebox{Beispiele für Fragen, die laut Rice unentscheidbar sind:
\begin{itemize}
\item "`Ist $\Sterm{aba}\in\Slang{L}(\Smach{M})$?"' 
\item "`Ist $\Slang{L}(\Smach{M})$ leer?"'
\item "`Ist $\Slang{L}(\Smach{M})$ endlich?"'
\item "`Ist $\Slang{L}(\Smach{M})$ regulär?"'
\item \ldots
\end{itemize}
Rice ist dagegen nicht anwendbar auf:
\begin{itemize}
\item "`Hat $\Smach{M}$ mindestens zwei Zustände?"'\\ (keine Eigenschaft von $\Slang{L}(\Smach{M})$) 
\item "`Ist $\Slang{L}(\Smach{M})$ semi-entscheibar?"' (trivial)
\item \ldots
\end{itemize}
}

Der Satz von Rice lässt sich sinngemäß auf alle Turing-mächtigen Formalismen übertragen.

\end{frame}

\sectionSlide{Typ-0-Sprachen}

\begin{frame}\frametitle{Typ-0-Grammatiken und Turingmaschinen}

Turingmaschinen charakterisieren die Typ-0-Sprachen:\medskip

\theobox{\emph{Satz:} Die Typ-0-Grammatiken erzeugen genau diejenigen Sprachen, die von einer Turingmaschine erkannt werden können.}\medskip\pause

\alert{Direkte Konsequenzen:}
\begin{itemize}
\item Typ-0-Grammatiken sind ein universelles (Turing-mächtiges) Berechnungsmodell
\item Typ-0-Sprachen sind die größte Klasse von Sprachen, die wir mit einem "`implementierbaren"' Formalismus beschreiben können
\item Die Typ-0-Sprachen sind genau die semi-entscheidbaren Sprachen
\item Das Wortproblem für Typ-0-Sprachen ist unentscheidbar
\end{itemize}

\end{frame}

\begin{frame}\frametitle{Typ 0 $\Leftrightarrow$ TM}

\theobox{\emph{Satz:} Die Typ-0-Grammatiken erzeugen genau diejenigen Sprachen, die von einer Turingmaschine erkannt werden können.}\medskip

\emph{Beweis:} Die beiden Richtungen werden einzeln gezeigt:
\begin{enumerate}[(1)]
\item Wenn eine Sprache von einer Typ-0-Grammatik erzeugt wird, dann kann sie von einer TM erkannt werden.
\item Wenn eine Sprache von einer TM erkannt wird, dann kann sie durch eine Typ-0-Grammatik erzeugt werden.
\end{enumerate}

\end{frame}

\begin{frame}\frametitle{Typ 0 $\Rightarrow$ TM}

\emph{Gegeben:} Eine Grammatik $G$
\medskip

\emph{Gesucht:} Eine TM $\Smach{M}$ mit $\Slang{L}(\Smach{M})=\Slang{L}(G)$
\medskip\pause

\emph{Idee:}
\begin{itemize}
\item Turingmaschinen können Ableitungsregeln anwenden
\item Bandinhalt: Zwischenstand der Ableitung (aus Terminalen und Nichtterminalen)
\item Ableitungsregel wird nichtdeterministisch gewählt
\item TMs beginnen mit dem von der Grammatik erzeugten Wort\\$\leadsto$ Ableitungsregeln werden rückwärts angewendet
\end{itemize}

\end{frame}

\begin{frame}\frametitle{Typ 0 $\Rightarrow$ TM (Details)}

Die TM für Grammatik $G=\tuple{V,\Sigma,P,S}$ arbeitet wie folgt:
\begin{itemize}
\item Eingabealphabet $\Sigma$
\item Bandalphabet $\Gamma=\Sigma\cup V\cup\{\blank\}$
\item Arbeitsweise:
	\begin{enumerate}[(1)]
	\item Wähle (nichtdeterministisch) eine Regel $u\to v\in P$ aus
	\item Finde (nichtdeterministisch) auf dem Band ein Vorkommen von $v$
	\item Ersetze das gewählte $v$ durch $u$ (dabei muss der restliche Bandinhalt verschoben werden, wenn $|u|\neq|v|$)
	\item Wiederhole ab (1) bis entweder (a) das Band nur noch $S$ enthält (Akzeptanz) oder (b) kein Vorkommen von $v$ gefunden wird (Ablehnung)
	\end{enumerate}
\end{itemize}

Offenbar gilt: Die TM bei Eingabe $w$ hat genau dann einen erfolgreichen Lauf wenn die Grammatik eine Ableitung von $w$ zulässt.\qed


\end{frame}

\begin{frame}\frametitle{Typ 0 $\Leftarrow$ TM}

\emph{Gegeben:} Eine TM $\Smach{M}$
\medskip

\emph{Gesucht:} Eine Grammatik $G$ mit $\Slang{L}(G)=\Slang{L}(\Smach{M})$
\medskip\pause

\emph{Idee:}
\begin{itemize}
\item Ein Wort kann die Konfiguration einer TM kodieren
\item Berechnungsschritte können durch Ersetzungen von Teilwörtern simuliert werden\pause
\item Grammatiken müssen die Wörter erzeugen, welche die TM akzeptiert\\ $\leadsto$ Vorgehen einer Grammatik:
\begin{enumerate}[(1)]
\item wähle ein beliebiges Eingabewort (nichtdeterministisch)
\item simuliere die TM auf dieser Eingabe
\item Falls TM akzeptiert: ersetze die simulierte Endkonfiguration durch das ursprüngliche Eingabewort
% \item (Falls TM nicht akzeptiert: tue nichts, so dass kein Wort aus Terminalen erzeugt wird)
\end{enumerate}
\end{itemize}

\end{frame}

% \newcommand{\vtriple}[3]{\left(\begin{array}{c}#1\\#2\\#3\end{array}\right)}
\newcommand{\vtriple}[3]{\small\begin{pmatrix}#1\\[-1.3ex]#2\\[-1.3ex]#3\end{pmatrix}}

\begin{frame}\frametitle{Typ 0 $\Leftarrow$ TM (Details 1)}

\emph{Kodierungstrick:}
\begin{itemize}
\item Variablen von $G$ kodieren dreierlei Informationen:
\begin{enumerate}[(1)]
\item Ursprüngliches Eingabeband: ein Zeichen aus $\Sigma\cup\{\blank\}$ 
\item Simuliertes Arbeitsband: ein Zeichen aus $\Gamma$ 
\item Simulierte Position und Zustand: ein Zeichen aus $Q\cup\{-\}$
\end{enumerate}\pause
\examplebox{Beispiel: Die Zeichenfolge $\vtriple{\Sterm{a}}{X}{-}\vtriple{\Sterm{a}}{\Sterm{a}}{-}\vtriple{\Sterm{b}}{X}{q}\vtriple{\Sterm{b}}{\Sterm{b}}{-}\vtriple{\blank}{X}{-}\vtriple{\blank}{\blank}{-}$ kodiert:\\[1ex]
(a) die Eingabe war $\Sterm{aabb}$,\\
(b) die aktuell simulierte Konfiguration ist $X\Sterm{a}\,q\,X\Sterm{b}X\blank$}\pause
% 
\item Außerdem verwenden wir Variablen $\Snterm{S}$ (Start), $\Snterm{A}$, $\Snterm{B}$ (Erzeugung der Startkonfiguration),
$\blank$ (entsteht beim Aufräumen nach akzeptierender Endkonfiguration)
\end{itemize}

\[\leadsto V=\{\Snterm{S},\Snterm{A},\Snterm{B},\blank\} \cup\Big( (\Sigma\cup\blank)\times\Gamma\times(Q\cup\{-\})\Big)\]

\end{frame}

\begin{frame}\frametitle{Typ 0 $\Leftarrow$ TM (Details 2)}

\emph{Phase 1:} Initialisiere TM für eine beliebige Eingabe:
\begin{align*}
\Snterm{S} &\to \vtriple{\Sterm{a}}{\Sterm{a}}{q_0}\Snterm{A}\;\; (\text{für beliebige }\Sterm{a}\in\Sigma)\;\mid\; \vtriple{\blank}{\blank}{q_0}\Snterm{B}\\
\Snterm{A} &\to \vtriple{\Sterm{a}}{\Sterm{a}}{-}\Snterm{A}\;\; (\text{für beliebige }\Sterm{a}\in\Sigma)\;\mid\;\Snterm{B}\\
\Snterm{B} &\to \vtriple{\blank}{\blank}{-}\Snterm{B}\;\mid\;\epsilon
\end{align*}
\begin{enumerate}[$\leadsto$]
\item erzeugt Eingabewort und einen beliebig langen (leeren) Arbeitsspeicher
\item Spur 1 speichert geratene Eingabe
\item Spuren 2 und 3 speichern TM-Startkonfiguration bei dieser Eingabe
\end{enumerate}

\end{frame}

\newcommand{\unemph}[1]{\textcolor{devilscss}{#1}}

\begin{frame}\frametitle{Typ 0 $\Leftarrow$ TM (Details 3)}

\emph{Phase 2:} Simuliere TM-Berechnung auf Spuren 2 und 3:

\begin{itemize}
\item Für jeden TM-Übergang $\tuple{q',y,R}\in\delta(q,x)$, beliebige $\unemph{a},\unemph{b}\in\Sigma\cup\{\blank\}$ und beliebige $\unemph{z}\in\Sigma\cup\Gamma$:
\[ \vtriple{\unemph{a}}{x}{q}\vtriple{\unemph{b}}{\unemph{z}}{-} \to \vtriple{\unemph{a}}{y}{-}\vtriple{\unemph{b}}{\unemph{z}}{q'} \]\pause
%
\item Für jeden TM-Übergang $\tuple{q',y,L}\in\delta(q,x)$, beliebige $\unemph{a},\unemph{b}\in\Sigma\cup\{\blank\}$ und beliebige $\unemph{z}\in\Sigma\cup\Gamma$
\[ \vtriple{\unemph{a}}{\unemph{z}}{-}\vtriple{\unemph{b}}{x}{q} \to \vtriple{\unemph{a}}{\unemph{z}}{q'}\vtriple{\unemph{b}}{y}{-} \]\pause
%
\item Für jeden TM-Übergang $\tuple{q',y,N}\in\delta(q,x)$ und $\unemph{a}\in\Sigma\cup\{\blank\}$:
\[ \vtriple{\unemph{a}}{x}{q} \to \vtriple{\unemph{a}}{y}{q'} \]
\end{itemize}
\ghost{\hspace{13.8cm}\rotatebox{90}{\tiny \hspace{5mm}Anmerkung: In Phase 1 vorbereiteter Bandbereich kann nicht verlassen werden!}}

\end{frame}

\begin{frame}\frametitle{Typ 0 $\Leftarrow$ TM (Details 4)}

\emph{Phase 3:} Akzeptanz und Aufräumen:

\begin{itemize}
\item Für alle $q\in F$ und $x\in\Sigma\cup\Gamma$ mit $\delta(q,x)=\emptyset$ und beliebige $\unemph{a}\in\Sigma\cup\{\blank\}$:\vspace{-0.5ex}
\[ \vtriple{\unemph{a}}{x}{q} \to \unemph{a} \]\vspace{-1.5ex}\pause
%
\item Für alle $a,b\in\Sigma\cup\{\blank\}$ und beliebige $\unemph{x}\in\Sigma\cup\Gamma$:
\begin{align*}
a\vtriple{b}{\unemph{x}}{-} &\to ab &
\vtriple{b}{\unemph{x}}{-}a &\to ba
\end{align*}\vspace{-1.5ex}\pause%
%
\item Dabei erzeugte Blanks werden entfernt:\vspace{-0.5ex}
\[\blank \to \epsilon\]\vspace{-3.5ex}
\end{itemize}\pause

Diese Grammatik erzeugt ein Wort $w$ genau dann wenn die TM einen akzeptierenden Lauf für $w$ hat (unter Verwendung von 
beliebig viel Speicher).\qed\bigskip

\end{frame}

\begin{frame}\frametitle{Zusammenfassung Typ 0}

Wir haben also gezeigt:\medskip

\theobox{\emph{Satz:} Die Typ-0-Grammatiken erzeugen genau diejenigen Sprachen, die von einer Turingmaschine erkannt werden können.}\medskip

Der Satz von Rice ist daher auf Typ-0-Grammatiken übertragbar:

\theobox{\emph{Satz (informell):} Für eine gegebene Typ-0-Grammatik $G$ und eine nichttriviale Eigenschaft $E$ von Typ-0-Sprachen ist es unentscheidbar, ob $\Slang{L}(G)$ die Eigenschaft $E$ hat.}

Probleme wie Leerheit, Universalität, Äquivalenz zu einer anderen Typ-0-Grammatik, usw.\ sind daher für Typ-0-Grammatiken (wie auch für TMs) unentscheidbar

\end{frame}

\sectionSlide{Typ-1-Sprachen}

\begin{frame}\frametitle{Automaten für Typ 1?}

Welches Berechnungsmodell entspricht den Typ-1-Sprachen?
\begin{itemize}
\item Kellerautomat: zu schwach (Typ 2)
\item Turingmaschine: zu stark (Typ 0)
\end{itemize}\pause

\emph{Lösung:} Beschränkung des Arbeitsspeichers einer TM:\medskip

\defbox{Eine \redalert{linear beschränkte Turingmaschine} (\alert{linear-bounded automaton}, \alert{LBA}) ist eine nichtdeterministische Turingmaschine, die den Lese-/Schreibkopf nicht über das letzte Eingabezeichen hinaus
bewegen kann. Versucht sie das, so bleibt der Kopf stattdessen an der letzten Bandstelle stehen.}\medskip

Ein LBA kann also nur die (lineare) Menge an Speicher nutzen, die durch die Eingabe belegt wird

\end{frame}

\begin{frame}\frametitle{Beispiel}

Die TM zur Erkennung von $\{\Sterm{a}^i\Sterm{b}^i\Sterm{c}^i\mid i\geq 0\}$ (Vorlesung 18) ist ein \ghost{LBA.}
\bigskip

\footnotesize
Arbeitsweise:
\begin{enumerate}[(1)]
% \item Falls $\blank$ gelesen wird, akzeptiere
\item Ersetze, angefangen von links, Vorkommen von $\Sterm{a}$ durch $\Snterm{$\hat{a}$}$
\item Immer wenn ein $\Sterm{a}$ ersetzt wurde, suche ein $\Sterm{b}$ und ersetze es durch $\Snterm{$\hat{b}$}$,
suche anschließend rechts davon ein $\Sterm{c}$ und ersetze es durch $\Snterm{$\hat{c}$}$
\item Gehe danach zurück zum ersten noch nicht ersetzten $\Sterm{a}$ und führe die Ersetzung (1) fort, bis alle $\Sterm{a}$ ersetzt worden sind
\item Akzeptiere, falls der Inhalt des Bandes die Form $\Snterm{$\hat{a}$}^*\Snterm{$\hat{b}$}^*\Snterm{$\hat{c}$}^*$ hat
\item Andernfalls oder falls eine der Ersetzungen in Schritt (2) fehlschlägt, weil es zu wenige $\Sterm{b}$ oder $\Sterm{c}$ gibt, lehne die Eingabe ab
\end{enumerate}



\end{frame}

\begin{frame}\frametitle{Typ 1 $\Leftrightarrow$ LBA}

% Linear beschränkte Turingmaschinen charakterisieren die Typ-1-Sprachen:\medskip

\redalert{Anmerkung:} Wir beschränken uns auf Typ-1-Spachen ohne das Wort $\epsilon$. Diesen Sonderfall 
müssten LBAs anders behandeln, da eine TM nicht mit $0$ Speicherzellen arbeiten kann. Das ist nicht schwer,\footnote{Z.B. durch Verwendung eines Endzeichens nach der Eingabe.} aber auch nicht sehr interessant.\pause\medskip

\theobox{\emph{Satz:} Die Typ-1-Grammatiken erzeugen genau diejenigen Sprachen, die von einem LBA erkannt werden können.}\medskip\pause

\emph{Beweis:} Wir können fast die gleichen Konstruktionen anwenden, wie bei Typ 0:
\begin{enumerate}[(1)]
\item \alert{Typ 1 $\Rightarrow$ LBA:} Eine TM kann wie zuvor Grammatikregeln rückwärts anwenden. Bei Typ-1-Regeln ist
sichergestellt, dass dabei niemals mehr Speicher benutzt wird als am Anfang
\item \alert{LBA $\Rightarrow$ Typ 1:} Die Konstruktion liefert schon fast eine Typ-1-Grammatik \ldots
\end{enumerate}

\end{frame}

\begin{frame}\frametitle{Typ 1 $\Leftarrow$ LBA (1)}

Die zuvor verwendete TM-Grammatik auf einen Blick:

\begin{center}
$
\Snterm{S} \to \vtriple{\Sterm{a}}{\Sterm{a}}{q_0}\Snterm{A}\;\; (\text{für beliebige }\Sterm{a}\in\Sigma)\;\mid\; \vtriple{\blank}{\blank}{q_0}\Snterm{B}$\\[1ex]
%
$\Snterm{A} \to \vtriple{\Sterm{a}}{\Sterm{a}}{-}\Snterm{A}\;\; (\text{für beliebige }\Sterm{a}\in\Sigma)\;\mid\;\Snterm{B}$\hfill$\Snterm{B} \to \vtriple{\blank}{\blank}{-}\Snterm{B}\;\mid\;\epsilon$\\[2ex]
%
$\vtriple{\unemph{a}}{x}{q}\vtriple{\unemph{b}}{\unemph{z}}{-} \to \vtriple{\unemph{a}}{y}{-}\vtriple{\unemph{b}}{\unemph{z}}{q'}$\hfill$\vtriple{\unemph{a}}{\unemph{z}}{-}\vtriple{\unemph{b}}{x}{q} \to \vtriple{\unemph{a}}{\unemph{z}}{q'}\vtriple{\unemph{b}}{y}{-}$\hfill$\vtriple{\unemph{a}}{x}{q} \to \vtriple{\unemph{a}}{y}{q'}$\\[2.5ex]
% 
~\hfill$\vtriple{\unemph{a}}{x}{q} \to \unemph{a}$\hfill$a\vtriple{b}{\unemph{x}}{-} \to ab$\hfill$\vtriple{b}{\unemph{x}}{-}a \to ba$\hfill$\blank \to \epsilon$\hfill~
\end{center}
Problematisch für Typ 1 sind nur die beiden $\epsilon$-Regeln, die aber nur wegen der zusätzlichen Blanks nötig sind.

\end{frame}

\begin{frame}\frametitle{Typ 1 $\Leftarrow$ LBA (2)}

Modifizierte Grammatik zur Simulation von LBAs:

\begin{center}
$
\Snterm{S} \to \vtriple{\Sterm{a}}{\Sterm{a}}{q_0}\Snterm{A}\;\; (\text{für beliebige }\Sterm{a}\in\Sigma)\;\mid\;\vtriple{\Sterm{a}}{\Sterm{a}}{q_0}\;\; (\text{für beliebige }\Sterm{a}\in\Sigma)$\\[1ex]
%
$\Snterm{A} \to \vtriple{\Sterm{a}}{\Sterm{a}}{-}\Snterm{A}\;\; (\text{für beliebige }\Sterm{a}\in\Sigma)\;\mid\;\vtriple{\Sterm{a}}{\Sterm{a}}{-}\;\; (\text{für beliebige }\Sterm{a}\in\Sigma)$\\[2ex]
%
$\vtriple{\unemph{a}}{x}{q}\vtriple{\unemph{b}}{z}{-} \to \vtriple{\unemph{a}}{y}{-}\vtriple{\unemph{b}}{\unemph{z}}{q'}$\hfill$\vtriple{\unemph{a}}{\unemph{z}}{-}\vtriple{\unemph{b}}{x}{q} \to \vtriple{\unemph{a}}{\unemph{z}}{q'}\vtriple{\unemph{b}}{y}{-}$\hfill$\vtriple{\unemph{a}}{x}{q} \to \vtriple{\unemph{a}}{y}{q'}$\\[2.5ex]
% 
~\hfill$\vtriple{\unemph{a}}{x}{q} \to \unemph{a}$\hfill$a\vtriple{b}{\unemph{x}}{-} \to ab$\hfill$\vtriple{b}{\unemph{x}}{-}a \to ba$\hfill~
\end{center}
Diese Grammatik simuliert wie zuvor beliebige (N)TMs, aber nur auf dem Speicherbereich, der von der Eingabe belegt wird.
\qed

\end{frame}

\begin{frame}\frametitle{Konfigurationsgraphen}

Das Wortproblem bei Typ 0 ist unentscheidbar. Und bei Typ 1?\pause
\medskip

\emph{Beobachtung:}
\begin{itemize}
\item Auf einem beschränkten Speicher gibt es nur beschränkt viele Konfigurationen, genauer gesagt:
	\[ \text{Konfigurationszahl bei $n$ Zellen: }\underbrace{|\Gamma|^n}_{\text{Bandinhalt}} \cdot \underbrace{\phantom{|}n\phantom{|}}_{\text{Kopfpositionen}}\cdot \underbrace{|Q|}_{\text{Zustände}} \]\pause
\item Man kann entscheiden, ob eine TM von einer Konfiguration in eine andere wechseln kann oder nicht
% \item Man kann entscheiden, ob eine Konfiguration eine akzeptierende Endkonfiguration ist
\end{itemize}\pause
Für eine Eingabe $w$ können wir also den kompletten Graphen aller möglichen LBA-Konfigurationen und Übergänge
berechnen.\\
$\leadsto$ \redalert{Konfigurationsgraph}

\end{frame}

\begin{frame}\frametitle{Das Wortproblem für Typ 1}

\emph{Wortproblem (anders ausgedrückt):} Gibt es eine akzeptierende
Endkonfiguration, die im Konfigurationsgraphen von der Startkonfiguration aus erreichbar ist?
\medskip\pause

Daraus folgt:\medskip

\theobox{\emph{Satz:} Das Wortproblem für Typ-1-Sprachen ist entscheidbar.}

\emph{Beweis:} Man kann den folgenden Algorithmus anwenden: (1) Berechne den (exponentiell großen) Konfigurationsgraph einer entsprechenden Turingmaschine für das gegebene Wort; (2) prüfe, ob es in diesem Graphen einen Pfad von der Startkonfiguration zu einer Endkonfiguration gibt.\qed
\medskip

Unser Algorithmus benötigt (immer) exponentiell viel Zeit.
\medskip

\alert{Aber:} Es ist bis heute nicht bekannt, ob es einen Algorithmus gibt, der im Worst-Case weniger als exponentiell viel Zeit benötigt!
\bigskip\pause

\examplebox{Beispiel: Das Halteproblem ist keine Typ-1-Sprache, da es nicht entscheidbar ist.}

\end{frame}

\sectionSlide{Abschlusseigenschaften\\Typ~0 und Typ~1}

\begin{frame}\frametitle{Bekannte Abschlusseigenschaften}

Wir wissen bereits:\medskip

\theobox{\emph{Satz (siehe Vorlesung 14):} Sowohl die Klasse der Typ-1-Sprachen als auch die Klasse der Typ-0-Sprachen ist unter Vereinigung abgeschlossen.}\medskip\pause

\theobox{\emph{Satz:} Die Klasse der Typ-0-Sprachen ist \redalert{nicht} unter Komplement abgeschlossen.}\medskip

\emph{Beweis:} Das Komplement des Halteproblems ist nicht semi-entscheidbar (siehe Vorlesung 19).\qed
% \pause\bigskip

% Für allgemeine TMs ist es zudem nicht schwer, eine TM für Schnitt, Konkatenation und Kleene-Stern zu konstruieren:
% \medskip

% \theobox{Satz: Die Klasse der Typ-0-Sprachen ist unter Schnitt, Konkatenation und Kleene-Stern abgeschlossen.}

\end{frame}

\begin{frame}\frametitle{Schnitt, Konkatenation und Kleene-Stern}

Weitere Abschlusseigenschaften sind nicht schwer zu finden:

\begin{itemize}
\item \alert{Schnitt:} Simuliere erst die erste TM, dann (bei Akzeptanz) die zweite; verwende ein "`mehrspuriges"' Alphabet, um die Eingabe für die zweite Simulation zu speichern
\item \alert{Konkatenation:} Rate und markiere die Trennstelle der beiden Wörter; teste dann jedes der Wörter einzeln
\item \alert{Kleene-Stern:} Rate und teste einen ersten nichtleeren Teilabschnitt; wiederhole dies bis das gesamte Wort erkannt wurde
\end{itemize}\pause

Diese Konstruktionen funktionieren auch bei linear beschränktem Speicher, also:\medskip

\theobox{\emph{Satz:} Sowohl die Klasse der Typ-1-Sprachen als auch die Klasse der Typ-0-Sprachen ist unter Schnitt, Konkatenation und Kleene-Stern abgeschlossen.}

\end{frame}

\begin{frame}\frametitle{Die LBA-Probleme}

Zwei Probleme sind schon seit Erfindung der LBAs bekannt (Kuroda, 1964):
\begin{enumerate}[(1)]
\item Erkennen LBA dieselben Sprachen wie deterministische LBA?
\item Sind die von LBA erkennbaren Sprachen unter Komplement abgeschlossen?
\end{enumerate}\pause

Das zweite Problem lösten überraschend nach über 20 Jahren unabhängig voneinander Robert Szelepcsényi (1987) und Neil Immerman (1988):\medskip

\theobox{\emph{Satz von Immerman und Szelepcsényi:} Die Typ-1-Sprachen sind unter Komplement abgeschlossen.}\medskip

\emph{Beweis:} siehe Sipser (Abschnitt 8.6) oder Schöning (Abschnitt 1.4) oder Kurs \alert{Complexity Theory} der TU Dresden; kein Stoff dieses Kurses.
\medskip\pause

Das erste LBA-Problem ist bis heute ungelöst.

\end{frame}

\newcommand{\myyes}{$\textcolor{darkgreen}{\checkmark}$}
\newcommand{\myno}{$\textcolor{darkred}{\times}$}

\begin{frame}\frametitle{Übersicht Abschlusseigenschaften}

\begin{center}
\begin{tabular}{r|ccccc|l}
	& \multicolumn{5}{c|}{Abschluss unter \ldots} &\\
Sprache & $\cap$ & $\cup$ & $\overline{\phantom{L}}$ & $\circ$ & $^*$ & Automat\\\hline
Typ 0 & \myyes & \myyes & \myno & \myyes & \myyes & TM (DTM/NTM)\\
Typ 1 & \myyes & \myyes & \myyes & \myyes & \myyes & LBA ($\stackrel{?}{=}$ det. LBA)\\
Typ 2 & \myno & \myyes & \myno & \myyes & \myyes & PDA\\
Det. Typ 2 & \myno & \myno & \myyes & \myno & \myno & DPDA\\
Typ 3 & \myyes & \myyes & \myyes & \myyes & \myyes & DFA/NFA
\end{tabular}
\end{center}

\end{frame}


\begin{frame}\frametitle{Zusammenfassung und Ausblick}

\redalert{Turingmaschinen} charakterisieren Typ-0-Sprachen.
\bigskip

\redalert{Linear beschränkte Turingmaschinen} charakterisieren Typ-1-Sprachen.
\bigskip

Das \redalert{Wortproblem für Typ-1-Sprachen} ist entscheidbar aber kompliziert
\bigskip

\anybox{yellow}{
Offene Fragen:
\begin{itemize}
\item Wollten wir nicht auch noch etwas Logik behandeln?
\item Was hat das mit Sprachen, Berechnung und TMs zu tun?
\end{itemize}
}

\end{frame}


\end{document}
