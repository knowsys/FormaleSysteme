\documentclass[aspectratio=1610,onlymath]{beamer}
% \documentclass[aspectratio=1610,onlymath,handout]{beamer}

\input{macros-lecture}
% Common notation

\usepackage{amsmath,amssymb,amsfonts}
\usepackage{xspace}

\newcommand{\lectureurl}{https://iccl.inf.tu-dresden.de/web/FS2025}

\DeclareMathAlphabet{\mathsc}{OT1}{cmr}{m}{sc} % Let's have \mathsc since the slide style has no working \textsc

% Dual of "phantom": make a text that is visible but intangible
\newcommand{\ghost}[1]{\raisebox{0pt}[0pt][0pt]{\makebox[0pt][l]{#1}}}

\newcommand{\tuple}[1]{\langle{#1}\rangle}
\newcommand{\defeq}{\mathrel{:=}}

%%% Annotation %%%

\usepackage{color}
\newcommand{\todo}[1]{{\tiny\color{red}\textbf{TODO: #1}}}



%%% Old macros below; move when needed

\newcommand{\blank}{\text{\textvisiblespace}} % empty tape cell for TM

% table syntax
\newcommand{\dom}{\textbf{dom}}
\newcommand{\adom}{\textbf{adom}}
\newcommand{\dbconst}[1]{\texttt{"#1"}}
\newcommand{\pred}[1]{\textsf{#1}}
\newcommand{\foquery}[2]{#2[#1]}
\newcommand{\ground}[1]{\textsf{ground}(#1)}
% \newcommand{\foquery}[2]{\{#1\mid #2\}} %% Notation as used in Alice Book
% \newcommand{\foquery}[2]{\tuple{#1\mid #2}}

\newcommand{\quantor}{\mathord{\reflectbox{$\text{\sf{Q}}$}}} % the generic quantor

% logic syntax
\newcommand{\Inter}{\mathcal{I}} %used to denote an interpretation
\newcommand{\Jnter}{\mathcal{J}} %used to denote another interpretation
\newcommand{\Knter}{\mathcal{K}} %used to denote yet another interpretation
\newcommand{\Zuweisung}{\mathcal{Z}} %used to denote a variable assignment

% query languages
\newcommand{\qlang}[1]{{\sf #1}} % Font for query languages
\newcommand{\qmaps}[1]{\textbf{QM}({\sf #1})} % Set of query mappings for a query language

%%% Complexities %%%

\hyphenation{Exp-Time} % prevent "Ex-PTime" (see, e.g. Tobies'01, Glimm'07 ;-)
\hyphenation{NExp-Time} % better that than something else

% \newcommand{\complclass}[1]{{\sc #1}\xspace} % font for complexity classes
\newcommand{\complclass}[1]{\ensuremath{\mathsc{#1}}\xspace} % font for complexity classes

\newcommand{\ACzero}{\complclass{AC$_0$}}
\newcommand{\LogSpace}{\complclass{L}}
\newcommand{\NLogSpace}{\complclass{NL}}
\newcommand{\PTime}{\complclass{P}}
\newcommand{\NP}{\complclass{NP}}
\newcommand{\coNP}{\complclass{coNP}}
\newcommand{\PH}{\complclass{PH}}
\newcommand{\PSpace}{\complclass{PSpace}}
\newcommand{\NPSpace}{\complclass{NPSpace}}
\newcommand{\ExpTime}{\complclass{ExpTime}}
\newcommand{\NExpTime}{\complclass{NExpTime}}
\newcommand{\ExpSpace}{\complclass{ExpSpace}}
\newcommand{\TwoExpTime}{\complclass{2ExpTime}}
\newcommand{\NTwoExpTime}{\complclass{N2ExpTime}}
\newcommand{\ThreeExpTime}{\complclass{3ExpTime}}
\newcommand{\kExpTime}[1]{\complclass{#1ExpTime}}
\newcommand{\kExpSpace}[1]{\complclass{#1ExpSpace}}


\defineTitle{26}{Zusammenfassung und Ausblick}{25. Januar 2024}

\begin{document}

\maketitle

% \sectionSlide{Zusammenfassung}

\frame{\begin{center}
{\LARGE
Zusammenfassung}\bigskip

~\hfill
\includegraphics[height=4.5cm]{images/xkcd-tsp}
\hfill~

\rotatebox{0}{\tiny Randall Munroe, \url{http://xkcd.com/399/}, CC-BY-NC 2.5}
\end{center}}

% \sectionSlide{Fragen?}

\begin{frame}\frametitle{Sprachen und Berechnung}

\begin{itemize}
\item Formale Wörter als allgemeine Abstraktion aller Daten, die in Computern verarbeitet werden können
\item Formale Sprachen als Mengen von Ein- oder Ausgaben
\item Worterkennung als allgemeine Berechnungsaufgabe
\end{itemize}

\examplebox{Beispiel: Auch die Berechnung von Funktionen kann als Wortproblem ausgedrückt werden. 
Anstatt zu fragen, "`Was ergibt $n+m$?"' kann man fragen "`Ist $n+m=r$?"'}

Daraus ergibt sich das Kernthema diese Vorlesung:\\[2ex]

\narrowcentering{\alert{Sprachen zu klassifizieren heißt Rechenaufgaben klassifizieren}}

\end{frame}

\begin{frame}\frametitle{Zwei Hierarchien}

Wir haben zwei Hierarchien von Sprachklassen kennengelernt\bigskip

\begin{enumerate}[(1)]
\item \alert{Chomsky-Hierarchie:}
\[ \text{Typ 3}\subset \text{det. kontextfrei}\subset\text{Typ 2}\subset\text{Typ 1}\subset\text{Typ 0}\]
Ansatz: natürliche Definition über Grammatiken\bigskip
\item \alert{Hierarchie der Komplexitätsklassen:}
\[\Scomplclass{L}\subseteq\Scomplclass{NL}\subseteq \Scomplclass{P}\subseteq\Scomplclass{NP}\subseteq\Scomplclass{PSpace}\subseteq \Scomplclass{Exp}\subseteq \Scomplclass{NExp}\]
Ansatz: natürliche und robuste Definition durch Beschränkung von Turingmaschinen
\end{enumerate}

\end{frame}

\begin{frame}[t]\frametitle{Eine Hierarchie?\phantom{p}}

\narrowcentering
{%
\scalebox{0.75}{%
\begin{tikzpicture}[decoration=penciline, decorate,
	mybox/.style args = {}{
		rectangle,rounded corners=5pt,minimum width=2.5cm,draw=black,fill=cyan!40,
		align=flush center
	},
	myarrow/.style args = {}{
		line width=0.4mm,
		draw=black
	}
]
\pgfmathsetseed{7729}
\path[use as bounding box] (-4,-6) rectangle (9,6); % add "draw" to see it
% \draw[help lines] (0,0) grid (5,5);

\node (nexp) [mybox=] at (0,4) {\Scomplclass{NExp}};
\node (exp) [mybox=] at (0,3) {\Scomplclass{Exp}};
\node (pspace) [mybox=] at (0,2) {\Scomplclass{PSpace}};
\node (lsavitch) [draw=none,fill=none,circle] at (1.5,2) {=};
\node (npspace) [mybox=] at (3,2) {\Scomplclass{NPSpace}};
\node (np) [mybox=] at (0,1) {\Scomplclass{NP}};
\node (p) [mybox=] at (0,0) {\Scomplclass{P}};
\node (nl) [mybox=] at (0,-1) {\Scomplclass{NL}};
\node (l) [mybox=] at (0,-2.3) {\Scomplclass{L}};

\node (re) [mybox=] at (2,5.5) {RE, Typ 0\\[-0.6ex]{\footnotesize(semi-entscheidbar, Turing-erkennbar)}};
\node (cs) [mybox=] at (4,0.5) {Typ 1, CSL\\[-0.6ex]{\footnotesize(kontextsensitiv)}};
\node (lcslspace) [draw=none,fill=none,circle] at (5.5,0.5) {=};
\node (nspacen) [mybox=] at (7,0.5) {\Scomplclass{NSpace}$(n)$};

\node (cf) [mybox=] at (4,-1) {Typ 2, CFL\\[-0.6ex]{\footnotesize(kontextfrei)}};
\node (dcf) [mybox=] at (4,-2.3) {DCFL\\[-0.6ex]{\footnotesize(det. kontextf.)}};
\node (reg) [mybox=] at (4,-3.6) {Typ 3, REG\\[-0.6ex]{\footnotesize(regulär)}};
\node (lregtime) [draw=none,fill=none,circle] at (5.5,-3.6) {=};
\node (nspacen) [mybox=] at (7,-3.6) {\Scomplclass{DSpace}$(1)$};

\draw[myarrow=] (l)--(nl);
\draw[myarrow=] (nl)--(p);
\draw[myarrow=] (p)--(np);
\draw[myarrow=] (np)--(pspace);
\draw[myarrow=] (pspace)--(exp);
\draw[myarrow=] (exp)--(nexp);
\draw[myarrow=] (nexp)--(re);

\draw[myarrow=] (reg)--(dcf);
\draw[myarrow=] (dcf)--(cf);
\draw[myarrow=] (cf)--(cs);
\draw[myarrow=] (cs)--(pspace);
\draw[myarrow=] (reg)--(l);
\draw[myarrow=] (cf)--(p);
\draw[myarrow=] (nl)--(cs);

\node (legend) [rectangle,text width=3.7cm,draw=darkred,fill=strongyellow!40,
		align=flush left] at (7.5,4) {Bekannte Beziehungen (Mengeninklusionen) zwischen den Sprachklassen; zumeist geht man von echten Teilmengen aus, aber bewiesen ist das nur in manchen Fällen};
\end{tikzpicture}}}


\end{frame}

\begin{frame}[t]\frametitle{Typische Beispiel-Sprachen}

\ghost{\hspace{2.0cm}\raisebox{6.3cm}{\includegraphics[height=1.5cm]{images/Sudoku-xkcd74}$*$}
\hspace{10.5cm}\raisebox{1.5cm}{\rotatebox{90}{\tiny $*$ Randall Munroe, \url{http://xkcd.com/74/}, CC-BY-NC 2.5}}}%
\narrowcentering
{%
\scalebox{0.75}{%
\begin{tikzpicture}[decoration=penciline, decorate,
	mybox/.style args = {}{
		rectangle,rounded corners=5pt,minimum width=2.5cm,draw=black,fill=cyan!40,
		align=flush center
	},
	myexbox/.style args = {}{
		rectangle,rounded corners=1pt,minimum width=2cm,draw=darkgreen,fill=darkgreen!40,
		align=flush center
	},
	myarrow/.style args = {}{
		line width=0.4mm,
		draw=black
	}
]
\pgfmathsetseed{7729}
\path[use as bounding box] (-4,-6) rectangle (9,6); % add "draw" to see it
% \draw[help lines] (0,0) grid (5,5);

\node (nexp) [mybox=] at (0,4) {\Scomplclass{NExp}};
\node (exp) [mybox=] at (0,3) {\Scomplclass{Exp}};
\node (pspace) [mybox=] at (0,2) {\Scomplclass{PSpace}};
% \node (lsavitch) [draw=none,fill=none,circle] at (1.5,2) {=};
% \node (npspace) [mybox=] at (3,2) {\Scomplclass{NPSpace}};
\node (np) [mybox=] at (0,1) {\Scomplclass{NP}};
\node (p) [mybox=] at (0,0) {\Scomplclass{P}};
\node (nl) [mybox=] at (0,-1) {\Scomplclass{NL}};
\node (l) [mybox=] at (0,-2.3) {\Scomplclass{L}};

\node (re) [mybox=] at (2,5.5) {RE, Typ 0\\[-0.6ex]{\footnotesize(semi-entscheidbar, Turing-erkennbar)}};
\node (cs) [mybox=] at (4,0.5) {Typ 1, CSL\\[-0.6ex]{\footnotesize(kontextsensitiv)}};
% \node (lsavitch) [draw=none,fill=none,circle] at (5.5,0.5) {=};
% \node (nspacen) [mybox=] at (7,0.5) {\Scomplclass{NSpace}$(n)$};

\node (cf) [mybox=] at (4,-1) {Typ 2, CFL\\[-0.6ex]{\footnotesize(kontextfrei)}};
\node (dcf) [mybox=] at (4,-2.3) {DCFL\\[-0.6ex]{\footnotesize(det. kontextf.)}};
\node (reg) [mybox=] at (4,-3.6) {Typ 3, REG\\[-0.6ex]{\footnotesize(regulär)}};
% \node (lsavitch) [draw=none,fill=none,circle] at (5.5,-3.6) {=};
% \node (nspacen) [mybox=] at (7,-3.6) {\Scomplclass{DSpace}$(1)$};

\draw[myarrow=] (l)--(nl);
\draw[myarrow=] (nl)--(p);
\draw[myarrow=] (p)--(np);
\draw[myarrow=] (np)--(pspace);
\draw[myarrow=] (pspace)--(exp);
\draw[myarrow=] (exp)--(nexp);
\draw[myarrow=] (nexp)--(re);

\draw[myarrow=] (reg)--(dcf);
\draw[myarrow=] (dcf)--(cf);
\draw[myarrow=] (cf)--(cs);
\draw[myarrow=] (cs)--(pspace);
\draw[myarrow=] (reg)--(l);
\draw[myarrow=] (cf)--(p);
\draw[myarrow=] (nl)--(cs);

\node (cslex) [myexbox=] at (6.3,0.8) {$\{\Sterm{a}^i\Sterm{b}^i\Sterm{c}^i\mid i\geq 0\}$};
\node (cslex2) [myexbox=] at (6.2,0.1) {\Slang{SAT}};
\node (cflex) [myexbox=] at (7,-0.7) {$\{\Sterm{a}^i\Sterm{b}^j\Sterm{c}^k\mid i\neq j\text{ oder }j\neq k\}$};
\node (cflex2) [myexbox=] at (6.5,-1.4) {$\{w w^r\mid w\in\{\Sterm{a},\Sterm{b}\}^*\}$};
\node (dcflex) [myexbox=] at (6,-2.3) {$\{\Sterm{a}^i\Sterm{b}^i\mid i\geq 0\}$};
\node (regex) [myexbox=] at (6.2,-3.3) {$\{\Sterm{a}^i\Sterm{b}^j\mid i,j\geq 0\}$};
\node (regex2) [myexbox=] at (7.0,-4.0) {alle endlichen Sprachen};

\node (npex) [myexbox=] at (-2.0,1.3) {\Slang{SAT}};
\node (pex) [myexbox=] at (-2.0,0.3) {\Slang{HornSAT}};
\node (pex) [myexbox=] at (5.0,4.9) {Halteproblem};

% \node (legend) [rectangle,text width=3.7cm,draw=darkred,fill=strongyellow!40,
% 		align=flush left] at (7.5,4) {Bekannte Beziehungen (Mengeninklusionen) zwischen den Sprachklassen; zumeist geht man von echten Teilmengen aus, aber bewiesen ist das nur in manchen Fällen};
\end{tikzpicture}}}

\end{frame}

\begin{frame}[t]\frametitle{Berechnungsmodelle}

\narrowcentering
{%
\scalebox{0.75}{%
\begin{tikzpicture}[decoration=penciline, decorate,
	mybox/.style args = {}{
		rectangle,rounded corners=5pt,minimum width=2.5cm,draw=black,fill=cyan!40,
		align=flush center
	},
	mymachbox/.style args = {}{
		rectangle,rounded corners=2pt,minimum width=2cm,draw=darkred,fill=darkred!40,
		align=flush center
	},
	myarrow/.style args = {}{
		line width=0.4mm,
		draw=black
	}
]
\pgfmathsetseed{7729}
\path[use as bounding box] (-4,-6) rectangle (9,6); % add "draw" to see it
% \draw[help lines] (0,0) grid (5,5);

\node (nexp) [mybox=] at (0,4) {\Scomplclass{NExp}};
\node (exp) [mybox=] at (0,3) {\Scomplclass{Exp}};
\node (pspace) [mybox=] at (0,2) {\Scomplclass{PSpace}};
% \node (lsavitch) [draw=none,fill=none,circle] at (1.5,2) {=};
% \node (npspace) [mybox=] at (3,2) {\Scomplclass{NPSpace}};
\node (np) [mybox=] at (0,1) {\Scomplclass{NP}};
\node (p) [mybox=] at (0,0) {\Scomplclass{P}};
\node (nl) [mybox=] at (0,-1) {\Scomplclass{NL}};
\node (l) [mybox=] at (0,-2.3) {\Scomplclass{L}};

\node (re) [mybox=] at (2,5.5) {RE, Typ 0\\[-0.6ex]{\footnotesize(semi-entscheidbar, Turing-erkennbar)}};
\node (cs) [mybox=] at (4,0.5) {Typ 1, CSL\\[-0.6ex]{\footnotesize(kontextsensitiv)}};
% \node (lsavitch) [draw=none,fill=none,circle] at (5.5,0.5) {=};
% \node (nspacen) [mybox=] at (7,0.5) {\Scomplclass{NSpace}$(n)$};

\node (cf) [mybox=] at (4,-1) {Typ 2, CFL\\[-0.6ex]{\footnotesize(kontextfrei)}};
\node (dcf) [mybox=] at (4,-2.3) {DCFL\\[-0.6ex]{\footnotesize(det. kontextf.)}};
\node (reg) [mybox=] at (4,-3.6) {Typ 3, REG\\[-0.6ex]{\footnotesize(regulär)}};
% \node (lsavitch) [draw=none,fill=none,circle] at (5.5,-3.6) {=};
% \node (nspacen) [mybox=] at (7,-3.6) {\Scomplclass{DSpace}$(1)$};

\draw[myarrow=] (l)--(nl);
\draw[myarrow=] (nl)--(p);
\draw[myarrow=] (p)--(np);
\draw[myarrow=] (np)--(pspace);
\draw[myarrow=] (pspace)--(exp);
\draw[myarrow=] (exp)--(nexp);
\draw[myarrow=] (nexp)--(re);

\draw[myarrow=] (reg)--(dcf);
\draw[myarrow=] (dcf)--(cf);
\draw[myarrow=] (cf)--(cs);
\draw[myarrow=] (cs)--(pspace);
\draw[myarrow=] (reg)--(l);
\draw[myarrow=] (cf)--(p);
\draw[myarrow=] (nl)--(cs);

\node (cslex) [mymachbox=] at (6.1,0.8) {LBA};
\node (cflex) [mymachbox=] at (6.2,-1.0) {PDA};
\node (dcflex) [mymachbox=] at (6.2,-2.3) {DPDA};
\node (regex) [mymachbox=] at (6.2,-3.3) {NFA};
\node (regex2) [mymachbox=] at (6.2,-4.0) {DFA};

\node (npex) [mymachbox=] at (-2.0,1.3) {Polyzeit NTM};
\node (pex) [mymachbox=] at (-2.0,0.3) {Polyzeit DTM};
\node (pex) [mymachbox=] at (5.6,4.9) {DTM};
\node (pex) [mymachbox=] at (5.6,5.5) {NTM};

% \node (legend) [rectangle,text width=3.7cm,draw=darkred,fill=strongyellow!40,
% 		align=flush left] at (7.5,4) {Bekannte Beziehungen (Mengeninklusionen) zwischen den Sprachklassen; zumeist geht man von echten Teilmengen aus, aber bewiesen ist das nur in manchen Fällen};
\end{tikzpicture}}}

\end{frame}

\begin{frame}\frametitle{Trennung der Sprachklassen}

Die Chomsky-Hierarchie ist echt. Methoden, um 
\redalert{Nicht}enthaltensein einer Sprache in einer bestimmten Hierarchieebene zu zeigen:

\begin{itemize}
\item \alert{Typ 3}: reguläres Pumping-Lemma, Myhill-Nerode-Index, Abschlusseigenschaften (V10)
\item \alert{Typ 2}: kontextfreies Pumping-Lemma (V13), Abschlusseigenschaften (V14)
\item \alert{det. Typ 2}: Abschlusseigenschaften (V16)
\item \alert{Typ 1}: Entscheidbarkeit, Abschlusseigenschaften (V19/V20)
\item \alert{Typ 0}: Semi-Entscheidbarkeit (V19/V20)
\end{itemize}

Bei den Komplexitätsklassen sind bisher weitaus weniger Unterschiede bewiesen.
Exponentielle Ressourcenzugaben erzeugen echt größere Klassen (z.B. $\Scomplclass{P}\subset\Scomplclass{Exp}$).
\medskip

\textcolor{devilscss}{Aus einem analogen Ergebnis für Speicher folgt auch, dass $\Scomplclass{ExpSpace}$-harte Sprachen nicht kontextsensitiv sind, auch wenn sie entscheidbar sind}

\end{frame}


\newcommand{\myyes}{$\textcolor{darkgreen}{\checkmark}$}
\newcommand{\myno}{$\textcolor{darkred}{\times}$}

\begin{frame}\frametitle{Übersicht Abschlusseigenschaften}

\begin{center}
\begin{tabular}{r|ccccc|l}
	& \multicolumn{5}{p{3cm}|}{Abschluss unter \ldots} &\\
Sprache & $\cap$ & $\cup$ & $\overline{\phantom{L}}$ & $\circ$ & $^*$ & Automat\\\hline
Typ 0 & \myyes & \myyes & \myno & \myyes & \myyes & TM (DTM/NTM)\\
Typ 1 & \myyes & \myyes & \myyes & \myyes & \myyes & LBA ($\stackrel{?}{=}$ det. LBA)\\
Typ 2 & \myno & \myyes & \myno & \myyes & \myyes & PDA\\
Det. Typ 2 & \myno & \myno & \myyes & \myno & \myno & DPDA\\
Typ 3 & \myyes & \myyes & \myyes & \myyes & \myyes & DFA/NFA
\end{tabular}
\end{center}

\end{frame}

\definecolor{CTypeNo}{HTML}{b6786c}
\definecolor{CTypeO}{HTML}{b69d6c}
\definecolor{CTypeI}{HTML}{aab66c}
\definecolor{CTypeII}{HTML}{85b66c}
\definecolor{CTypeDII}{HTML}{6cb678}
\definecolor{CTypeIII}{HTML}{6cb69d}
%
\newcommand{\typetxt}[1]{\ghost{\textcolor{white}{#1}}\phantom{\myyes}}
%
\newcommand{\typeNo}{\cellcolor{CTypeNo}\typetxt{--}}
\newcommand{\typeO}{\cellcolor{CTypeO}\typetxt{0}}
\newcommand{\typeI}{\cellcolor{CTypeI}\typetxt{1}}
\newcommand{\typeII}{\cellcolor{CTypeII}\typetxt{2}}
\newcommand{\typeDII}{\cellcolor{CTypeDII}\typetxt{$\!\!$D2}}
\newcommand{\typeIII}{\cellcolor{CTypeIII}\typetxt{3}}

\begin{frame}\frametitle{Übersicht Abschlusseigenschaften}

\begin{center}
\begin{tabular}{r|ccccc|l}
	& \multicolumn{5}{p{3cm}|}{\ghost{Ergebnis von Typ \ldots}\phantom{Abschluss unter \ldots }} &\\
Sprache & $\cap$ & $\cup$ & $\overline{\phantom{L}}$ & $\circ$ & $^*$ & Automat\\\hline
Typ 0 & \typeO & \typeO & \typeNo & \typeO & \typeO & TM (DTM/NTM)\\
Typ 1 & \typeI & \typeI & \typeI & \typeI & \typeI & LBA ($\stackrel{?}{=}$ det. LBA)\\
Typ 2 & \typeI & \typeII & \typeI & \typeII & \typeII & PDA\\
Det. Typ 2 & \typeI & \typeII & \typeDII & \typeII & \typeII & DPDA\\
Typ 3 & \typeIII & \typeIII & \typeIII & \typeIII & \typeIII & DFA/NFA
\end{tabular}
\end{center}

\end{frame}


\begin{frame}\frametitle{Übersicht Probleme}

Die Entscheidbarkeit verschiedener relevanter Probleme ist je nach Sprachklasse
unterschiedlich:

\begin{center}
\begin{tabular}{r|cccccc}
% 	& \multicolumn{5}{c|}{Abschluss unter \ldots} &\\
Sprache & \rotatebox{90}{Wortproblem} & \rotatebox{90}{Leerheit} & \rotatebox{90}{Äquivalenz} & \rotatebox{90}{Regularität} & \rotatebox{90}{Inklusion} & \rotatebox{90}{Schnitt}\\\hline
Typ 0 & \myno & \myno & \myno & \myno & \myno & \myno\\
Typ 1 & \myyes & \myno & \myno & \myno & \myno & \myno\\
Typ 2 & \myyes & \myyes & \myno & \myno & \myno & \myno\\
Det. Typ 2 & \myyes & \myyes & \myyes & \myyes & \myno & \myno\\
Typ 3 & \myyes & \myyes & \myyes & (\myyes) & \myyes & \myyes
\end{tabular}
\end{center}

\end{frame}

\begin{frame}\frametitle{Wortprobleme lösen}

Wie schwer ist es, dass Wortproblem zu lösen, wenn die Eingabe eine (geeignet kodierte) Sprache und ein zu testendes Wort ist?\bigskip

\narrowcentering{
\begin{tabular}{rl}
Sprache & Zeitkomplexität bzgl. Wortlänge $n$\\\hline
Typ 3 & $O(n)$ (Abarbeitung DFA)\\
Det. Typ 2 & $O(n)$ (Abarbeitung DPDA)\\
Typ 2 &  $O(n^3)$ (CYK-Algorithmus)\\
\Scomplclass{P} & polynomiell (z.B. Hyperresolution für Hornlogik)\\
\Scomplclass{NP} & exponentiell (z.B. Resolution allgemein)\\
Typ 1 & $O(n\cdot |\Gamma|^n)$ (z.B. über LBA-Konfigurationsgraph)\\
Typ 0 & unentscheidbar\\
\end{tabular}}\bigskip

Das Wortproblem für \Scomplclass{NP} und Typ 1 ist \Scomplclass{NP}-vollständig bzw. \Scomplclass{PSpace}-vollständig.
Es ist nicht bewiesen aber wahrscheinlich, dass es keine subexponentiellen Algorithmen gibt.

\end{frame}

\begin{frame}\frametitle{Nichtdeterminismus}

% \begin{itemize}
% \item
\alert{Nichtdeterministische Akzeptanzbedingung:}
\\Gibt es mindestens einen Lauf, der akzeptiert?\\
% $\leadsto$ nichtsymmetrische Bedingung!
% \item Determinisierung nicht immer klar.
% \end{itemize}
\bigskip

\narrowcentering{
\begin{tabular}{ccc}
Det. & Nichtdet. & \\[-1ex]
Automatenmodell &  Automatenmodell & äquivalent?\\\hline
DFA & NFA & \myyes \\
DPDA & PDA & \myno \\
DLBA & LBA & ? \\
DTM & NTM & \myyes \\
\end{tabular}}\bigskip

\begin{itemize}
\item Oft ist es schwierig, Nichtdeterminismus unter Ressourcenbeschränkungen aufzulösen, nicht nur bei LBAs\\
z.B. ist auch $\Scomplclass{P}\neq\Scomplclass{NP}$ offen
\item Wegen der Asymmetrie hat jede nichtdeterministische Klasse eine Komplementärklasse, die oft (vermutlich) unterschiedlich ist (z.B. \Scomplclass{coNP} vs. \Scomplclass{NP})
\end{itemize}

\end{frame}

\begin{frame}\frametitle{Ein einfacher Beweis für \Scomplclass{P} = \Scomplclass{NP} ;-)}

\begin{align*}
\text{Wir wissen:}      && \Slang{L} \in \Scomplclass{P} \quad& \text{impliziert}\quad \Slang{L} \in \Scomplclass{NP}  \\
\text{Daher gilt:}  &&          \Slang{L} \notin \Scomplclass{NP} \quad& \text{impliziert}\quad \Slang{L} \notin \Scomplclass{P}\\
\text{Anders gesagt:}  &&   \Slang{L} \in \Scomplclass{coNP} \quad& \text{impliziert}\quad \Slang{L} \in \Scomplclass{coP} \\
\text{Das heißt:} &&       \Scomplclass{coNP} &\subseteq  \Scomplclass{coP}\\
\text{Wegen $\Scomplclass{coP} = \Scomplclass{P}$ gilt:}  &&            \Scomplclass{coNP} &\subseteq \Scomplclass{P}\\
\text{Somit gilt:} &&       \Scomplclass{NP} &\subseteq \Scomplclass{P}\\
\text{D.h., wegen $\Scomplclass{P} \subseteq \Scomplclass{NP}$ gilt:} &&       \Scomplclass{NP} &= \Scomplclass{P}
\end{align*}

\mbox{}\hfill q.e.d.?

\end{frame}

% Darstellung der entsprechenden Grammatiken (wo verfügbar)
% Trennung der Sprachklassen: Methoden

% \sectionSlide{Ausblick und Anwendungen}
\frame{\begin{center}
{\LARGE
Ausblick und Anwendungen}\bigskip

~\hfill
\includegraphics[height=5.5cm]{images/xkcd246-labyrinth-puzzle}
\hfill~
\rotatebox{90}{\tiny Randall Munroe, \url{http://xkcd.com/246/}, CC-BY-NC 2.5}
\end{center}}

\begin{frame}\frametitle{Formale Sprachen}

Formale Sprachen in der Praxis:
\begin{itemize}
\item Typ 3: extrem weit verbreitet in Form von regulären Ausdrücken; im \alert{Kompilerbau} als Lexer; noch einfachere Sprachen bei Anfrage/Auswahlmechanismen z.B. CSS-Selektoren
\item det. Typ 2: besonders relevant im \alert{Kompilerbau} (LR(k)-Grammatiken)
\item nichtdet. Typ 2: in der \alert{Sprachverarbeitung}; in dieser Anwendung teils auch etwas stärkere Sprachklassen (z.B. Tree-Adjoining Grammars)
\end{itemize}
Typ-1-Sprachen haben kaum praktische Anwendungen, Typ-0-Sprachen fallen mit allgemeinen TMs zusammen

\end{frame}


\begin{frame}\frametitle{Automatentheorie}

Es gibt viele Automatenmodelle jenseits der hier vorgestellten:
\begin{itemize}
\item \alert{Baumautomaten} arbeiten auf Baumstrukturen, die sie von oben oder unten her lesen
\item \alert{Automaten für unendliche Strukturen} verwenden andere Akzeptanzbedingungen, die für unendliche Abarbeitungen Sinn ergeben
\item \alert{Hybride Automaten} modellieren komplexe dynamische Systeme mithilfe von Differentialgleichungen
\item \alert{Eingeschränkte Automatenmodelle} z.B. partiell geordnete Automaten, erkennen spezielle reguläre Sprachen
\item \ldots
\end{itemize}
Wesentliche Anwendungen von Automaten:
\begin{itemize}
\item Definition "`interessanter"' Sprachklassen
\item Lösung algorithmischer Probleme (z.B. Inklusionstest von Sprachen)
\end{itemize}

\end{frame}

\begin{frame}\frametitle{Logik}

Aussagenlogik ist nur der Anfang \ldots

\begin{itemize}
\item \alert{Prädikatenlogik/Logik erster Stufe} erweitert die Struktur atomarer Aussagen (Prädikate, Terme, Variablen, \ldots); Quantoren $\forall$ und $\exists$ ermöglichen es, sich auf viele Aussagen zu beziehen ohne alle einzeln zu nennen
\item \alert{Logik zweiter Stufe} führt zudem Variablen für Prädikate und entsprechende Quantoren ein
\end{itemize}
$\leadsto$ Ausgangspunkt vieler anwendungsspezifischer Logiken
\bigskip

Wesentliche Anwendungen:
\begin{itemize}
\item Wissensrepräsentation
\item Logikprogrammierung
\item Constraint-Erfüllungsprobleme
\item Verifikation
\end{itemize}

\end{frame}

\begin{frame}\frametitle{Logisches Schließen}

Großes Fachgebiet; sehr stark anwendungsspezifisch
\bigskip

Nennenswerte Klassen stark optimierter Logiktools:

\begin{itemize}
\item SAT-Solver: aussagenlogisches Schließen
\item Theorembeweiser: Entwickeln formaler Beweise in sehr ausdrucksstarken Logiken
\item Model-Checker: effiziente Verifikation von formalen Aussagen bzgl. abstrakter Programmmodelle
\item Logikprogramm-Systeme: Berechnung der Ergebnisse logischer Programme verschiedenster Form
\item Ontologie-Reasoner: Anfragebeantwortung über logischen Wissensbasen und Datenbanken
\end{itemize}

\end{frame}

% \begin{frame}\frametitle{}
% 
% ~\hspace{-1.6cm}
% \includegraphics[width=16.1cm]{images/fly-brain}
% 
% \end{frame}

\begin{frame}\frametitle{Komplexitätstheorie}

Großes Gebiet in der theoretischen Informatik, mit zwei wesentlichen Bedeutungen:

\begin{enumerate}[1]
\item \alert{Eigenständiges Forschungsgebiet}, das sich vielen grundlegenden Fragen widmet (einschl. $\Scomplclass{P}\neq\Scomplclass{NP}?$); Theorie der Kryptographie; Quantenkomplexität
% 
\item \alert{Methoden für andere Forschungsfelder}, welche die komplexitätstheoretische Analyse von Problemen
in vielen Fachgebieten ermöglichen; Themen wie parametrisierte Komplexität oder Ausgabekomplexität sind aus
Anwendungen motiviert
\end{enumerate}

Weiteres Fachgebiet: \redalert{Berechenbarkeitstheorie} (Klassifikation unentscheidbarer Probleme, alternative Berechnungsmodelle)

\end{frame}

% 
% Weiterführende Themen
% % Formale Sprachen (?)
% % Automatentheorie (Baumauomaten, Hybride Automaten, quantitative Modelle, ...)
% % Komplexitätstheorie (Beziehungen zwischen Komplexitätsklassen, ...)
% % Berechenbarkeitstheorie (unentscheidbare Probleme, weitere Unterteilung des Unentscheidbaren)
% % Kompilerbau
% % logikbasierte KR
% % Verifikation
% % Alternative Berechnungsmodelle: z.B. Quantumcomputing
%
% Weiterführende Vorlesungen
% % Theoretische Informatik und Logik
% % Deduction Systems
% % Advanced Logic
% % Datenbanken -- Grundlagen
% % Intelligente Systeme
% % Complexity Theory
% % Database Theory
% % Foundations of Semantic Web Technologies
% % was zu Verifikation?

\begin{frame}\frametitle{TheoLog}

Im nächsten Semester gibt es \alert{"`Theoretische Informatik und Logik"'} {\tiny(Pflicht für einige, offen für alle)}\medskip

Hören Sie die großen Fragen der Mathematik -- und die Antworten der Informatik!
\begin{itemize}
\item \redalert{Berechenbarkeit}
\begin{itemize}
\item Fleißige Biber und manch Unentscheidbares
\item Von Turingmaschinen zu Programmen
\end{itemize}
\item \redalert{Komplexität}
\begin{itemize}
\item NP: Spiele für eine Person
\item PSpace: einfache Spiele für zwei
\end{itemize}
\item \redalert{Prädikatenlogik}
\begin{itemize}
\item Die Sprache der Mathematik
\item Resolution reloaded
\item Endliche Modelle (besser bekannt als ``Datenbanken'')
\end{itemize}
\item \redalert{Mathematiker:innen als Programmierer:innen}
\begin{itemize}
\item Die Grenzen der Mathematik
\item Gödel, Turing und der ganze Rest
\end{itemize}
\end{itemize}

\end{frame}

% \sectionSlide{Fragen?}

\begin{frame}\frametitle{Zusammenfassung}

\redalert{Formale Sprachen} sind die Grundlage zahlreicher Forschungs- und Anwendungsfelder der Informatik.
\bigskip

\redalert{Berechnungsmodelle} erlauben uns, allgemeine Aussagen über die Schwere und Lösbarkeit
von Berechnungsaufgaben zu treffen
\bigskip

\redalert{Formale Logik} wird als Spezifikationssprache für (zumeist anspruchsvolle) Probleme in vielen Gebieten verwendet
\bigskip

\anybox{strongyellow}{
Offene Fragen:
\begin{itemize}
\item Haben Sie noch inhaltliche Fragen? ($\leadsto$ Konsultationen und zusätzlichen Übungstermin nutzen)
\item Haben Sie sich ausreichend auf die Prüfung vorbereitet?
\end{itemize}
}


\end{frame}

\end{document}
