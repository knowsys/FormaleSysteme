\documentclass[onlymath]{beamer}
% \documentclass[onlymath,handout]{beamer}

% Macros used by all lectures, but not necessarily by excercises

%%% General setup and dependencies:

% \usetheme[ddcfooter,nosectionnum]{tud}
\usetheme[nosectionnum,pagenum,noheader]{tud}
% \usetheme[nosectionnum,pagenum]{tud}
\setbeamertemplate{enumerate items}[default]

% Increase body font size to a sane level:
\let\origframetitle\frametitle
% \renewcommand{\frametitle}[1]{\origframetitle{#1}\normalsize}
\renewcommand{\frametitle}[1]{\origframetitle{#1}\fontsize{10pt}{13.2}\selectfont}
\setbeamerfont{itemize/enumerate subbody}{size=\small} % tud defaults to scriptsize!
\setbeamerfont{itemize/enumerate subsubbody}{size=\small}
% \setbeamerfont{normal text}{size=\small}
% \setbeamerfont{itemize body}{size=\small}

\renewcommand{\emph}[1]{\textbf{#1}}

\def\arraystretch{1.3}% Make tables even less cramped vertically

\usepackage[ngerman]{babel}
\usepackage[utf8]{inputenc}
\usepackage[T1]{fontenc}

%\usepackage{graphicx}
\usepackage[export]{adjustbox} % loads graphicx
\usepackage{import}
\usepackage{stmaryrd}
\usepackage[normalem]{ulem} % sout command
% \usepackage{times}
\usepackage{txfonts}
\usepackage{array}

% \usepackage[perpage]{footmisc} % reset footnote counter on each page -- fails with beamer (footnotes gone)
\usepackage{perpage}  % reset footnote counter on each page
\MakePerPage{footnote}

\usepackage{tikz}
\usetikzlibrary{arrows,positioning,decorations.pathreplacing}
% Inspired by http://www.texample.net/tikz/examples/hand-drawn-lines/
\usetikzlibrary{decorations.pathmorphing}
\pgfdeclaredecoration{penciline}{initial}{
    \state{initial}[width=+\pgfdecoratedinputsegmentremainingdistance,
    auto corner on length=1mm,]{
        \pgfpathcurveto%
        {% From
            \pgfqpoint{\pgfdecoratedinputsegmentremainingdistance}
                      {\pgfdecorationsegmentamplitude}
        }
        {%  Control 1
        \pgfmathrand
        \pgfpointadd{\pgfqpoint{\pgfdecoratedinputsegmentremainingdistance}{0pt}}
                    {\pgfqpoint{-\pgfdecorationsegmentaspect
                     \pgfdecoratedinputsegmentremainingdistance}%
                               {\pgfmathresult\pgfdecorationsegmentamplitude}
                    }
        }
        {%TO 
        \pgfpointadd{\pgfpointdecoratedinputsegmentlast}{\pgfpoint{1pt}{1pt}}
        }
    }
    \state{final}{}
}
\tikzset{handdrawn/.style={decorate,decoration=penciline}}
\tikzset{every shadow/.style={fill=none,shadow xshift=0pt,shadow yshift=0pt}}
% \tikzset{module/.append style={top color=\col,bottom color=\col}}

% Use to make Tikz attributes with Beamer overlays
% http://tex.stackexchange.com/a/6155
\tikzset{onslide/.code args={<#1>#2}{%
  \only<#1| handout:0>{\pgfkeysalso{#2}} 
}}
\tikzset{onslideprint/.code args={<#1>#2}{%
  \only<#1>{\pgfkeysalso{#2}} 
}}

%%% Title -- always set this first

\newcommand{\defineTitle}[3]{
	\newcommand{\lectureindex}{#1}
	\title{Formale Systeme}
	\subtitle{\href{\lectureurl}{#1. Vorlesung: #2}}
	\author{\href{https://iccl.inf.tu-dresden.de/web/Markus_Kr\%C3\%B6tzsch}{Markus Kr\"{o}tzsch}\\[1ex]Professur für Wissensbasierte Systeme}
	\date{#3}
	\datecity{TU Dresden}
% 	\institute{CC-By 3.0, sofern keine anderslautenden Bildrechte angegeben sind}
}

%%% Table of contents:

\RequirePackage{ifthen}

\newcommand{\highlight}[2]{%
	\ifthenelse{\equal{#1}{\lectureindex}}{\alert{#2}}{#2}%
}

\def\myspace{-0.7ex}
\newcommand{\printtoc}{
\begin{tabular}{r@{$\quad$}l}
\highlight{1}{1.} & \highlight{1}{Willkommen/Einleitung formale Sprachen}\\[\myspace]
\highlight{2}{2.} & \highlight{2}{Grammatiken und die Chomsky-Hierarchie}\\[\myspace]
\highlight{3}{3.} & \highlight{3}{Endliche Automaten}\\[\myspace]
\highlight{4}{4.} & \highlight{4}{Complexity of FO query answering}\\[\myspace]
\highlight{5}{5.} & \highlight{5}{Conjunctive queries}\\[\myspace]
\highlight{6}{6.} & \highlight{6}{Tree-like conjunctive queries}\\[\myspace]
\highlight{7}{7.} & \highlight{7}{Query optimisation}\\[\myspace]
\highlight{8}{8.} & \highlight{8}{Conjunctive Query Optimisation / First-Order~Expressiveness}\\[\myspace]
\highlight{9}{9.} & \highlight{9}{First-Order~Expressiveness / Introduction to Datalog}\\[\myspace]
\highlight{10}{10.} & \highlight{10}{Expressive Power and Complexity of Datalog}\\[\myspace]
\highlight{11}{11.} & \highlight{11}{Optimisation and Evaluation of Datalog}\\[\myspace]
\highlight{12}{12.} & \highlight{12}{Evaluation of Datalog (2)}\\[\myspace]
\highlight{13}{13.} & \highlight{13}{Graph Databases and Path Queries}\\[\myspace]
\highlight{14}{14.} & \highlight{14}{Outlook: database theory in practice}
\end{tabular}
}

\newcommand{\overviewslide}{%
\begin{frame}\frametitle{Overview}
\printtoc
\medskip

Siehe \href{\lectureurl}{course homepage [$\Rightarrow$ link]} for more information and materials
\end{frame}
}

%%% Colours:
\usepackage{xcolor,colortbl}
\definecolor{redhighlights}{HTML}{FFAA66}
\definecolor{lightblue}{HTML}{55AAFF}
\definecolor{lightred}{HTML}{FF5522}
\definecolor{lightpurple}{HTML}{DD77BB}
\definecolor{lightgreen}{HTML}{55FF55}
\definecolor{darkred}{HTML}{CC4411}
\definecolor{darkblue}{HTML}{176FC0}%{1133AA}
\definecolor{nightblue}{HTML}{2010A0}%{1133AA}
\definecolor{alert}{HTML}{176FC0}
\definecolor{darkgreen}{HTML}{36AB14}
\definecolor{strongyellow}{HTML}{FFE219}
\definecolor{devilscss}{HTML}{666666}

\newcommand{\redalert}[1]{\textcolor{darkred}{#1}}

%%% Style commands

\newcommand{\quoted}[1]{\texttt{"}{#1}\texttt{"}}
\newcommand{\squote}{\texttt{"}} % straight quote
\newcommand{\Sterm}[1]{\ensuremath{\mathtt{\textcolor{purple}{#1}}}}    % letters in alphabets
\newcommand{\Snterm}[1]{\textsf{\textcolor{darkblue}{#1}}} % nonterminal symbols
\newcommand{\Sntermsub}[2]{\ensuremath{\Snterm{#1}_{\Snterm{#2}}}} % nonterminal symbols
\newcommand{\Slang}[1]{\textbf{\textcolor{black}{#1}}}    % languages
\newcommand{\Slangsub}[2]{\ensuremath{\Slang{#1}_{\Slang{#2}}}}    % languages
% Code
\newcommand{\Scode}[1]{\textbf{#1}}    % reserved words in program listings, e.g., "if"
\newcommand{\Scodelit}[1]{\textcolor{purple}{#1}}    % literals in program listings, e.g., strings
\newcommand{\Scomment}[1]{\textcolor{gray}{#1}}    % comment in program listings

\newcommand{\epstrastar}{\mathrel{\mathord{\stackrel{\epsilon}{\to}}{}^*}} % transitive reflexive closure of epsilon transitions in an epslion-NFA

\newcommand{\narrowcentering}[1]{\mbox{}\hfill#1\hfill\mbox{}}

\newcommand{\Smach}[1]{\ensuremath{\mathcal{#1}}}    % machines

\newcommand{\mytrue}{\Scodelit{1}}
\newcommand{\myfalse}{\Scodelit{0}}
% \newcommand{\emptyClause}{\bot}

\newcommand{\Scomplclass}[1]{{\textsc{#1}}} % font for complexity classes, used on slides where the "too many alphabets" LaTeX error appears when using the correct sc font :-(
% \newcommand{\complclass}[1]{\ensuremath{\mathsc{#1}}} % font for complexity classes

%%% Slide layout commands:

\newcommand{\sectionSlide}[1]{
\frame{\begin{center}
\LARGE
#1
\end{center}}
}
\newcommand{\sectionSlideNoHandout}[1]{
\frame<handout:0>{\begin{center}
\LARGE
#1
\end{center}}
}

\newcommand{\mydualbox}[3]{%
 \begin{minipage}[t]{#1}
 \begin{beamerboxesrounded}[upper=block title,lower=block body,shadow=true]%
    {\centering\usebeamerfont*{block title}#2}%
    \raggedright%
    \usebeamerfont{block body}
%     \small
    #3%
  \end{beamerboxesrounded}
  \end{minipage}
}
% 
\newcommand{\myheaderbox}[2]{%
 \begin{minipage}[t]{#1}
 \begin{beamerboxesrounded}[upper=block title,lower=block title,shadow=true]%
    {\centering\usebeamerfont*{block title}\rule{0pt}{2.6ex} #2}%
  \end{beamerboxesrounded}
  \end{minipage}
}

\newcommand{\mycontentbox}[2]{%
 \begin{minipage}[t]{#1}%
 \begin{beamerboxesrounded}[upper=block body,lower=block body,shadow=true]%
    {\centering\usebeamerfont*{block body}\rule{0pt}{2.6ex}#2}%
  \end{beamerboxesrounded}
  \end{minipage}
}

\newcommand{\mylcontentbox}[2]{%
 \begin{minipage}[t]{#1}%
 \begin{beamerboxesrounded}[upper=block body,lower=block body,shadow=true]%
    {\flushleft\usebeamerfont*{block body}\rule{0pt}{2.6ex}#2}%
  \end{beamerboxesrounded}
  \end{minipage}
}

% label=180:{\rotatebox{90}{{\footnotesize\textcolor{darkgreen}{Beispiel}}}}
% \hspace{-8mm}\ghost{\raisebox{-7mm}{\rotatebox{90}{{\footnotesize\textcolor{darkgreen}{Beispiel}}}}}\hspace{8mm}
\newcommand{\examplebox}[1]{%
	\begin{tikzpicture}[decoration=penciline, decorate]
		\pgfmathsetseed{1235}
		\node (n1) [decorate,draw=darkgreen, fill=darkgreen!10,thick,align=left,text width=\linewidth, inner ysep=2mm, inner xsep=2mm] at (0,0) {#1};
% 		\node (n2) [align=left,text width=\linewidth,inner sep=0mm] at (n1.92) {{\footnotesize\raisebox{3mm}{\textcolor{darkgreen}{Beispiel}}}};
% 		\node (n2) [decorate,draw=darkgreen, fill=darkgreen!10,thick, align=left,text width=\linewidth,inner sep=2mm] at (n1.90) {{\footnotesize\raisebox{0mm}{\textcolor{darkgreen}{Beispiel}}}};
	\end{tikzpicture}%
}%

\newcommand{\codebox}[1]{%
	\begin{tikzpicture}[decoration=penciline, decorate]
		\pgfmathsetseed{1236}
		\node (n1) [decorate,draw=strongyellow, fill=strongyellow!10,thick,align=left,text width=\linewidth, inner ysep=2mm, inner xsep=2mm] at (0,0) {#1};
	\end{tikzpicture}%
}%

\newcommand{\defbox}[1]{%
	\begin{tikzpicture}[decoration=penciline, decorate]
		\pgfmathsetseed{1237}
		\node (n1) [decorate,draw=darkred, fill=darkred!10,thick,align=left,text width=\linewidth, inner ysep=2mm, inner xsep=2mm] at (0,0) {#1};
	\end{tikzpicture}%
}%

\newcommand{\theobox}[1]{%
	\begin{tikzpicture}[decoration=penciline, decorate]
		\pgfmathsetseed{1240}
		\node (n1) [decorate,draw=darkblue, fill=darkblue!10,thick,align=left,text width=\linewidth, inner ysep=2mm, inner xsep=2mm] at (0,0) {#1};
	\end{tikzpicture}%
}%

\newcommand{\anybox}[2]{%
	\begin{tikzpicture}[decoration=penciline, decorate]
		\pgfmathsetseed{1240}
		\node (n1) [decorate,draw=#1, fill=#1!10,thick,align=left,text width=\linewidth, inner ysep=2mm, inner xsep=2mm] at (0,0) {#2};
	\end{tikzpicture}%
}%


\newsavebox{\mybox}%
\newcommand{\doodlebox}[2]{%
\sbox{\mybox}{#2}%
	\begin{tikzpicture}[decoration=penciline, decorate]
		\pgfmathsetseed{1238}
		\node (n1) [decorate,draw=#1, fill=#1!10,thick,align=left,inner sep=1mm] at (0,0) {\usebox{\mybox}};
	\end{tikzpicture}%
}%
\newcommand{\widedoodlebox}[2]{%
\sbox{\mybox}{#2}%
	\begin{tikzpicture}[decoration=penciline, decorate]
		\pgfmathsetseed{1238}
		\node (n1) [decorate,draw=#1, fill=#1!10,thick,align=left,inner sep=1mm,text width=\linewidth] at (0,0) {\usebox{\mybox}};
	\end{tikzpicture}%
}%

% Common notation

\usepackage{amsmath,amssymb,amsfonts}
\usepackage{xspace}

\newcommand{\lectureurl}{https://iccl.inf.tu-dresden.de/web/FS2023}

\DeclareMathAlphabet{\mathsc}{OT1}{cmr}{m}{sc} % Let's have \mathsc since the slide style has no working \textsc

% Dual of "phantom": make a text that is visible but intangible
\newcommand{\ghost}[1]{\raisebox{0pt}[0pt][0pt]{\makebox[0pt][l]{#1}}}

\newcommand{\tuple}[1]{\langle{#1}\rangle}
\newcommand{\defeq}{\mathrel{:=}}

%%% Annotation %%%

\usepackage{color}
\newcommand{\todo}[1]{{\tiny\color{red}\textbf{TODO: #1}}}



%%% Old macros below; move when needed

\newcommand{\blank}{\text{\textvisiblespace}} % empty tape cell for TM

% table syntax
\newcommand{\dom}{\textbf{dom}}
\newcommand{\adom}{\textbf{adom}}
\newcommand{\dbconst}[1]{\texttt{"#1"}}
\newcommand{\pred}[1]{\textsf{#1}}
\newcommand{\foquery}[2]{#2[#1]}
\newcommand{\ground}[1]{\textsf{ground}(#1)}
% \newcommand{\foquery}[2]{\{#1\mid #2\}} %% Notation as used in Alice Book
% \newcommand{\foquery}[2]{\tuple{#1\mid #2}}

\newcommand{\quantor}{\mathord{\reflectbox{$\text{\sf{Q}}$}}} % the generic quantor

% logic syntax
\newcommand{\Inter}{\mathcal{I}} %used to denote an interpretation
\newcommand{\Jnter}{\mathcal{J}} %used to denote another interpretation
\newcommand{\Knter}{\mathcal{K}} %used to denote yet another interpretation
\newcommand{\Zuweisung}{\mathcal{Z}} %used to denote a variable assignment

% query languages
\newcommand{\qlang}[1]{{\sf #1}} % Font for query languages
\newcommand{\qmaps}[1]{\textbf{QM}({\sf #1})} % Set of query mappings for a query language

%%% Complexities %%%

\hyphenation{Exp-Time} % prevent "Ex-PTime" (see, e.g. Tobies'01, Glimm'07 ;-)
\hyphenation{NExp-Time} % better that than something else

% \newcommand{\complclass}[1]{{\sc #1}\xspace} % font for complexity classes
\newcommand{\complclass}[1]{\ensuremath{\mathsc{#1}}\xspace} % font for complexity classes

\newcommand{\ACzero}{\complclass{AC$_0$}}
\newcommand{\LogSpace}{\complclass{L}}
\newcommand{\NLogSpace}{\complclass{NL}}
\newcommand{\PTime}{\complclass{P}}
\newcommand{\NP}{\complclass{NP}}
\newcommand{\coNP}{\complclass{coNP}}
\newcommand{\PH}{\complclass{PH}}
\newcommand{\PSpace}{\complclass{PSpace}}
\newcommand{\NPSpace}{\complclass{NPSpace}}
\newcommand{\ExpTime}{\complclass{ExpTime}}
\newcommand{\NExpTime}{\complclass{NExpTime}}
\newcommand{\ExpSpace}{\complclass{ExpSpace}}
\newcommand{\TwoExpTime}{\complclass{2ExpTime}}
\newcommand{\NTwoExpTime}{\complclass{N2ExpTime}}
\newcommand{\ThreeExpTime}{\complclass{3ExpTime}}
\newcommand{\kExpTime}[1]{\complclass{#1ExpTime}}
\newcommand{\kExpSpace}[1]{\complclass{#1ExpSpace}}


\defineTitle{11}{Von regulären zu kontextfreien Sprachen}{17. November 2016}

\begin{document}

\maketitle

\sectionSlide{Probleme auf Automaten}

\begin{frame}\frametitle{Wiederholung}

Einige wichtige Probleme für Automaten:\bigskip

\begin{tabular}{@{}rll@{}}
\emph{Problem} & \emph{Fragestellung} & \emph{Komplexität} \\[1ex]
Leerheit & $\Slang{L}(\Smach{M})\stackrel{?}{=}\emptyset $  & polynomiell\\[1ex]
Inklusion & $\Slang{L}(\Smach{M}_1)\stackrel{?}{\subseteq}\Slang{L}(\Smach{M}_2)$  & polynomiell falls $\Smach{M}_2$ DFA\\
	& & exponentiell falls $\Smach{M}_2$ NFA\\[1ex]
Äquivalenz & $\Slang{L}(\Smach{M}_1)\stackrel{?}{=}\Slang{L}(\Smach{M}_2)$  & polynomiell falls $\Smach{M}_1$ und $\Smach{M}_2$ DFA\\
	& & exponentiell falls $\Smach{M}_1$ oder $\Smach{M}_2$ NFA
\end{tabular}

\end{frame}

\begin{frame}\frametitle{Das Wortproblem für reguläre Sprachen}

% \defbox{
% Das \redalert{Wortproblem} für DFAs über Alphabet $\Sigma$:\\[1ex]
% \emph{Eingabe:} ein DFA $\Smach{M}$ und ein Wort $w\in\Sigma^*$\\
% \emph{Ausgabe:} "`ja"' wenn $w\in\Slang{L}(\Smach{M})$ und "`nein"' wenn $w\notin\Slang{L}(\Smach{M})$
% }

\theobox{Das Wortproblem für DFAs kann in polynomieller Zeit entschieden werden.}\pause

\emph{Beweis:} Es genügt, den DFA für $|w|$ Schritte zu simulieren und zu prüfen, ob danach ein
Endzustand erreicht ist. Die Berechnung von $\delta(q,\Sterm{a})$ ist in
polynomieller Zeit möglich -- die Details hängen davon ab, wie genau $\Smach{M}$ in der Eingabe kodiert wurde.
\qed\pause

\bigskip

\theobox{Das Wortproblem für reguläre Sprachen kann in linearer Zeit $O(|w|)$ entschieden werden.}\pause

\emph{Beweis:} Man kann den DFA als gegeben annehmen, so dass die Berechnung von $\delta(q,\Sterm{a})$ in 
konstanter Zeit erfolgen kann. Die Simulation benötigt daher insgesamt $|w|$ Rechenschritte.\qed

\end{frame}

\begin{frame}\frametitle{Das Wortproblem für NFAs}

Was tun, wenn ein NFA gegeben ist?\pause
\begin{enumerate}[{Variante} 1:]
\item \alert{NFA in DFA umwandeln (Potenzmengenkonstruktion), Wortproblem für DFA lösen}\\\pause
	Exponentieller Algorithmus: Potenzmengen-DFA ist exponentiell groß\pause
\item \alert{NFA direkt mit Zustandsmengen simulieren (vergleichbar "`on-the-fly Version von Variante 1"')}\\
	\onslide<0>{Polynomieller Algorithmus: Zustandsmengen sind von linearer Größe; Berechnung der Nachfolgemenge als
	Vereinigung linear vieler $\delta$-Ergebnismengen}
\onslide<0>{
\item \alert{Konstruiere einen DFA $\Smach{M}_w$ mit $\Slang{L}(\Smach{M}_w)=\{w\}$ und prüfe ob
$\Smach{M}\cap \Smach{M}_w \neq\emptyset$}\\
	Polynomieller Algorithmus: $\Smach{M}_w$ ist linear in $|w|$; Schnittmengen-DFA ist quadratisch groß;
	Leerheitstest ist polynomiell in dieser Größe
}
\end{enumerate}

\end{frame}

\begin{frame}\frametitle{Details: Variante 2}

\codebox{
\emph{Eingabe:} NFA $\Smach{M}=\tuple{Q,\Sigma,\delta,Q_0,F}$ und Wort $w$\\
\emph{Ausgabe:} Ist $w\in\Slang{L}(\Smach{M})$?
\begin{itemize}
	\item Initialisiere $Z\defeq Q_0$
	\item Für jedes Symbol $\sigma_i$ in $w=\sigma_1\cdots\sigma_{|w|}$:\\
		Berechne $Z\defeq \bigcup_{q\in Z}\delta(q,\sigma_i)$
	\item Für alle $q\in F$:\\
		Falls $q\in Z$: das Ergebnis ist "`ja"'
	\item Falls kein $q\in F\cap Z$ gefunden wurde:  das Ergebnis ist "`nein"'
\end{itemize}
}

Alle Teilberechnungen können in polynomieller Zeit ausgeführt werden,
sofern $\Smach{M}$ "`vernünftig"' kodiert wird

\end{frame}

\begin{frame}\frametitle{Das Wortproblem für NFAs}

Was tun, wenn ein NFA gegeben ist?
\begin{enumerate}[{Variante} 1:]
\item \alert{NFA in DFA umwandeln (Potenzmengenkonstruktion), Wortproblem für DFA lösen}\\
	Exponentieller Algorithmus: Potenzmengen-DFA ist exponentiell groß
\item \alert{NFA direkt mit Zustandsmengen simulieren (vergleichbar "`on-the-fly Version von Variante 1"')}\\
	Polynomieller Algorithmus: Zustandsmengen sind von linearer Größe; Berechnung der Nachfolgemenge als
	Vereinigung linear vieler $\delta$-Ergebnismengen\pause
\item \alert{Konstruiere einen DFA $\Smach{M}_w$ mit $\Slang{L}(\Smach{M}_w)=\{w\}$ und prüfe ob
$\Slang{L}(\Smach{M})\cap \Slang{L}(\Smach{M}_w) \neq\emptyset$}\\
	\onslide<0>{Polynomieller Algorithmus: $\Smach{M}_w$ ist linear in $|w|$; Schnittmengen-DFA ist quadratisch groß;
	Leerheitstest ist polynomiell in dieser Größe}
\end{enumerate}

\end{frame}

\begin{frame}\frametitle{Details: Variante 3}

Der Automat $\Smach{M}_w$ für $w=\sigma_1\cdots\sigma_{|w|}$ ist leicht gefunden:
\bigskip

 \begin{tikzpicture}[baseline={(current bounding box.center)}]
% \draw[help lines] (0,0) grid (7,2);
\node (s0) [circle,draw=black,thick] at (0,0) {$0$};
\node (s1) [circle,draw=black,thick] at (2,0) {$1$};
\node (s2) [circle,draw=black,thick] at (4,0) {$2$};
\node (sdots) [circle,draw=white,thick] at (6,0) {$\ldots$};
\node (sn) [circle,draw=black,thick,double] at (8,0) {$|w|$};
%
\path[->,line width=0.5mm](-1,0) edge (s0);
\path[->,line width=0.5mm](s0) edge node[above] {$\sigma_1$} (s1);
\path[->,line width=0.5mm](s1) edge node[above] {$\sigma_2$} (s2);
\path[->,line width=0.5mm](s2) edge node[above] {$\sigma_3$} (sdots);
\path[->,line width=0.5mm](sdots) edge node[above] {$\sigma_n$} (sn);
\end{tikzpicture}
\bigskip

\codebox{
\emph{Eingabe:} NFA $\Smach{M}=\tuple{Q,\Sigma,\delta,Q_0,F}$ und Wort $w$\\
\emph{Ausgabe:} Ist $w\in\Slang{L}(\Smach{M})$?
\begin{enumerate}[(1)]
	\item Konstruiere $\Smach{M}_w$
	\item Berechne Produktautomat $\Smach{M}\otimes \Smach{M}_w$
	\item Entscheide ob $\Slang(\Smach{M}\otimes \Smach{M}_w)\neq\emptyset$
\end{enumerate}
}

\end{frame}

\begin{frame}\frametitle{Das Wortproblem für NFAs}

Was tun, wenn ein NFA gegeben ist?
\begin{enumerate}[{Variante} 1:]
\item \alert{NFA in DFA umwandeln (Potenzmengenkonstruktion), Wortproblem für DFA lösen}\\
	Exponentieller Algorithmus: Potenzmengen-DFA ist exponentiell groß
\item \alert{NFA direkt mit Zustandsmengen simulieren (vergleichbar "`on-the-fly Version von Variante 1"')}\\
	Polynomieller Algorithmus: Zustandsmengen sind von linearer Größe; Berechnung der Nachfolgemenge als
	Vereinigung linear vieler $\delta$-Ergebnismengen
\item \alert{Konstruiere einen DFA $\Smach{M}_w$ mit $\Slang{L}(\Smach{M}_w)=\{w\}$ und prüfe ob
$\Slang{L}(\Smach{M})\cap \Slang{L}(\Smach{M}_w) \neq\emptyset$}\\
	Polynomieller Algorithmus: $\Smach{M}_w$ ist linear in $|w|$; Schnittmengen-DFA ist quadratisch groß;
	Leerheitstest ist polynomiell in dieser Größe
\end{enumerate}

\end{frame}

\begin{frame}[t]\frametitle{Wortproblem für NFAs -- Komplexität}

\begin{enumerate}[{Variante} 1:]
\item \alert{NFA in DFA umwandeln (Potenzmengenkonstruktion), Wortproblem für DFA lösen}
\item \alert{NFA direkt mit Zustandsmengen simulieren (vergleichbar "`on-the-fly Version von Variante 1"')}
\item \alert{Konstruiere einen DFA $\Smach{M}_w$ mit $\Slang{L}(\Smach{M}_w)=\{w\}$ und prüfe ob
$\Slang{L}(\Smach{M})\cap \Slang{L}(\Smach{M}_w) \neq\emptyset$}
\end{enumerate}

Mit Variante 2 und 3 erhalten wir:\medskip

\theobox{Satz: Das Wortproblem für NFAs kann in polynomieller Zeit entschieden werden.}

\end{frame}

\begin{frame}\frametitle{Weitere Probleme für Automaten}

% Wichtig ist auch der Vergleich zweier durch Automaten gegebener Sprachen:

\defbox{
Das \redalert{Endlichkeitsproblem} für FAs über Alphabet $\Sigma$ besteht darin, die folgende Funktion zu berechnen:\\[1ex]
\emph{Eingabe:} ein FA $\Smach{M}$\\
\emph{Ausgabe:} "`ja"' wenn $\Slang{L}(\Smach{M})$ endlich ist; andernfalls "`nein"'
}\pause

Idee wie Pumping-Lemma: unendliche Sprachen erfordern Zyklus\\
$\leadsto$ suche nach Zyklen, die auf einem Pfad von einem Start- zu einem Endzustand liegen (polynomiell)\pause

\defbox{
Das \redalert{Universalitätsproblem} für FAs über Alphabet $\Sigma$ besteht darin, die folgende Funktion zu berechnen:\\[1ex]
\emph{Eingabe:} ein FA $\Smach{M}$\\
\emph{Ausgabe:} "`ja"' wenn $\Slang{L}(\Smach{M})=\Sigma^*$; andernfalls "`nein"'
}\pause

Komplement des Leerheitsproblems: $\Slang{L}(\Smach{M})=\Sigma^*$ wenn $\Slang{L}(\overline{\Smach{M}})=\emptyset$\\
$\leadsto$ Komplexität abhängig von FA-Komplementierung


\end{frame}

\begin{frame}\frametitle{Zusammenfassung}

Einige wichtige Probleme für Automaten:\bigskip

\begin{tabular}{@{}rll@{}}
\emph{Problem} & \emph{Fragestellung} & \emph{Komplexität} \\[1ex]
Leerheit & $\Slang{L}(\Smach{M})\stackrel{?}{=}\emptyset$  & polynomiell\\[1ex]
Inklusion & $\Slang{L}(\Smach{M}_1)\stackrel{?}{\subseteq}\Slang{L}(\Smach{M}_2)$  & polynomiell falls $\Smach{M}_2$ DFA\\[-0.5ex]
	& & exponentiell falls $\Smach{M}_2$ NFA\\[1ex]
Äquivalenz & $\Slang{L}(\Smach{M}_1)\stackrel{?}{=}\Slang{L}(\Smach{M}_2)$  & polynomiell falls $\Smach{M}_1$ und $\Smach{M}_2$ DFA\\[-0.5ex]
	& & exponentiell falls $\Smach{M}_1$ oder $\Smach{M}_2$ NFA\\[1ex]
Wortproblem & $w\stackrel{?}{\in}\Slang{L}(\Smach{M})$  & polynomiell\\[1ex]
Universalität & $\Slang{L}(\Smach{M})\stackrel{?}{=}\Sigma^*$  & polynomiell falls $\Smach{M}$ DFA\\[-0.5ex]
	& & exponentiell falls $\Smach{M}$ NFA\\[1ex]
Endlichkeit & $\Slang{L}(\Smach{M})$ endlich? & polynomiell 
\end{tabular}

\end{frame}

% \sectionSlide{Viereinhalb Wochen\\reguläre Sprachen}

\frame{\begin{center}
\LARGE
Viereinhalb Wochen\\reguläre Sprachen
\bigskip

\normalsize
auf sieben Folien
\end{center}}

\begin{frame}\frametitle{Die Chomsky-Hierarchie}

\defbox{
Die \redalert{Chomsky-Hierarchie} unterteilt Grammatiken in vier Stufen:
\begin{itemize}
\item \emph{Typ 0}: beliebige Grammatiken
\item \emph{Typ 1}: \redalert{kontextsensitive Grammatiken}:\\
	\begin{enumerate}[(a)]
	\item Alle Regeln $w\to v$ erfüllen die Bedingung $|w|\leq|v|$.
	\item Es gibt eine Regel $\Snterm{S}\to\epsilon$ und alle anderen Regeln $w\to v$ enthalten kein $\Snterm{S}$ in $v$ und erfüllen $|w|\leq|v|$.
	\end{enumerate}
\item \emph{Typ 2}: \redalert{kontextfreie Grammatiken}:\\
	Alle Regeln haben die Form $\Snterm{A}\to v$ für eine Variable $\Snterm{A}$.
\item \emph{Typ 3}: \redalert{reguläre Grammatiken}:\\
	Alle Regeln haben eine der Formen
	\[ \Snterm{A}\to \Sterm{c}\Snterm{B}\qquad \Snterm{A}\to \Sterm{c} \qquad \Snterm{A}\to \epsilon\]
	wobei $\Snterm{A}$ und $\Snterm{B}$ Variablen sind und $\Sterm{c}$ ein Terminalsymbol ist.
\end{itemize}

}

\end{frame}

\begin{frame}\frametitle{Chomsky's Hierarchie ist eine Hierarchie}

\begin{tikzpicture}[decoration=penciline, decorate]
\node (nl) [decorate,draw=nightblue, text depth = 4cm, text width = \linewidth, fill=nightblue!10,thick,align=left, inner ysep=2mm, inner xsep=2mm] at (0,0) {Formale Sprachen};
\node (n0) [decorate,draw=darkgreen, text depth = 3cm, text width = 8cm, fill=darkgreen!10,thick,align=left, inner ysep=2mm, inner xsep=2mm] at ([yshift=-0.5em]nl.center) {Typ-0-Sprachen};
\node (n1) [decorate,draw=strongyellow, text depth = 2cm, text width = 6.7cm,  fill=strongyellow!10,thick,align=left, inner ysep=2mm, inner xsep=2mm] at ([yshift=-0.5em]n0.center) {Kontextsensitive Sprachen (Typ 1)};
\node (n2) [decorate,draw=orange, text depth = 1cm, text width = 5.4cm,  fill=orange!10,thick,align=left, inner ysep=2mm, inner xsep=2mm] at ([yshift=-0.5em]n1.center) {Kontextfreie Sprachen (Typ 2)};
\node (n3) [decorate,draw=red, fill=red!10,thick,align=left, inner ysep=2mm, inner xsep=2mm] at ([yshift=-0.5em]n2.center) {Reguläre Sprachen (Typ 3)};
\end{tikzpicture}
% \bigskip

(Dafür mussten wir Typ-1 erweitern und $\epsilon$-Regeln bei Typ-2 eliminieren.)

\end{frame}

\begin{frame}\frametitle{Automaten}

Wir kennen mehrere \alert{Varianten endlicher Automaten}:
\begin{itemize}
\item \redalert{Deterministischer endlicher Automat (DFA)}
	\begin{itemize}
	\item mit totaler Übergangsfunktion
	\end{itemize}
\item \redalert{Nichtdeterministischer endlicher Automat (NFA)}
	\begin{itemize}
	\item mit $\epsilon$-Übergängen
	\item mit Wortübergängen
	\item mit Übergängen für reguläre Ausdrücke (nur für Umwandlung reg.\ Ausdruck $\to$ $\epsilon$-NFA)
	\end{itemize}
\end{itemize}\medskip

Die \alert{Sprache eines Automaten} haben wir auf zwei Arten definiert
\begin{itemize}
\item Mithilfe einer verallgemeinerten Übergangsfunktion, die ganze Wörter einliest
\item Durch akzeptierende Läufe, die einem Wort zugeordnet werden können
\end{itemize}

\end{frame}

\begin{frame}[fragile]\frametitle{Darstellungen von Typ-3-Sprachen}

\mbox{}\hspace{-0cm}%
\begin{tikzpicture}[
	decoration=penciline, decorate,
	node distance = 7mm and 9mm,
	mybox/.style args = {#1/#2}{
		draw=#1,% line color
		fill=#2,% fill color
% 		rounded corners,
		thick,
		text width=18mm, minimum height=12mm, inner sep=1mm,
		align=flush center
	},
	myboxlabel/.style args = {}{
		draw=devilscss,% line color
		fill=strongyellow!40,% fill color
% 		rounded corners,
		thick,
		text width=17mm, minimum height=10mm, inner sep=1.5mm,
		align=flush center
	},
	myarrow/.style args = {#1}{
		line width=0.8mm,
		draw=#1,%line color
		%-{Triangle[length=2.8mm,width=4mm,fill=#1]},
		->,
		shorten >=0.5mm, shorten <=0.1mm
	}
]
\pgfmathsetseed{7729}
% \draw[help lines] (0,0) grid (5,5);
\node (reg) [decorate,mybox=black/cyan!40] at (2,-0.6) {reguläre Grammatik};
\node (dfa) [decorate,mybox=black/cyan!40] at (0,-4) {DFA};
\node (nfa) [decorate,mybox=black/cyan!40] at (4,-4) {NFA};
\node (re) [decorate,mybox=black/cyan!40] at (6,-0.6) {regulärer Ausdruck};
\node (enfa) [decorate,mybox=black/cyan!40] at (8,-4) {$\epsilon$-NFA};
%
\path[myarrow=devilscss,bend left=20](dfa) edge (reg.180);
% \node (dfareglabel) [decorate,myboxlabel=,text width=19mm] at (-0.4,-0.7) {"`$q_1 \stackrel{\Sterm{a}}{\to} q_2$"' $\leadsto$ "`$q_1\to\Sterm{a}q_2$"'};
%
\path[myarrow=devilscss,-,bend left=20](reg.340) edge[->] (nfa.110);
\path[myarrow=devilscss,-,bend right=20](nfa.130) edge[->] (reg.320);
\node (regnfalabel) [decorate,myboxlabel=] at (1.6,-1.9) {\footnotesize"`$q_1\to\Sterm{a}q_2$"' $\Leftrightarrow$ "`$q_1 \stackrel{\Sterm{a}}{\to} q_2$"'};
%
\draw[myarrow=devilscss](dfa.10)--(nfa.170);
\node (dfanfalabel) [decorate,myboxlabel=,text width=16mm, minimum height=5mm,inner sep=1mm] at (1.95,-3.3) {\scalebox{0.8}{"`DFA${}\subseteq{}$NFA"'}};
%
\draw[myarrow=devilscss](nfa.190)--(dfa.350);
\node (nfadfalabel) [decorate,myboxlabel=,text width=13mm] at (2.1,-5.0) {\footnotesize Potenzm.\-konstr.};
%
\node (sd) [decorate,mybox=black/cyan!40,text width=14mm, minimum height=10mm] at (4,-6.5) {\footnotesize Syntax\-diagramm};
\draw[myarrow=devilscss,<->](sd)--(nfa);
\node (sdnfalabel) [decorate,myboxlabel=, minimum height=0mm,inner sep=1.0mm,text width=19mm] at (5.3,-5.4) {\footnotesize dualer Graph};
%%
\path[myarrow=devilscss,-,bend left=20](nfa.70) edge[->] (re.180);
%
\path[myarrow=devilscss,-,bend left=20](re.0) edge[->] (enfa);
%
\draw[myarrow=devilscss](enfa.190)--(nfa.350);
\draw[myarrow=devilscss](nfa.10)--(enfa.170);
\node (nfaenfalabel) [decorate,myboxlabel=,text width=16mm, minimum height=5mm,inner sep=1mm] at (5.95,-3.3) {\scalebox{0.75}{"`NFA${}\subseteq{}${}$\epsilon$-NFA"'}};
\node (enfanfalabel) [decorate,myboxlabel=,text width=13mm, minimum height=5mm] at (6.1,-4.6) {\footnotesize $\epsilon$-Elim.};
%
\node (nfarelabel) [decorate,myboxlabel=,align=flush left,text width=17mm,inner sep=1mm] at (8.6,-0.7) {\footnotesize 1) komposit.\\2) explizit};

%
\node (reenfalabel) [decorate,myboxlabel=,align=flush left,text width=18mm,inner sep=1mm] at (5.6,-2.0) {\footnotesize 1) Ersetzung\\2) Dyn. Prog.};
\end{tikzpicture}

\end{frame}

\newcommand{\simquot}[1]{#1/_{\!\!{\sim}}}

\begin{frame}\frametitle{Umformungsalgorithmen (1)}

\begin{tabular}{@{}rll@{}}
\emph{Eingabe} & \emph{Ausgabe} & \emph{V.} \\
kontextfreie Grammatik & $\epsilon$-freie kontextfreie Grammatik & 2\\
DFA $\Smach{M}$ & totaler DFA $\Smach{M}_{\text{total}}$ & 3\\
DFA $\Smach{M}$ & reguläre Grammatik $G_{\Smach{M}}$ & 3\\
Syntaxdiagramm & NFA & 4\\
NFA $\Smach{M}$ & Potenzmengen-DFA $\Smach{M}_{\text{DFA}}$ & 4\\
reguläre Grammatik $G$ & NFA $\Smach{M}_G$ & 5\\
NFA mit Wortübergängen & $\epsilon$-NFA & 5\\
$\epsilon$-NFA $\Smach{M}$ & NFA $\textsf{elim}_\epsilon(\Smach{M})$ & 5
\end{tabular}

\end{frame}

\begin{frame}\frametitle{Umformungsalgorithmen (2)}

\begin{tabular}{@{}rll@{}}
\emph{Eingabe} & \emph{Ausgabe} & \emph{V.} \\
NFAs $\Smach{M}_1$, $\Smach{M}_2$ & Vereinigungs-NFA $\Smach{M}_1\oplus\Smach{M}_2$ & 5\\
NFAs/DFAs $\Smach{M}_1$, $\Smach{M}_2$ & Produkt-NFA/DFA $\Smach{M}_1\otimes\Smach{M}_2$ & 5\\
totaler DFA $\Smach{M}$ & Komplement-DFA $\overline{\Smach{M}}$ & 5\\
NFAs $\Smach{M}_1$, $\Smach{M}_2$ & $\epsilon$-NFA $\Smach{M}_1\odot\Smach{M}_2$ für Konkatenation & 5\\
NFA $\Smach{M}$ & $\epsilon$-NFA $\Smach{M}^*$ für Kleene-Abschluss & 5\\
regulärer Ausdruck & $\epsilon$-NFA (Komposition) & 6 \\
regulärer Ausdruck & $\epsilon$-NFA (explizit) & 6 \\
NFA & regulärer Ausdruck (Gleichungssystem) & 7 \\
NFA & regulärer Ausdruck (dyn. Programmierung) & 7 \\
totaler DFA $\Smach{M}$ & Quotienten-DFA $\simquot{\Smach{M}}$ & 8 \\
totaler DFA $\Smach{M}$ & reduzierter DFA $\Smach{M}_r$ & 8
\end{tabular}

\end{frame}

\begin{frame}\frametitle{Reguläre Sprachen}

\theobox{%
Die Menge der regulären Sprachen ist \ldots

\begin{itemize}
\item die Menge genau all jener Sprachen \ldots 
	\begin{itemize}
	\item die von einer Typ-3-Grammatik beschrieben werden
	\item die von einem DFA erkannt werden
	\item die von einem NFA erkannt werden
	\item die durch einen regulären Ausdruck beschrieben werden
	\item die endlich viele Myhill-Nerode-Kongruenzklassen haben
	\end{itemize}
\item die kleinste Menge von Sprachen \ldots
	\begin{itemize}
	\item welche alle endlichen Sprachen enthält und unter $\cap$, $\cup$, $\overline{\phantom{L}}$, $\circ$ und ${}^*$ abgeschlossen ist
	\item welche die Sprachen $\emptyset$, $\{\epsilon\}$ und $\{\Sterm{a}\}$ ($\Sterm{a}\in\Sigma$) enthält und unter $\cup$, $\circ$ und ${}^*$ abgeschlossen ist
	\end{itemize}
\end{itemize}}

Alle endlichen Sprachen sind regulär (aber nicht umgekehrt)\\
Alle regulären Sprachen erlauben Pumping (aber nicht umgekehrt)

\end{frame}

\sectionSlide{Kontextfreie Sprachen}

\begin{frame}\frametitle{Kontextfreie Sprachen}

Wir hatten kontextfreie Sprachen wie folgt definiert:

\medskip
\defbox{Eine \redalert{kontextfreie Grammatik} (oder \redalert{Typ-2-Grammatik} oder \redalert{CFG}) enthält nur Regeln der Form
$\Snterm{A}\to v$, wobei $\Snterm{A}$ eine Variable ist.\medskip

Eine  Sprache ist \redalert{kontextfrei} (oder \redalert{Typ 2}), wenn sie durch eine kontextfreie Grammatik dargestellt werden kann.
}\medskip\pause

Das genügt, um nichtreguläre Sprachen darzustellen:

\examplebox{Beispiel: Die Sprache $\{\Sterm{a}^n\Sterm{b}^n\mid n\geq 0\}$ ist kontextfrei, da sie durch die folgende CFG dargestellt werden kann:
\begin{align*}
\Snterm{S} & \to \epsilon \mid \Sterm{a}\Snterm{S}\Sterm{b}
\end{align*}
{\footnotesize(Übung: Beweise, dass die Grammatik wirklich diese Sprache darstellt.)}
}

\end{frame}

\begin{frame}\frametitle{Beispiel}

CFGs eignen sich zur Darstellung vollständig geklammerter Ausdrücke.
\bigskip

\examplebox{Beispiel: Vollständig geklammerte reguläre Ausdrücke über Alphabet $\Sigma=\{\sigma_1, \ldots, \sigma_n\}$ sind als CFG über dem Alphabet $\Sigma\cup\{\Sterm{\emptyset},\Sterm{\epsilon},\Sterm{(},\Sterm{)},\Sterm{\mid},\Sterm{{}^*}\}$ darstellbar:
\begin{align*}
 \Snterm{S} & \to \Sterm{\emptyset} \mid \Sterm{\epsilon} \mid \Snterm{A} \mid \Sterm{(}\Snterm{S}\Snterm{S}\Sterm{)} \mid \Sterm{(}\Snterm{S}\Sterm{\mid}\Snterm{S}\Sterm{)}\mid \Sterm{(}\Snterm{S}\Sterm{)}\Sterm{{}^*}\\
 \Snterm{A} & \to \sigma_1\mid \ldots \mid \sigma_n
\end{align*}
}\bigskip

Allgemein ist die Beschreibung korrekt geklammerter Ausdrücke für viele Sprachen sehr wichtig, nicht zuletzt für Programmiersprachen

\end{frame}

\begin{frame}[t,fragile]\frametitle{Beispiel Compiler}

\begin{tikzpicture}[
	decoration=penciline, decorate,
	node distance = 7mm and 9mm,
	mybox/.style args = {#1/#2}{
		draw=#1,% line color
		fill=#2,% fill color
% 		rounded corners,
		thick,
		text width=47mm, minimum height=8mm, inner sep=1mm, 
		align=flush center
	},
	myarrow/.style args = {#1}{
		line width=0.8mm,
		draw=#1,%line color
		%-{Triangle[length=2.8mm,width=4mm,fill=#1]},
		->,
		shorten >=0.5mm, shorten <=0.1mm
	},
	myarrowb/.style args = {#1}{
		line width=0.3mm,
		draw=#1,%line color
		%-{Triangle[length=2.8mm,width=4mm,fill=#1]},
		->,
		shorten >=1mm, shorten <=0.1mm
	}
]
\pgfmathsetseed{7729}
% \draw[help lines] (0,0) grid (5,5);
\node (n1) [decorate,mybox=black/cyan!40] at (0,6) {Quellprogramm};
{
\node (n2) [decorate,mybox=black/white] at (0,4.5) {Lexer};
	\node (t2) [right=of n2,align=left] {\alert{lexikalische Analyse}\\erzeugt Folge von Tokens\\$\leadsto$ \redalert{reguläre Sprache}};
\node (n3) [decorate,mybox=black/white] at (0,3) {Parser};
	\node (t3) [right=of n3,align=left] {\alert{syntaktische Analyse}\\erzeugt Strukturbaum\\$\leadsto$ \redalert{kontextfreie Sprache}};
\node (n4) [decorate,mybox=black/white] at (0,1.5) {$\ldots$};
	\node (t4) [right=of n4,align=left] {Codeerzeugung,\\Optimierung,\ldots};
}
\node (n5) [decorate,mybox=black/cyan!40] at (0,0) {Programm in Zielsprache};
%
\draw[myarrow=black]    (n1) edge[decorate] (n2);
{
\draw[myarrow=black]    (n2) edge[decorate] (n3);
\draw[myarrow=black]    (n3) edge[decorate] (n4);
}
\draw[myarrow=black]    (n4) edge[decorate] (n5);
%
{
\draw[myarrowb=alert]    (t2) edge[decorate] (n2);
\draw[myarrowb=alert]    (t3) edge[decorate] (n3);
\draw[myarrowb=alert]    (t4) edge[decorate] (n4);
}
\end{tikzpicture}

\end{frame}




\begin{frame}\frametitle{Zusammenfassung und Ausblick}

% Es gibt viele \redalert{nicht-reguläre Sprachen}, was man mit verschiedenen Methoden beweisen kann (Myhill-Nerode, Abschlusseigenschaften, Pumping)
% \bigskip
% 
% \redalert{$\{\Sterm{a}^n\Sterm{b}^n\mid  n\geq 0\}$ ist nicht regulär}, da FAs nicht zählen können
% \bigskip
% 
% Algorithmische Probleme zu Automaten sind \redalert{Leerheit, Inklusion, Äquivalenz, Universalität, Endlichkeit} und das \redalert{Wortproblem}
% \bigskip
% 
% Für DFAs können sie in polynomieller Zeit gelöst werden, für NFAs dagegen zum Teil nur in exponentieller
% 
% \anybox{yellow}{
% Offene Fragen:
% \begin{itemize}
% \item Wir haben schon so viel über reguläre Sprachen gelernt -- könnten wir das bitte noch einmal alles zusammenfassen?
% \item Und was ist mit kontextfreien Sprachen?
% \end{itemize}
% }

\end{frame}


\end{document}