\begin{exercise}
  \label{logic:vampire-1}
  % Quelle: Raymond Smullyan: The Lady or the Tiger. Alfred A. Knopf, 1986.
  Inspektor Craig wird zu einem wichtigen Fall nach Transsilvanien gerufen, in
  dem es darum geht herauszufinden, welcher von zwei Beschuldigten ein Vampir
  ist.  Wie allgemein bekannt ist, besteht die Bevölkerung in Transsilvanien zum
  Teil aus Menschen und zum anderen Teil aus Vampiren.  Die Menschen sagen stets
  die Wahrheit, die Vampire lügen stets.  Außerdem ist ein Teil der Bevölkerung
  Transsilvaniens verrückt: Alles, was wahr ist, glauben sie, sei falsch, und
  umgekehrt.  Nicht verrückte Transsilvanier hingegen halten genau das für wahr,
  was wahr ist.  Insbesondere sagt ein verrückter Vampir (wie auch ein
  nicht-verrückter Mensch) stets das Richtige: Eine Aussage, die wahr ist,
  glaubt er, sei falsch, da er aber stets lügt, gibt er dennoch eine richtige
  Antwort.

  Craig verhört die zwei Beschuldigten Lucy und Minna, von denen bekannt ist,
  dass eine ein Vampir ist und die andere nicht.  Das Verhör geht wie folgt
  vonstatten:
  \begin{quote}
    \emph{Craig} (zu Lucy): Erzählen Sie mir von Ihnen.\\
    \emph{Lucy}: Wir sind beide verrückt.\\
    \emph{Craig} (zu Minna): Ist das richtig?\\
    \emph{Minna}: Natürlich nicht!
  \end{quote}
  Formalisieren Sie dieses Szenario mithilfe aussagenlogischer Formeln und
  finden Sie heraus, wer der Vampir ist!

  \vskip+1\baselineskip

  {\footnotesize Diese Aufgabe stammt aus: \emph{Raymond Smullyan: The Lady or the
      Tiger?.  Alfred A.\@ Knopf, 1986.}}
\end{exercise}
