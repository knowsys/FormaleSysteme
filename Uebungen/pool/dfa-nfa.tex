
\begin{exercise}
\vspace*{0.1cm}
\begin{itemize}
\item[a)] Erklären Sie, wann zwei NFAs  $\mathcal{M}_1$ und $\mathcal{M}_2$ äquivalent sind. 
\item[b)] Geben Sie einen DFA  $\mathcal{M}'$ an, der zum NFA  $\mathcal{M}=(\{q_0,q_1,q_2\},\{a,b\},\delta, \{q_0\}, \{q_2\})$ äquivalent ist;
  für $\mathcal{M}$ ist die Übergangsfunktion $\delta$ grafisch angegeben:\\[0.5cm]
\begin{center}
  
\begin{exercise}
\vspace*{0.1cm}
\begin{itemize}
\item[a)] Erklären Sie, wann zwei NFAs  $\mathcal{M}_1$ und $\mathcal{M}_2$ äquivalent sind. 
\item[b)] Geben Sie einen DFA  $\mathcal{M}'$ an, der zum NFA  $\mathcal{M}=(\{q_0,q_1,q_2\},\{a,b\},\delta, \{q_0\}, \{q_2\})$ äquivalent ist;
  für $\mathcal{M}$ ist die Übergangsfunktion $\delta$ grafisch angegeben:\\[0.5cm]
\begin{center}
  
\begin{exercise}
\vspace*{0.1cm}
\begin{itemize}
\item[a)] Erklären Sie, wann zwei NFAs  $\mathcal{M}_1$ und $\mathcal{M}_2$ äquivalent sind. 
\item[b)] Geben Sie einen DFA  $\mathcal{M}'$ an, der zum NFA  $\mathcal{M}=(\{q_0,q_1,q_2\},\{a,b\},\delta, \{q_0\}, \{q_2\})$ äquivalent ist;
  für $\mathcal{M}$ ist die Übergangsfunktion $\delta$ grafisch angegeben:\\[0.5cm]
\begin{center}
  
\begin{exercise}
\vspace*{0.1cm}
\begin{itemize}
\item[a)] Erklären Sie, wann zwei NFAs  $\mathcal{M}_1$ und $\mathcal{M}_2$ äquivalent sind. 
\item[b)] Geben Sie einen DFA  $\mathcal{M}'$ an, der zum NFA  $\mathcal{M}=(\{q_0,q_1,q_2\},\{a,b\},\delta, \{q_0\}, \{q_2\})$ äquivalent ist;
  für $\mathcal{M}$ ist die Übergangsfunktion $\delta$ grafisch angegeben:\\[0.5cm]
\begin{center}
  \input{pool/graphics/dfa-nfa}
\end{center}
\end{itemize}
\end{exercise}


\end{center}
\end{itemize}
\end{exercise}


\end{center}
\end{itemize}
\end{exercise}


\end{center}
\end{itemize}
\end{exercise}

