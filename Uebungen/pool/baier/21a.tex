\begin{exercise}

Wir betrachten die folgenden beiden Problemstellungen:

\begin{itemize}
\item Das Job-Rechnerproblem:
Gegeben sind $n$ Rechner $R_1,\ldots,R_n$ 
und $m$ Jobs $J_1,\ldots,J_m$.
Weiter liegt eine Tabelle 
$$T = (t_{i,k})_{1 \le i \le n, 1 \le k \le m}
$$ 
vor, in der eingetragen ist, welcher Rechner 
f\"ur welchen Job die notwendige Ausstattung besitzt.
Dabei ist $t_{i,k} = 1$, falls der $k$-te Job $J_k$ auf Rechner $R_i$
ausf\"uhrbar ist. Andernfalls ist $t_{i,k} = 0$.
Gefragt ist, ob es eine Zuordnung
%\begin{center}
  Jobs $\longrightarrow$ Rechner
%\end{center}
gibt, so da{\ss} die $m$ Jobs gleichzeitig ausgef\"uhrt werden
k\"onnen.

Wir nehmen dabei an, da{\ss} kein Rechner zwei oder mehr Jobs
gleichzeitig ausf\"uhren kann.

\item Matchingproblem (einfache Variante f\"ur bipartite Graphen):
Gegeben ist ein ungerichteter Graph $G = (V,E)$ und eine Partition
$V = V_L \cup V_R$ mit disjunkten nichtleeren Mengen $V_L$ und $V_R$,
so da{\ss}  
jede Kante zwischen $V_L$ und $V_R$ verl\"auft; 
d.h.~f\"ur jede Kante $e \in E$ gilt:
\begin{center}
$ e \cap V_L \not= \emptyset$ und
$ e \cap V_R \not= \emptyset$ 
\end{center}
Gefragt ist, ob es eine Kantenmenge $M \subseteq E$ gibt, so da{\ss}
jeder Knoten $v \in V$ auf h\"ochstens einer Kante von $M$ liegt und
$|M| = |V_L|$.
\end{itemize}
Zeigen Sie, da{\ss} das Job-Rechnerproblem auf das 
Matchingproblem
reduzierbar ist.

\end{exercise}
