\begin{exercise}

Wir betrachten das Handlungsreisendenproblem (TSP).
Gegeben ist ein Digraph $G = (V,E)$ mit einer Kostenfunktion $c : E \to \Nat.$

\begin{itemize}
  \item Die Optimierungsvariante vom Typ 2 fragt  nach einer k\"urzesten Rundreise. 
  \item Die Optimierungsvariante vom Typ 1 fragt  nach den Kosten einer k\"urzesten
Rundreise.
  \item Die Entscheidungsvariante hat neben $G$ und $c$ eine Zahl $t \in \Nat$ als Eingabe und fragt, ob es eine Rundreise mit den Kosten $\le t$ gibt.
\end{itemize}
Zeigen Sie, dass aus der effizienten L\"osbarkeit\footnote{Mit ``effizienter L\"osbarkeit'' ist die Existenz eines
polynomiell zeitbeschr\"ankten Algorithmus gemeint.
} einer der drei Varianten  
die effiziente L\"osbarkeit der beiden anderen Varianten folgt.

{\bfseries Hinweis}:
Eine Rundreise ist ein Hamiltonkreis, d.h. ein Kreis $v_0v_1\ldots v_n v_0$ in $G$, wobei $n = |V|$ gilt und $v_0, \ldots, v_n$ paarweise verschieden sind. Ein solcher Zyklus durchl\"auft also alle Knoten in $G$ genau einmal.
\end{exercise}

