\begin{exercise}
\begin{enumerate}
\item[S9)] Gegeben sind die folgenden Grammatiken $G_i$ mit $1\le i\le 4$:
    \begin{itemize}
      \item[] $G_1=(\{S\},\{a,b\},\{S\rightarrow aS,S\rightarrow Sb,S\rightarrow a\},S)$
      \item[] $G_2=(\{S\},\{a,b\},\{S\rightarrow aS,S\rightarrow SbS,S\rightarrow a\},S)$
      \item[] $G_3=(\{S,B\},\{a,b\},\{S\rightarrow \varepsilon,S\rightarrow aSb,aS\rightarrow aB, B\rightarrow bB, B\rightarrow b\},S)$
      \item[] $G_4=(\{S,A\},\{a,b\},\{S\rightarrow a,A\rightarrow b\},S)$
    \end{itemize}
   Geben Sie f\"ur jede Grammatik $G_i$ den maximalen Chomsky-Typ $j$ an. Begr\"unden Sie Ihre Antwort.
\item[S10)] Welche der folgenden Aussagen sind wahr? Begr\"unden Sie Ihre Antwort.
   \begin{enumerate}
       \item[a)] F\"ur den regul\"aren Ausdruck $\alpha = (b(ab\mid b)^*)^*(a\mid b)^*a$ gilt: $aba \in L(\alpha)$.
       \item[b)] F\"ur die Grammatik $G=(\{S,X,Y,Z\},\{a,b\},\{S\rightarrow
         Y,\;X\rightarrow b,\;Y\rightarrow
         aYYb,aY\rightarrow aZ,\;ZY\rightarrow ZX,\;Z\rightarrow a\},S)$ gilt: $aabb\in L(G)$.
   \end{enumerate}
 \end{enumerate}
\end{exercise}
