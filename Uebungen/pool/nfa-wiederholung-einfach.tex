\begin{exercise}
  \begin{enumerate}
  \item[S5)] Es sei der $\varepsilon$-NFA
    $\mathcal M=(\{q_0,\ldots ,q_4\},\{a,b\}, \delta , \{q_0\}, \{q_2\})$
    gegeben mit $\delta$ wie unten graphisch dargestellt:\\[1ex]
    \begin{center}
      
\begin{tikzpicture} [->, >=stealth', initial text=, auto, node
  distance=23mm, bend angle=20, semithick]%[node distance=2cm,auto]
 \node[state,initial] (q_0) {$q_0$}; 
 \node[state] (q_1) [above right of=q_0] {$q_1$}; 
 \node[state,accepting] (q_2) [right of=q_1] {$q_2$}; 
 \node[state] (q_3) [below of=q_1] {$q_3$};
 \node[state] (q_4) [right of=q_3] {$q_4$};
 \path[->]
  (q_0) edge node {$a$} (q_1) 
  (q_0) edge node {$b$} (q_3) 
  (q_1) edge node {$a, b$} (q_3) 
  (q_1) edge node {$\varepsilon$} (q_2) 
  (q_2) edge node {$a$} (q_3) 
  (q_2) edge node {$a$} (q_4) 
  (q_3) edge node {$b$} (q_4)
  (q_4) edge [loop right] node {$a$} (q_4) ;
\end{tikzpicture}
    \end{center}
    Konstruieren Sie einen zu $\mathcal M$ äquivalenten DFA $\mathcal M'$.
  \item[S6)] Es sei $\Sigma = \{a,b,c\}$. Geben Sie NFAs ${\mathcal{M}}_1$,
    ${\mathcal{M}}_2$ an mit
    \begin{enumerate}
    \item $L({\mathcal{M}}_1)=\{w\in \Sigma^* \mid (|w|_a \;\text{ist ungerade und}\;|w|_b \;\text{ist gerade}) \;\text{oder}$\\[0.5ex]
      \hspace*{3.245cm}$(\text{es gibt }u,v\in \Sigma^* \;\text{mit} \;w = u ccc v) \}$
    \item $L({\mathcal{M}}_2) =  \{w\in \Sigma^* \mid (\text{es gibt} \;u,v\in \Sigma^* \;\text{mit}\; w = u babc v)\;\text{und}$\\[0.5ex]
      \hspace*{3.28cm}$(\text{es gibt }u,v\in \Sigma^* \;\text{mit}\; w = u ccc v ) \;\text{und}$\\[0.5ex]
      \hspace*{3.3cm}$(\text{es gibt kein}\;u \in \Sigma^* \;\text{mit}\; w = au)\}$
    \end{enumerate}
  \end{enumerate}
\end{exercise}
