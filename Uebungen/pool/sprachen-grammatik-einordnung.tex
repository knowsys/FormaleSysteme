
\begin{exercise}
Sei $\Sigma=\{a,b\}$. Gegeben sind die Grammatiken $G_k$ mit $1\le k\le 4$:
\begin{itemize}
\item $G_1=\tuple{V_1,\Sigma,P_1,S_1}$ mit
      $V_1=\{S,T\}$ und \\
      $P=\{S\rightarrow aT,\;S\rightarrow \varepsilon, T\rightarrow Sb\}$
\item $G_2=\tuple{V_2,\Sigma,P_2,S_2}$ mit
      $V_2=\{S,A,B\}$ und \\
      $P_2=\{S\rightarrow SAS,\;S\rightarrow SBBS,\;S\rightarrow
      \varepsilon,\;A\rightarrow a,\;B\rightarrow
      b\}$
\item $G_3=\tuple{V_3,\Sigma,P_3,S_3}$ mit
      $V_3=\{S,A,B\}$ und \\
      $P_3=\{S\rightarrow A,\;S\rightarrow \varepsilon,\;A\rightarrow
      ab,\;A\rightarrow aBb,\;aB\rightarrow
      aaBb,\; aB\rightarrow a\}$
\item $G_4=\tuple{V_4,\Sigma,P_4,S_4}$ mit
      $V_4=\{S,T\}$ und \\
      $P_4=\{S\rightarrow aSb,\;S\rightarrow aTb,\;S\rightarrow \varepsilon,\;aTb\rightarrow
      T,\;aTb\rightarrow S\}$.
\end{itemize}
\begin{enumerate}
\item[a)] Geben Sie zu jeder dieser Grammatiken $G_k$ das maximale $i$ an, so dass $G_k$ eine Grammatik vom Typ~$i$ ist.
\item[b)] Beschreiben Sie $L(G_k)$ in einer geeigneten Form, z.B. mittels einer Mengennotation.  
\end{enumerate}
\end{exercise}
