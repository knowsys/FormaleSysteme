\begin{exercise}
  \label{logic:formula-size}
  Für eine Formel $F$ ist die Größe \size{F} definiert durch:
  \begin{align*}
    \size{p} &:= 1 \\
    \size{\lnot G} &:= \size{G} + 1 \\
    \size{(G_1 \vee G_2)} &:= \size{G_1} + \size{G_2} + 1 \\
    \size{(G_1 \wedge G_2)} &:= \size{G_1} + \size{G_2} + 1 \\
    \size{(G_1 \rightarrow G_2)} & := \size{G_1} + \size{G_2} + 1 \\
    \size{(G_1 \leftrightarrow G_2)} & := \size{G_1} + \size{G_2} + 1,
  \end{align*}
  wobei $G_1$ und $G_2$ Formeln sind und $p \in {\mathcal{P}}$ ist.  Zeigen Sie
  die folgenden Aussagen:
  \begin{enumerate}[label={\alph*)}]
  \item Die Anzahl der Variablen in $F$ ist beschränkt durch $\size{F}$.
  \item Die Anzahl der Unterformeln in $F$ ist beschränkt durch $\size{F}$.
  \end{enumerate}
\end{exercise}
